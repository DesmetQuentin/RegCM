%
% This file is part of ICTP RegCM model.
% Copyright (C) 2011 ICTP Trieste
% See the file COPYING for copying conditions.
%

RegCM-4.1 is a new step in recoding the RegCM3 model after the effort put into
the RegCM4.0 version. The code base now is actively developed by a community
of developers internal and external to ICTP, and this work is merged on the
eforge site on e-science-lab.org site.

The main new technical features of the code are summarized in the following
points

\begin{itemize}
\item Dynamic memory allocation
\item F95 programming language
\item Module approach
\item Unique makefile for the whole code from pre-processing to post-processing
\item netCDF I/O format from model components following the CF-1.4 standard
\end{itemize}

The model code is in Fortran 90 ANSI standard with some language extensions of
Fortran 2003 implemented in all the supported compilers.
The development is done on Linux boxes, and the model is known to run
on Oracle Solaris\texttrademark platforms, IBM AIX\texttrademark platforms,
MacOS\texttrademark platforms.
No porting effort has been done towards non Unix-like Operating Systems.
We will for this User Guide assume that the reference platform is a recent
Linux distributon with a \verb=bash= shell.
Tipographical convention is the following:

\begin{table}[ht]
\caption{Conventions}
\vspace{0.05 in}
\centering
\begin{tabular}{l|l}
\hline
\verb=$> = & normal shell prompt \\
\verb=#> = & root shell prompt \\
\verb=$SHELL_VARIABLE = & a shell variable \\
\hline
\end{tabular}
\label{conventions}
\end{table}

Any shell variable is supposed to be set by the User with the following example
syntax:

\begin{Verbatim}
$> export REGCM_ROOT="/home/graziano/RegCM4.1"
\end{Verbatim}

Hope you will find this document useful. Any error found belongs to me and can
be reported to be corrected in future revisions. Enjoy.
