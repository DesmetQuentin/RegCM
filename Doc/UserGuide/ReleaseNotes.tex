%
% This file is part of ICTP RegCM model.
% Copyright (C) 2011 ICTP Trieste
% See the file COPYING for copying conditions.
%

RegCM-4.3 is a new step in recoding the RegCM model after the effort put into
the RegCM4 version. The code base now is actively developed by a community
of developers internal and external to ICTP, and this work is merged on the
Gforge site on gforge.ictp.it site.

The main new technical features of the code are summarized in the following
points:

\begin{itemize}
  \item A number of new global models input layers have been added to preproc
    stage, and we plan to support any request to add a new layer.
    What is needed is just send us a couple of months worth of data, specify
    format if not netCDF, fix a naming convention. Just ask, we will help you.
    Now supported at SST stage:
  \begin{enumerate}
    \item GISST Met Office's Global Ice coverage and Sea Surface Temperatures
     This data required a pre-processing (from ASCII to binary), and are
     available from ICTP up to 2002 from clima-dods server
    \item NOAA Optimal Interpolation SST dataset
    \item ECMWF Era Interim SST dataset
    \item CMIP5 global ocean model datasets:
      \begin{enumerate}
        \item Canadian Centre for Climate Modelling and Analysis CanESM2
        \item Met Office Hadley Centre HadGEM2-ES
        \item Commonwealth Scientific and Industrial Research Organization Mk3.6
        \item EC-EARTH consortium (need conversion from GRIB to netCDF)
        \item Institut Pierre-Simon Laplace CM5A-MR
        \item NOAA Geophysical Fluid Dynamics Laboratory ESM2M
        \item Centre National de Recherches Météorologiques CM5 model
        \item Max-Planck-Institut für Meteorologie MPI-ESM-MR
      \end{enumerate}
    \item ECHAM 5 dataset (A1B in binary format from ICTP, A2 converted from
      GRIB
    to netCDF using cdo)
    \item 33 years Era Interim average available from ICTP
  \end{enumerate}
  Code for FV model and some CAM 2/4 is present, but is obsolete or untested.
  Supported at ICBC stage:
  \begin{enumerate}
    \item ERA Interim and ERA 40 reanalysis
    \item NNRP reanalysys (V1 and V2)
    \item CMIP5 global atmosphere model datasets:
    \begin{enumerate}
      \item Canadian Centre for Climate Modelling and Analysis CanESM2
      \item Met Office Hadley Centre HadGEM2-ES
      \item Commonwealth Scientific and Industrial Research Organization Mk3.6
      \item EC-EARTH consortium (need conversion from GRIB to netCDF)
      \item Institut Pierre-Simon Laplace CM5A-MR
      \item NOAA Geophysical Fluid Dynamics Laboratory ESM2M
      \item Centre National de Recherches Météorologiques CM5 model
      \item Max-Planck-Institut für Meteorologie MPI-ESM-MR
    \end{enumerate}
    \item ECHAM 5 dataset (A1B in binary format from ICTP, A2 converted from
      GRIB to netCDF using cdo)
    \item 33 years Era Interim average available from ICTP
  \end{enumerate}
  Code for FV model and some CAM 2/4 is present, but is obsolete or untested.
  \item Boundary conditions / emissions for tracers both aerosols and chemically
   active creation from different sources
  \item 2D model decomposition is fully tested and model scales easily up to 512
   processors what was scaling previously up to 64 processors.
  \item RCP standard 2.6, 4.5, 6.0 and 8.5 Greenhouse gas concentration data for
   historical and future scenarios.
  \item A new option to use measured total solar irradiance from SOLARIS instead
   of the previously used hardcoded solar constant.
  \item Climatological and future 2.6, 4.5 and 8.5 scenario data for ozone
   concentration.
  \item Experimental new empirical cumulus cloud representation option.
  \item In the model output each file tipology can be disabled, and in each file
   each output 2D/3D variable can be disable. The default is to write all the
   defined variables, but the user may play with the i\verb=enable_xxx_vars=
   flags to eventually disable variables.
   Some variables cannot be disabled: xlat , xlon , mask , topo , ps.
   Those are needed for model geolocation.
  \item Model input is read now in parallel by all processors, i.e. a complete
   diskless model for computing nodes is not supported: each processor needs
   access to the input data directory.
  \item Model output can be written in parallel by the model taking advantage
    of a parallel file system and the mpiio capability of the HDF5 library.
    Model must be compiled with the \verb=--enable-nc4-parallel= flag and the
    namelist option \verb=do_parallel_netcdf_io= must be set to true.
    As to take advantage from this a particular harware/software stack is
    needed, this is NOT the default option.
  \item Model output files contain when applicable geolocation informations
   following the CF-1.6 convention. IDV can now use this informations, except
   for the rotated mercator projection which is unsupported in the java IDV
   library.
  \item Model output variable naming convention and unit of measure shouldi
    respect the CORDEX experiment guidelines. User should report any
    non-conformity.
  \item The Kallen 1996 algorithm is used to compute Mean Sea Level pressure.
  \item A Gauss-Seidel smoothing has been applied to the Mean Sea Level and the
   Geopotential Height post-processing calculation subroutines.
\end{itemize}
Bug Fixing:
\begin{itemize}
  \item Cressman type interpolation is used for all Gaussian grids in SST/ICBC
  \item Added extra stratospheric layers in the RRTM radiation model.
  \item Model input/output of netCDF files is consistent at all stages using a
   common library.
  \item Fix for tracer gases optical properties in the visible spectra
  \item Relaxing upstream scheme for cloud disabled as it may need reworking.
  \item The GrADSNc programs now is able to guess when the data are monthly
    averages calculated with cdo and use 1mo as timestep in the ctl file
  \item All model output is now using the write intrinsic instead of print.
\end{itemize}

Planned Development for V 4.4 :

\begin{itemize}
  \item Mandatory add interface for Community Earth System Model dataset
  \item The new semi-implicit moisture scheme is almost completed.
  \item Input interface for snow/ice informations from measured datasets/models.
  \item The Tiedtke cumulus scheme will be upgraded to ECMWF Version 38r2
  \item A new slab ocean model option to enhance fluxes representation over
    ocean
  \item Use the netCDF format also for the SAV files.
\end{itemize}

Next release V 5.X :

\begin{itemize}
  \item New dynamical core from Giovanni Tumolo semi-implicit, semi-Lagrangian,
   p-adaptive Discontinuous Galerkin method three dimensional model.
\end{itemize}

The model code is in Fortran 2003 ANSI standard.
The development is done on Linux boxes, and the model is known to run
on Oracle Solaris\texttrademark platforms, IBM AIX\texttrademark platforms,
MacOS\texttrademark platforms.
No porting effort has been done towards non Unix-like Operating Systems.
We will for this User Guide assume that the reference platform is a recent
Linux distributon with a \verb=bash= shell.
Typographical convention is the following:

\begin{table}[ht]
\caption{Conventions}
\vspace{0.05 in}
\centering
\begin{tabular}{l|l}
\hline
\verb=$> = & normal shell prompt \\
\verb=#> = & root shell prompt \\
\verb=$SHELL_VARIABLE = & a shell variable \\
\hline
\end{tabular}
\label{conventions}
\end{table}

Any shell variable is supposed to be set by the User with the following example
syntax:

\begin{Verbatim}
$> export REGCM_ROOT="/home/user/RegCM4.3.5"
\end{Verbatim}

Hope you will find this document useful. Any error found belongs to me and can
be reported to be corrected in future revisions. Enjoy.
