%
% This file is part of ICTP RegCM model.
% Copyright (C) 2011 ICTP Trieste
% See the file COPYING for copying conditions.
%

We will examin in this chapter in more detail the namelist configuration file,
to give You the User a deeper knowledge of model capabilities.

\section{The commented namelist}

In this section we will show You the commented namelist input file You will
find under \verb=$REGCM_ROOT/Doc= with the name \verb=README.namelist= .
All model programs seen so far, with the exception of the GrADS helper program,
use as input this namelist file, which is unique to a particular simulation.
The model input namelist file is divided in stanzas, each one devoted to
configuring the model capabilities.
A stanza in the namelist is identified with a starting \verb=&= character
followed by stanza name, and ends on a single line with the \verb=\=
character.

\subsection{dimparam stanza}

This stanza contains the base X,Y,Z domain dimension information, used
by the model dynamic memory allocator to request the Operating System the
memory space to store the model internal variables.

\footnotesize{
\begin{Verbatim}
 &dimparam
 iy     = 34,  ! This is number of points in the N/S direction
 jx     = 48,  ! This is number of points in the E/W direction
 kz     = 18,  ! Number of vertical levels: supported are 14, 18 and 23
 nsg    = 1,   ! For subgridding, number of points to decompose. If nsg=1,
               ! no subgridding is performed. CLM does NOT work as of now with
               ! subgridding enabled.
 /
\end{Verbatim}
}

The things You need to know here:

\begin{enumerate}
\item In the current version 4.1 the model parallelize execution dividing
the work between the processors along the jx (longitude) dimension. The
minimum work per processor is 3 points along the jx dimension, so the maximum
number of processors which can be used in a parallel run for the above
configuration is just $16$. In future revision ICTP plans to introduce 2D
decomposition.
\item In the future model revision the kz (vertical) dimension will be
more configurable, but for now you are limited to have 14, 18 and 23 levels.
We at ICTP normally use 18 levels.
\item Specifying an nsg number greater than one triggers the subgrid BATS model
on. There is no plan to extend this feature to CLM model. This affects only
surface variable calculations. All dynamics variable are calculated still
on the coarser grid. Rain in the current implementation is also calculated on
the coarser grid.
\end{enumerate}

\subsection{geoparam stanza}

This stanza is used by the \verb=terrain= program to geolocate the model grid
on the earth surface. The RegCM model uses a limited number of projection
engines. The value here are used by the other model programs to assert
consistency with the geolocation information written by the \verb=terrain=
program in the \verb=DOMAIN= file.

\footnotesize{
\begin{Verbatim}
 &geoparam
 iproj = 'LAMCON', ! Domain cartographic projection. Supported values are:
                   ! 'LAMCON', Lambert conformal.
                   ! 'POLSTR', Polar stereographic. (Doesn't work)
                   ! 'NORMER', Normal  Mercator.
                   ! 'ROTMER', Rotated Mercator.
 ds = 60.0,        ! Grid point horizontal resolution in km
 ptop = 5.0,       ! Pressure of model top in cbar
 clat = 45.39,     ! Central latitude  of model domain in degrees
                   ! North hemisphere is positive
 clon = 13.48,     ! entral longitude of model domain in degrees
                   ! West is negative.
 plat = 45.39,     ! Pole latitude (only for rotated Mercator Proj)
 plon = 13.48,     ! Pole longitude (only for rotated Mercator Proj)
 truelatl = 30.0,  ! Lambert true latitude (low latitude side)
 truelath = 60,    ! Lambert true latitude (high latitude side)
 /
\end{Verbatim}
}

The things You need to know here:

\begin{enumerate}
\item The different projection engines produce better results depending on the
position and extent of the domain. In particular, regardless of emisphere:
\begin{itemize}
\item Middle latitudes (around 45 degrees) - Lambert Conformal
\item Polar latitudes (more than 75 degrees) - Polar Stereographic
\item Low latitudes (up to 30 degrees and crossing the equator) - Mercator
\item Crossing more than 45 degrees extent in latitude - Rotated Mercator
\end{itemize}
\item The model hydrostatic engine does not allow resolution lower than
$20 km$. If want higher resolution consider using the subgridding scheme.
ICTP plans to introduce in the future a non-hydrostatic compressible core to
the RegCM model.
\item Lowering the top pressure of the model can give You problems in regions
with complex topography. Touch the default after thinking twice on that.
\item Always specify \verb=clat= and \verb=clon=, the central domain point,
and do fine adjustment of the position moving it around a little bit. A
little shift in position and some tests can help You obtain a better
representation of coastlines and topography at the coarse resolutions.
\item If using \verb=LAMCON= projection, take care to place the two
true latitudes at around one fourth and three fourth of the domain latitude
space to better correct the projection distorsion of the domain.
\item The pole position for the rotated mercator position should be as near as
possible to the center domain position.
\end{enumerate}

\subsection{terrainparam stanza}

This stanza is used by the \verb=terrain= program to geolocate the model grid
on the earth surface. The RegCM model uses a limited number of projection
engines. The value here are used by the other model programs to assert
consistency with the geolocation information written by the \verb=terrain=
program in the \verb=DOMAIN= file.

