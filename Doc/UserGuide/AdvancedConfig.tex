%
% This file is part of ICTP RegCM model.
% Copyright (C) 2011 ICTP Trieste
% See the file COPYING for copying conditions.
%

We will examine in this chapter in more detail the namelist configuration file,
to give you the User a deeper knowledge of model capabilities.

\section{The commented namelist}

In this section we will show you the commented namelist input file you will
find under \verb=$REGCM_ROOT/Doc= with the name \verb=README.namelist= .
All model programs seen so far, with the exception of the GrADS helper program,
use as input this namelist file, which is unique to a particular simulation.
The model input namelist file is divided in stanzas, each one devoted to
configuring the model capabilities.
A stanza in the namelist is identified with a starting \verb=&= character
followed by stanza name, and ends on a single line with the \verb=\=
character.

\subsection{dimparam stanza}
\label{dimparam}

This stanza contains the base X,Y,Z domain dimension information, used
by the model dynamic memory allocator to request the Operating System the
memory space to store the model internal variables.

{\footnotesize
\begin{Verbatim}
 &dimparam
 iy     = 34,   ! This is number of points in the N/S direction
 jx     = 48,   ! This is number of points in the E/W direction
 kz     = 18,   ! Number of vertical levels
 dsmin  = 0.01, ! Minimum sigma spacing (only used if kz is not 14, 18, or 23)
 dsmax  = 0.05, ! Maximum sigma spacing (only used if kz is not 14, 18, or 23)
 nsg    = 1,    ! For subgridding, number of points to decompose. If nsg=1,
                ! no subgridding is performed. CLM does NOT work as of now with
                ! subgridding enabled.
 /
\end{Verbatim}
}

The things you need to know here:

\begin{enumerate}
\item In the current version 4.2 the model parallelizes execution dividing
the work between the processors along the jx (longitude) dimension. The
minimum work per processor is 3 points along the jx dimension, so the maximum
number of processors which can be used in a parallel run for the above
configuration is just $16$. In future revision ICTP plans to introduce 2D
decomposition.
\item If a custom number of sigma level is chosen (not 14, 18 or 23), the actual
sigma values are calculated mimimizing the $a,b$ coefficients for the 
equation:

\begin{equation}
  dsig(i) = dsmax*a^{i-1}*b^{0.5*(i-2)*(i-1)}
\end{equation}

derived from the recursive relation:

\begin{equation}
  dsig(i) = a(i)*dsig(i-1)
\end{equation}

where $a(i) = b*a(i-1)$. We at ICTP normally use 18 levels.
\item Specifying an nsg number greater than one triggers the subgrid BATS model
on. There is no plan to extend this feature to CLM model. This affects only
surface variable calculations. All dynamical variables are calculated still
on the coarser grid. Rain in the current implementation is also calculated on
the coarser grid.
\end{enumerate}

\subsection{geoparam stanza}
\label{geoparam}

This stanza is used by the \verb=terrain= program to geolocate the model grid
on the earth surface. The RegCM model uses a limited number of projection
engines. The value here are used by the other model programs to assert
consistency with the geolocation information written by the \verb=terrain=
program in the \verb=DOMAIN= file.

The first step in any application is the selection of model domain and
resolution. There are no strict rules for this selection, which in fact is
mostly determined by the nature of the problem and the availability of
computing resources. The domain should be large enough to allow the model to
develop its own circulations and to include all relevant forcings and
processes, and the resolution should be high enough to capture local processes
of interest (e.g. due to complex topography or land surface).

On the other hand the model computational cost increases rapidly with
resolution and domain size, so a compromise needs to be usually reached
between all these factors.

This is usually achieved by experience, understanding of the problem or
trial and error, however one tip to remember is to avoid that the boundaries
of the domain cross major topographical systems.

This is because the mismatch in the resolution of the coarse scale lateral
driving fields and the model fields in the presence of steep topography may
generate spurious local effects (e.g. localized precipitation areas) which can
affect the model behavior, at least in adjacent areas. 

{\footnotesize
\begin{Verbatim}
 &geoparam
 iproj = 'LAMCON', ! Domain cartographic projection. Supported values are:
                   ! 'LAMCON', Lambert conformal.
                   ! 'POLSTR', Polar stereographic. (Doesn't work)
                   ! 'NORMER', Normal  Mercator.
                   ! 'ROTMER', Rotated Mercator.
 ds = 60.0,        ! Grid point horizontal resolution in km
 ptop = 5.0,       ! Pressure of model top in cbar
 clat = 45.39,     ! Central latitude  of model domain in degrees
                   ! North hemisphere is positive
 clon = 13.48,     ! Central longitude of model domain in degrees
                   ! West is negative.
 plat = 45.39,     ! Pole latitude (only for rotated Mercator Proj)
 plon = 13.48,     ! Pole longitude (only for rotated Mercator Proj)
 truelatl = 30.0,  ! Lambert true latitude (low latitude side)
 truelath = 60,    ! Lambert true latitude (high latitude side)
 i_band = 0,       ! Enable ONLY if BAND option activated.
 /
\end{Verbatim}
}

The things you need to know here:

\begin{enumerate}
\item The different projection engines produce better results depending on the
position and extent of the domain. In particular, regardless of hemisphere:
\begin{itemize}
\item Middle latitudes (around 45 degrees) - Lambert Conformal
\item Polar latitudes (more than 75 degrees) - Polar Stereographic
\item Low latitudes (up to 30 degrees and crossing the equator) - Mercator
\item Crossing more than 45 degrees extent in latitude - Rotated Mercator
\end{itemize}
\item The model hydrostatic engine does not allow a resolution lower than
$20 km$. If you want a higher resolution consider using the subgridding scheme.
ICTP plans to introduce in the future a non-hydrostatic compressible core to
the RegCM model.
\item Lowering the top pressure of the model can give you problems in regions
with complex topography. Touch the default after thinking twice on that.
\item Always specify \verb=clat= and \verb=clon=, the central domain point,
and do fine adjustment of the position moving it around a little bit. A
little shift in position and some tests can help you obtain a better
representation of coastlines and topography at the coarse resolutions.
\item If using \verb=LAMCON= projection, take care to place the two
true latitudes at around one fourth and three fourth of the domain latitude
space to better correct the projection distortion of the domain.
\item The pole position for the rotated mercator position should be as near as
possible to the center domain position.
\item For the \verb=i_band= parameter, see below in the \verb=BAND= option
discussion in \ref{i_band}.
\end{enumerate}

\subsection{aerosolparam stanza}
\label{aerosolparam}
This stanza allows the user to specify aerosol usage in the model. It does
enable building of soil texture database in the \verb=terrain= program and
controls the dimension of the number of optical active tracers used in the
active chemistry tracers part of the model.

{\footnotesize
\begin{Verbatim}
 &aerosolparam
 aertyp = 'AER00D0' ! Aerosol dataset used
                    ! One in :
                    ! AER00D0 -> Neither aerosol, nor dust used
                    ! AER01D0 -> Biomass, SO2 + BC + OC, no dust
                    ! AER10D0 -> Anthropogenic, SO2 + BC + OC, no dust
                    ! AER11D0 -> Anthropogenic+Biomass, SO2 + BC + OC, no dust
                    ! AER00D1 -> No aerosol, with dust
                    ! AER01D1 -> Biomass, SO2 + BC + OC, with dust
                    ! AER10D1 -> Anthropogenic, SO2 + BC + OC, with dust
                    ! AER11D1 -> Anthropogenic+Biomass, SO2 + BC + OC, with dust
 ntr =  4,          ! Tracer parameters: number of tracers
 nbin = 2,          ! Tracer parameters: bins number for dust
 /
\end{Verbatim}
}

The things you need to know here:

\begin{enumerate}
\item If \verb=aertyp= is left to \verb=AER00D0=, it is nonsense to activate
chemistry in the model.
\item The total number of tracers activated must be greater than nbin.
\item If Anthropogenic and/or Biomass is activated, the model will also need
the user to run the aerosol program. It can be run at the same level as the
sst program, with the same calling syntax. Just replace sst with aerosol.
\begin{Verbatim}
$> cd $REGCM_RUN
$> ./Bin/aerosol myregcm.in
\end{Verbatim}
The aerosol program prepares an emission dataset used by the model
to consider optical active chemistry species effects in the radiation
calculation. The surface dust emission are calculated using the soil
texture dataset prepared by the \verb=terrain= program if the last
$0$ is set to $1$ in \verb=aertyp=.
\end{enumerate}

\subsection{terrainparam stanza}
\label{terparam}
This stanza is used by the \verb=terrain= program to know how you want
to generate the \verb=DOMAIN= file. You can control its work using a number of
parameters to obtain what you consider the best representation of the
physical reality. Do not underestimate what you can do at this early stage,
having a good representation of the surface can lead to valuable results
later when the model calculates climatic parameters.

{\footnotesize
\begin{Verbatim}
 &terrainparam
 domname  = 'AQWA',      ! Name of the domain. Controls naming of input files
 ntypec = 5,             ! Resolution of the global terrain and landuse data
                         ! Use 60, for  1  degree resolution
                         !     30, for 30 minutes resolution
                         !     10, for 10 minutes resolution
                         !      5, for  5 minutes resolution
                         !      3, for  3 minutes resolution
                         !      2, for  2 minutes resolution
 ntypec_s = 2,           ! Same for subgrid (Used only if nsg > 1)
 smthbdy = .false.,      ! Smoothing Control flag
                         !  true  -> Perform extra smoothing in boundaries
 lakedpth    = .false.,  ! If using lakemod (see below), produce from
                         ! terrain program the domain bathymetry
 fudge_lnd   = .false.,  ! Fudging Control flag, for landuse of grid
 fudge_lnd_s = .false.,  ! Fudging Control flag, for landuse of subgrid
 fudge_tex   = .false.,  ! Fudging Control flag, for texture of grid
 fudge_tex_s = .false.,  ! Fudging Control flag, for texture of subgrid
 fudge_lak   = .false.,  ! Fudging Control flag, for lake of grid
 fudge_lak_s = .false.,  ! Fudging Control flag, for lake of subgrid
 h2opct = 75.,           ! Surface minimum H2O percent to be considered water
 dirter = 'input/',      ! Output directory for terrain files
 inpter = 'globdata/',   ! Input directory for SURFACE dataset
 /
\end{Verbatim}
}

The things you need to know here:

\begin{enumerate}
\item The \verb=domname= will control the output file naming convention, all
generated files will add this prefix to the old V3 naming convention,
giving you the capability to recognize different runs. Try to use always
meaningful names.
\item In version 4.2 does exist a single input dataset, the 30s one. The
\verb=ntypec= parameter controls initial subsampling of input dataset before
the smoothing interpolation performed by the \verb=terrain= program.
\item Use lakedepth if you plan to use the Hostetler lake model later on.
It will be useless otherwise. It may be used in the future to
have a common sea bathymetry with the ocean coupled model. The coupling
engine of RegCM will be included in future model releases.
\item You can control the final land-water mask using the h2opct parameter.
This parameter can be used to have more land points than calculated by
the simple interpolation engine. Try it with different values to find best
land shapes. A zero value means use just the interpolation engine, higher
values will extend into ocean points the land at land-water interface.
\item A number of flags control the capability of the \verb=terrain= program
to modify on request the class type variables in the \verb=DOMAIN= file. You can
modify on request the landuse, the texture and the lake/land interface.
Running once the \verb=terrain= program, it will generate for you aside from the
\verb=DOMAIN= file a series of ASCII files you can modify with any text
editor. Running the \verb=terrain= program the second time and setting
a \verb=fudge= flag, will tell the program to overwrite the selected
variable with the modified value in the ASCII file. This can be useful
for sensitivity experiments in the BATS surface model or to design
a scenario experiment.
\item Some of the land surface types in BATS have been little tested and used
or are extremely simplified and thus should be used cautiously. Specifically
the types are: sea ice, bog/marsh, irrigated crop, glacier. If such types are
present in a domain, the user is advised to carefully check the model behavior
at such points and eventually substitute these types with others.
\item The \verb=inpter= directory is expected to contain a \verb=SURFACE=
directory where the actual netCDF global dataset are stored. The overall
path is limited to $256$ characters.
\label{pathnote}
\item If the netCDF library is compiled with OpenDAP support, an URL
can be used as a path in the \verb=dirter= and \verb=inpter= variables.
Note that the $256$ character limit for paths holds in the whole program.
For \verb=terrain= program you may want to try the following URL: \\
http://clima-dods.ictp.it/thredds/dodsC
\item The texture dataset is built if the aerosol model is activated.
This is controlled by the \verb=AERTYP= flag. See above in \ref{aerosolparam}.
\end{enumerate}

\subsection{globdatparam stanza}

This stanza is used by the \verb=sst= and \verb=icbc= ICBC programs. You can
tell them how to build initial and bondary conditions.

{\footnotesize
\begin{Verbatim}
 &globdatparam
 ibdyfrq =     6,            ! boundary condition interval (hours)
 ssttyp = 'OI_WK',           ! Type of Sea Surface Temperature used
                             !  One in: GISST, OISST, OI2ST, OI_WK, OI2WK,
                             !          FV_RF, FV_A2, FV_B2,
                             !          EH5RF, EH5A2, EH5B1, EHA1B,
                             !          ERSST, ERSKT, CCSST, CA_XX, HA_XX
 dattyp = 'EIN15',           ! Type of global analysis datasets used
                             !  One in: ECMWF, ERA40, EIN75, EIN15, EIN25,
                             !          ERAHI, NNRP1, NNRP2, NRP2W, GFS11,
                             !          FVGCM, FNEST, EH5RF, EH5A2, EH5B1,
                             !          EHA1B, CCSMN, ECEXY, CA_XX, HA_XX
 gdate1 = 1990060100,        ! Start date for ICBC data generation
 gdate2 = 1990070100,        ! End data for ICBC data generation
 calendar = 'gregorian',     ! Calendar to use (gregorian, noleap or 360_day)
 dirglob = 'input/',         ! Path for ICBC produced input files
 inpglob = 'globdata/',      ! Path for ICBC global input datasets.
                   ! Look http://users.ictp.it/~pubregcm/RegCM4/globedat.htm
                   ! on how to download them.
 /
\end{Verbatim}
}

Things you need to know here:

\begin{enumerate}
\item The gdate time window to build ICBC must be always greater or equal to
the time window you plan to run the model in.
Different GCMs and reanalysis products have different length of the year.
For example, the reanalysis products employ the real year length (365 days +
real leap years, i.e. and average length of 365.2422), the CCSM has a length
of 365 days (no leap year), the HadCM has a length of 360 days (30 day months).
The RegCM4 length of the year has to be the same as in the forcing fields, and
this can be set in the variable \verb=dayspy=.
Please remember to always check the consistency of the length of the year.
\item Even if listed, not all the input engines are fully tested. Some of them
need data which have been reformatted by ICTP (they are not in the original
format with which they are distributed by the institution producing them).
Some input data are not freely distibutable by ICTP, and you need a special
agreement with the owner to use them.
Hopefully the situation is changing, and data exchange is becoming more and more
the basis for good science in the climatic field.
\item For notes on path, you can see the above in terrainparam stanza
description at \ref{pathnote}.
\end{enumerate}

\subsection{ioparam stanza}

{\footnotesize
\begin{Verbatim}
 &ioparam
 ibyte = 4,   ! Number of bytes in reclen. Usually 4
 /
\end{Verbatim}
}

Leave this untouched. The model expects input record syze to be 4 bytes.
You will need to change some compilation parameters if you change this value.
\footnote{This namelist stanza will be removed in future versions}

\subsection{debugparam stanza}

This stanza is used by all RegCM programs to enable/disable some debug printout.
In the current release this flag is honored only by the model itself. If you
are not a developer you may find this flags useless.

{\footnotesize
\begin{Verbatim}
 &debugparam
 debug_level = 0, ! Currently value of 2 and 3 control previous DIAG flag
 dbgfrq = 3,      ! Interval for printout if debug_level >= 3
 /
\end{Verbatim}
}

Just note that with current implementation, the output file syncing is left
to the netCDF library. If You want to examine step by step the output while
the model is running, set the \verb=debug_level= at value 3.

\subsection{boundaryparam stanza}

Being a limited area model, in order to be run RegCM4 requires the provision
of meteorological initial and time dependent lateral boundary conditions,
typically for wind components, temperature, water vapor and surface pressure.
These are obtained by interpolation from output from reanalysis of observations
or global climate model simulations, which thus “drive” the regional climate
model.

The lateral boundary conditions (LBC) are provided through the so called
relaxation/diffusion technique which consists of:

\begin{enumerate}
\item selecting a lateral buffer zone of n grid point width (\verb=nspgx=)
\item interpolating the driving large scale fields onto the model grid
\item applying the relaxation + diffusion term
\begin{equation}
\frac{\partial \alpha}{\partial t} = F(n)F_1 * (\alpha_{LBC}-\alpha_{mod}) -
    F(n)F2 * \Delta_2(\alpha_{LBC}-\alpha_{mod})
\end{equation}
where $\alpha$ is a prognostic variable (wind components, temperature, water
vapor, surface pressure). The first term on the rhs is a Newtonian relaxation
term which brings the model solution ($mod$) towards the LBC field ($LBC$)
and the second term diffuses the differences between model solution and LBC.
$F(n)$ is an exponential function given by:
\begin{equation}
F(n) = exp\left(\frac{-(n-1)}{anudge(k)}\right)
\end{equation}
Where $n$ is the grid point distance from the boundary (varying from $1$ to
$nspgx$): $n-1$ is the outermost grid point, $n=2$ the adjacent one etc.
The $anudge$ array determines the strength of the LBC forcing and depends on
the model level $k$. In practice $F(n)$ is equal to 1 at the outermost grid
point row and decreases exponentially to $0$ at the internal edge of the buffer
zone ($nspgd$) at a rate determined by $anudge$. Larger buffer zones and larger
values of $anudge$ will yield a greater forcing by the LBC.  
\end{enumerate}

Typically for domain sizes of $~100$ grid points we use a buffer zone width
of $10-12$ grid points, for large domains this buffer zone can increase to
values of $15$ or even $20$.

In the model $anudge$ has three increasing values from the lower, to the mid
and higher troposphere. For example for $nspgx = 10$ we use $anudge$ equals to
$1, 2, 3$ for the lower, mid and upper troposphere, respectively.

This allows a stronger forcing in the upper troposphere to insure a greater
consistency of large scale circulations with the forcing LBC while allowing
more freedom to the model in the lower troposphere where local high resolution
forcings (e.g. complex topography) are more important.

For nspgx of $15-20$, for example, $anudge$ values could be increased to
$2,3,4$. As a rule of thumb, the choice of the maximum $anudge$ value should
follow the conditions:

\begin{equation}
\frac{(nspgx-1)}{anudge(k)} \ge 3
\end{equation}

{\footnotesize
\begin{Verbatim}
 &boundaryparam
 nspgx  = 12, ! nspgx-1 represent the number of cross point slices on
              ! the boundary sponge or relaxation boundary conditions.
 nspgd  = 12, ! nspgd-1 represent the number of dot point slices on
              ! the boundary sponge or relaxation boundary conditions.
 high_nudge   =     3.0, ! Nudge value high range
 medium_nudge =     2.0, ! Nudge value medium range
 low_nudge    =     1.0  ! Nudge value low range
 /
\end{Verbatim}
}

\subsection{modesparam stanza}

This needs not to be changed. Leave it to the default value.

{\footnotesize
\begin{Verbatim}
 &modesparam
 nsplit = 2, ! Number od split exp modes
 /
\end{Verbatim}
}

\subsection{restartparam stanza}

This stanza lets you control the time period the model is currently simulating
in this particular run. You may want to split longer runs for which you have
prepared the ICBC's into shorter runs, to schedule HPC resource usage in a more
collaborative way with other researcher sharing it: the regcm model allows
restart, so be friendly with other research projects which may not have this
fortune (unless you are late for publication).

{\footnotesize
\begin{Verbatim}
 &restartparam
 ifrest  = .false. ,   ! If a restart
 mdate0  = 1990060100, ! Global start (is gdate1, most probably)
 mdate1  = 1990060100, ! Start date of this run
 mdate2  = 1990060200, ! End date for this run
 /
\end{Verbatim}
}

Things you need to know here:

\begin{enumerate}
\item After the simulation starts, on restart NEVER change the \verb=mdate0=
value. The correct scheme for restart is:
\begin{itemize}
\item Set \verb=ifrest= to \verb=.true.=
\item Set \verb=mdate1= to the value in \verb=mdate2=
\item Define the new value for \verb=mdate2=
\end{itemize}
\item Consider that current RegCM convention is to place midnight of first
day of month as the last timestep in previous month, except on first model
output file (\verb=ifrest= = \verb=.false.=). It is for this reason better
to use as start and end time a month boundary. We usually consider a month
data file the basic unit of output, each time you cross a month a new output
file will be created for you.
\end{enumerate}

\subsection{timeparam stanza}

This stanza contains model internal timesteps, used by the model as basic
integration timestep and triggers for calling internal parametric schemes.

{\footnotesize
\begin{Verbatim}
 &timeparam
 dt     =   150., ! time step in seconds
 dtrad  =    30., ! time interval solar radiation calculated (minutes)
 dtabem =    18., ! time interval absorption-emission calculated (hours)
 dtsrf  =   600., ! time interval at which land model is called (seconds)
 /
\end{Verbatim}
}

Things you need to know here:

\begin{enumerate}
\item The dynamical hydrostatical core of RegCM requires a fixed timestep,
and you need to manually find the correct value which permits not to break
the Courant–Friedrichs–Lewy condition considering \cite{CFL}. A good rule
of thumb is to have a \verb=dt= not greater than three times the \verb=ds=
value in $km$ specified in the \verb=geoparam= stanza at \ref{geoparam}.
A greater value may lower computing time, but in case of strong advection
may lead to non accurate computation or even the violation of CFL condition
and the divergence of the solution.
\item All the other internal timesteps need to be multiples of the base timestep.
Note that the units are different, so you need to convert the other timesteps
in seconds before the check.
\item In case of strong surface gradients, a low value for the surface timesteps
may help the model better describe the interaction with the atmosphere and
obtain a stable solution.
\item If you hit a non stable condition, the restart capability of the model
may help find the correct timestep just for a particular period, using a
different timestep at different times.
\end{enumerate}

\subsection{outparam stanza}

This stanza controls the model output engine, allowing you to enable/disable
any of the output file writeout, or to modify the frequency the fields are
written in the files.

{\footnotesize
\begin{Verbatim}
 &outparam
 ifsave  = .true. ,         ! Create SAV files for restart
 savfrq  =    48.,          ! Frequency in hours to create them
 ifatm   = .true. ,         ! Output ATM ?
 atmfrq  =     6.,          ! Frequency in hours to write to ATM
 ifrad   = .true. ,         ! Output RAD ?
 radfrq  =     6.,          ! Frequency in hours to write to RAD
 ifsrf   = .true. ,         ! Output SRF ?
 ifsts   = .true. ,         ! Output STS ?
 ifsub   = .true. ,         ! Output SUB ?
 srffrq  =     3.,          ! Frequency in hours to write to SRF and SUB (and CLM)
 iflak   = .true.,          ! Output LAK ?
 lakfrq  =     6.,          ! Frequency in hours to write to LAK if lakemod is 1
                            ! It must be an integer multiple of batfrq
 ifchem  = .true.,          ! Output CHE ?
 chemfrq =     6.,          ! Frequency in hours to write to CHE
 atm_enablevar = 14*.true., ! Mask to eventually disable variables ATM
 srf_enablevar = 24*.true., ! Mask to eventually disable variables SRF
 sts_enablevar = 9*.true.,  ! Mask to eventually disable variables STS
 lak_enablevar = 16*.true., ! Mask to eventually disable variables LAK
 sub_enablevar = 16*.true., ! Mask to eventually disable variables SUB
 rad_enablevar = 15*.true., ! Mask to eventually disable variables RAD
 che_enablevar = 17*.true., ! Mask to eventually disable variables CHE
 dirout  = './output',      ! Path where all output will be placed
 /
\end{Verbatim}
}

Things you need to know here:

\begin{enumerate}
\item The surface fields are the mean values in the interval specified by
the frequency values. The dynamical fields are instead the point value at the
output time. Refer to the Reference Manual \cite{refman_11} for a detailed
description of the model output fields.
\item If the chemistry or lake model are not enabled, the values specified in
the control flags are not considered. If \verb=nsg= is not greater than one
in dimparam at \ref{dimparam}, the \verb=ifsub= flag is not considered.
\item For the output directory, the path variable has a limit of $256$
characters. This path must be a local path on disk where the user running
the model has write permissions granted.
\item The enablevar logical arrays can be used to avoid saving one of the time
dependent variables in the output file, in the order they are saved in the output
file itself. Note that the variables time, tbnds and ps cannot be disabled.
\end{enumerate}

\subsection{physicsparam stanza}
\label{physicsparam}

This stanza controls the model physics. You have a number of option here,
and the best way to select the right set is to carefully read the the
Reference Manual \cite{refman_11}. We are for the purposes of this User Guide
not going in detail in here, except in saying that probably you will need
to run some experiments especially with different cumulus convection
schemes before finding out the best model setting.
Although the mixed convection scheme (Grell over land and Emanuel over ocean)
seems to provide an overall better performance, our experience is that there
is no scheme that works best everywhere, therefore we advice to always do some
sensitivity experiments to select the best scheme for your application.

{\footnotesize
\begin{Verbatim}
 &physicsparam
 iboudy  =          5,  ! Lateral Boundary conditions scheme
                        !   0 => Fixed
                        !   1 => Relaxation, linear technique.
                        !   2 => Time-dependent
                        !   3 => Time and inflow/outflow dependent.
                        !   4 => Sponge (Perkey & Kreitzberg, MWR 1976)
                        !   5 => Relaxation, exponential technique.
 ibltyp  =          1,  ! Boundary layer scheme
                        !   0 => Frictionless
                        !   1 => Holtslag PBL (Holtslag, 1990)
                        !   2 => UW PBL (Bretherton and McCaa, 2004)
                        !  99 => Holtslag PBL, with UW in diag. mode
 icup    =          4,  ! Cumulus convection scheme
                        !   1 => Kuo
                        !   2 => Grell
                        !   3 => Betts-Miller (1986) DOES NOT WORK !!!
                        !   4 => Emanuel (1991)
                        !   5 => Tiedtke (1986) UNTESTED !!!
                        !  99 => Use Grell over land and Emanuel over ocean
                        !  98 => Use Emanuel over land and Grell over ocean
   igcc  =          1,  ! Grell Scheme Cumulus closure scheme
                        !   1 => Arakawa & Schubert (1974)
                        !   2 => Fritsch & Chappell (1980)
 ipptls  =          1,  ! Moisture scheme
                        !   1 => Explicit moisture (SUBEX; Pal et al 2000)
 iocnflx =          2,  ! Ocean Flux scheme
                        !   1 => Use BATS1e Monin-Obukhov
                        !   2 => Zeng et al (1998)
   iocnrough =      1,  ! Zeng Ocean model roughness formula to use.
                        !   1 => (0.0065*ustar*ustar)/egrav
                        !   2 => (0.013*ustar*ustar)/egrav + 0.11*visa/ustar
 ipgf    =          0,  ! Pressure gradient force scheme
                        !   0 => Use full fields
                        !   1 => Hydrostatic deduction with pert. temperature
 iemiss  =          0,  ! Calculate emission
 lakemod =          0,  ! Use lake model
 ichem   =          1,  ! Use active aerosol chemical model
 scenario =      'A1B', ! IPCC Scenario to use in A1B,RF,A2,B1,B2
                        ! RCP Scenarios in RCP3PD,RCP4.5,RCP6,RCP8.5
 idcsst   =          0, ! Use diurnal cycle sst scheme
 iseaice  =          0, ! Model seaice effects
 idesseas =          1, ! Model desert seasonal albedo variability
 iconvlwp =          1, ! Use convective liquid water path as the large-scale
                        ! liquid water path
 \
\end{Verbatim}
}

\subsection{subexparam stanza}

This stanza controls the moisture scheme. Please consider carefully reporting
in your work the tuning you perform on this parameters. The parameters below
are the ones currently used at ICTP.

{\footnotesize
\begin{Verbatim}
 &subexparam
 ncld      =          1, ! # of bottom model levels with no clouds
 fcmax     =       0.80, ! Maximum cloud fraction cover
 qck1land  =   .250E-03, ! Autoconversion Rate for Land
 qck1oce   =   .250E-03, ! Autoconversion Rate for Ocean
 gulland   =        0.4, ! Fract of Gultepe eqn (qcth) when precip occurs
 guloce    =        0.4, ! Fract of Gultepe eqn (qcth) for ocean
 rhmax     =       1.01, ! RH at whicn FCC = 1.0
 rh0oce    =       0.90, ! Relative humidity threshold for ocean
 rh0land   =       0.80, ! Relative humidity threshold for land
 tc0       =      238.0, ! Below this temperature, rh0 begins to approach unity
 cevap     =   .100E-02, ! Raindrop evap rate coef [[(kg m-2 s-1)-1/2]/s]
 caccr     =      3.000, ! Raindrop accretion rate [m3/kg/s]
 cllwcv    =     0.3E-3, ! Cloud liquid water content for convective precip.
 clfrcvmax =       0.25, ! Max cloud fractional cover for convective precip.
 cftotmax  =       0.75, ! Max total cover cloud fraction for radiation
 /
\end{Verbatim}
}

We found that RegCM4 is especially sensitive to:

\begin{enumerate}
\item \verb=cevap= : increasing cevap will generally decrease precipitation
\item \verb=gulland=, \verb=guloce= : increase of guland/guloce will generally
lead to reduce precipitation
\end{enumerate}

\subsection{grellparam, emanparam and tiedtkeparam stanzas}

You are allowed here to tune the convection scheme selected above in
\ref{physicsparam} with the \verb=icup= number if selected number is
$2, 4, 98, 99$. 

{\footnotesize
\begin{Verbatim}
 &grellparam
 shrmin = 0.25,       ! Minimum Shear effect on precip eff.
 shrmax = 0.50,       ! Maximum Shear effect on precip eff.
 edtmin = 0.25,       ! Minimum Precipitation Efficiency
 edtmax = 0.50,       ! Maximum Precipitation Efficiency
 edtmino = 0.25,      ! Minimum Precipitation Efficiency (o var)
 edtmaxo = 0.50,      ! Maximum Precipitation Efficiency (o var)
 edtminx = 0.25,      ! Minimum Precipitation Efficiency (x var)
 edtmaxx = 0.50,      ! Maximum Precipitation Efficiency (x var)
 shrmin_ocn = 0.25,   ! Minimum Shear effect on precip eff. OCEAN points
 shrmax_ocn = 0.50,   ! Maximum Shear effect on precip eff.
 edtmin_ocn = 0.25,   ! Minimum Precipitation Efficiency
 edtmax_ocn = 0.50,   ! Maximum Precipitation Efficiency
 edtmino_ocn = 0.25,  ! Minimum Precipitation Efficiency (o var)
 edtmaxo_ocn = 0.50,  ! Maximum Precipitation Efficiency (o var)
 edtminx_ocn = 0.25,  ! Minimum Precipitation Efficiency (x var)
 edtmaxx_ocn = 0.50,  ! Maximum Precipitation Efficiency (x var)
 pbcmax = 150.0,      ! Max depth (mb) of stable layer b/twn LCL & LFC
 mincld = 150.0,      ! Min cloud depth (mb).
 htmin = -250.0,      ! Min convective heating
 htmax = 500.0,       ! Max convective heating
 skbmax = 0.4,        ! Max cloud base height in sigma
 dtauc = 30.0,        ! Fritsch & Chappell (1980) ABE Removal Timescale (min)
 /

 &emanparam
 minsig = 0.95,   ! Lowest sigma level from which convection can originate
 elcrit = 0.0011, ! Autoconversion threshold water content (g/g)
 tlcrit = -55.0,  ! Below tlcrit auto-conversion threshold is zero
 entp = 1.5,      ! Coefficient of mixing in the entrainment formulation
 sigd = 0.05,     ! Fractional area covered by unsaturated dndraft
 sigs = 0.12,     ! Fraction of precipitation falling outside of cloud
 omtrain = 50.0,  ! Fall speed of rain (Pa/s)
 omtsnow = 5.5,   ! Fall speed of snow (Pa/s)
 coeffr = 1.0,    ! Coefficient governing the rate of rain evaporation
 coeffs = 0.8,    ! Coefficient governing the rate of snow evaporation
 cu = 0.7,        ! Coefficient governing convective momentum transport
 betae = 10.0,    ! Controls downdraft velocity scale
 dtmax = 0.9,     ! Max negative parcel temperature perturbation below LFC
 alphae = 0.2,    ! Controls the approach rate to quasi-equilibrium
 damp = 0.1,      ! Controls the approach rate to quasi-equilibrium
 /

 &tiedtkeparam
 iconv = 1,         ! Actual used scheme.
 entrpen = 1.0D-4,  ! Entrainment rate for penetrative convection
 entrscv = 3.0D-4,  ! Entrainment rate for shallow convection
 entrmid = 1.0D-4,  ! Entrainment rate for midlevel convection
 entrdd = 2.0D-4,   ! Entrainment rate for cumulus downdrafts
 cmfcmax = 1.0D0,   ! Maximum massflux value 
 cmfcmin = 1.0D-10, ! Minimum massflux value (for safety)
 cmfdeps = 0.3D0,   ! Fractional massflux for downdrafts at lfs
 rhcdd = 1.0D0,     ! Relative saturation in downdrafts
 cmtcape = 40.0D0,  ! CAPE adjustment timescale parameter
 zdlev = 1.5D4,     ! Restrict rainfall up to this elevation
 cprcon = 1.0D-4,   ! Coefficients for determining conversion
 nmctop = 4,        ! max. level for cloud base of mid level conv.
 cmfctop = 0.35D0,  ! Relat. cloud massflux at level above nonbuoyancy
 lmfpen = .true.,   ! true if penetrative convection is switched on
 lmfscv = .true.,   ! true if shallow convection is switched on
 lmfmid = .true.,   ! true if midlevel convection is switched on
 lmfdd = .true.,    ! true if cumulus downdraft is switched on
 lmfdudv = .true.,  ! true if cumulus friction is switched on
 /
\end{Verbatim}
}

Things you need to know here:

\begin{enumerate}
\item In case of the mixed schemes $98, 99$, both the Grell and Emanuel stanzas
are read in. Note in this case for Grell scheme only the relevant
(Ocean or Land) control values are used.
\item Minimum and maximum values of the fraction of reevaporated water in the
downdraft for the Grell scheme is essentially a measure of the precipitation
efficiency: increasing their value generally decrease convective precipitation.
\item Again, read carefully the Reference Manual before attempting any tuning,
and report in any work modification of this parameters.
\end{enumerate}

\subsection{uwparam stanza}

You are allowed here to tune the UW PBL scheme selected above in
\ref{physicsparam} with the \verb=ibltyp= number if selected number is
$2, 99$. 

{\footnotesize
\begin{Verbatim}
 &uwparam
 iuwvadv = 0,   ! ?????????????
 ilenparam = 0, ! ?????????????
 atwo = 15.0D0, ! ?????????????
 rstbl = 1.5D0, ! ?????????????
 /
\end{Verbatim}
}

Travis need to add something here.

\subsection{chemparam stanza}

This stanza controls the optical active aerosols scheme in the RegCM model.
\footnote{In the future model version a more complete chemical scheme will
be introduced} 

{\footnotesize
\begin{Verbatim}
 &chemparam
 idirect      =    1, ! enable or not aerosol feedbacks on radiation and
                      ! dynamics (aerosol direct and semi direct effcts):
                      !  0 = no coupling. Aerosol are only transported and
                      !      don't interact with radiation scheme.
                      !  1 = no coupling to dynamic and thermodynamic. However
                      !      the clear sky surface and top of atmosphere
                      !      aerosol radiative forcings are diagnosed.
                      !  2 = allows aerosol feedbacks on radiative,
                      !      thermodynamic and dynamic fields.
 ichremlsc    =    1, ! 1 = allows tracer removal (wet deposition) by large
                      !     scale cloud
 ichremcvc    =    1, ! 1 = allows tracer removal by convective clouds
 ichdrdepo    =    1, ! 1 = enable tracer surface dry deposition. For dust,
                      !     it is calculated by a size settling and dry
                      !     deposition scheme. For other aerosol,a dry
                      !     deposition velocity is simply prescribed further.
                      !     Next release will include an improved aerosol dry
                      !     deposition scheme for non dust aerosols.
 ichcumtra    =    1, ! 1 = enable tracer convective transport and mixing.
 inpchtrname  =  'DUST','DUST,'BC_HB','BC_HL',
                      ! Tracer identifier. The number of input should be equal
                      ! to ntr you have the choice between: 
                      ! DUST  = Dust particle from soil
                      ! BC_HB = Hydrophobic Black carbon aerosol
                      ! BC_HL = Hydrophilic or aged black carbon
                      ! OC_HB = Hydrophobic organic carbon aerosl
                      ! OC_HL = Hydrophilic or aged organic carbon
                      ! SO2   = sulfur dioxide
                      ! SO4   = sulfate aerosol
 inpchtrsol   =  0.1, 0.1, 0.05, 0.8,
                      ! Tracer solubility (fraction). The number of input
                      ! should be equal to ntr. Will determine if tracer are
                      ! efficiently removed by wet deposition or not
 inpchtrdpv   =  0.,0.,0.00025,0.00025,  0.,0.,0.00025,0.00250,
                      ! Dry deposition velocity (in m/s) over land (first ntr)
                      ! and ocean (second ntr values), a total of ntr*2 values.
                      ! Should be consistent with tracer identifier.
                      ! for DUST type this value is not effectively considered
                      ! since a dry deposition scheme is explicitely included
                      ! in RegCM.
 inpdustbsiz  =   0.1,  1., 1., 2.5,
                      ! Lower Size limit (first nbin) and Upper Size limit
                      ! (second nbin values) of diameter bin classes for dust
                      ! (in micrometer). Should never exceed nbin * 2 values.
                      ! So in this example there are two bins of
                      !   *  0.1 - 1.0 micrometer
                      !   *  1.0 - 2.5 micrometer
 /
\end{Verbatim}
}

Things you need to know here:

\begin{enumerate}
\item Always doublecheck consistency in dimensions specified in aerosolparam at
\ref{aerosolparam} and the number of elements in input arrays here.
\item This stanza is not considered if \verb=ichem= in physicsparam at
\ref{physicsparam} is not set to $1$.
\item Dust optical properties have been calculated for 4 defaults size bins
in RegCM. If you want to modify the bin size for dust / climate feedback
interactions consider extending this by yourself. Current bins are $0.01-1.00$,
$1.00-2.50$, $2.50-5.00$ and $5.00-20.0$ micron diameter.
\end{enumerate}

\section{The BAND and the CLM options}

We will now discuss from the user point of view how to use the two model
setups which need to be activated at configure stage.

\subsection{BAND option}

The BAND option if activated allows the user to run a simulation over a tropical
band symmetric around the equator. The executable of the model is different
in the case of the band, and is named \verb=regcmMPI_band= or
\verb=regcmSerial_band=. Note that due to the computational need of the
\verb=BAND= model, it is strongly suggested to run it on parallel machines.
\footnote{The serial option will not be supported in future releases}

\subsubsection{Enable}

At configure stage (see \ref{modconf}), the option is to be enabled with the
right command line argument to the configure script

\begin{Verbatim}
  --enable-band           Supply this option if you plan on using tropical
                          band option.
\end{Verbatim}

This will enable a preprocessing flag, and build a different model executable.
Note that no modifications are needed for any other part of the model.

\subsubsection{Prepare and run}
\label{i_band}

In the case of \verb=BAND= run, the \verb=geoparam= stanza described above in
\ref{geoparam} is mostly ignored, as the projection is set to Normal Mercator,
the center of the projection is set to \verb'clat = 0.0', \verb'clon = 180.0',
and the grid point resolution is calculated as:

\begin{equation}
\frac{2*\pi*6370.0}{jx}
\end{equation}

The only parameter you need to set for a \verb=BAND= run is the \verb=i_band=
value: set it to $1$.

No special modification in model run is required, all steps are equal as in
chapter \ref{tutorial}. Just substitute the executable name:

\begin{Verbatim}
$> mpirun -np 2 ./Bin/regcmMPI_band band.in
\end{Verbatim}

Some notes:

\begin{enumerate}
\item The model using the \verb=BAND= option is heavy, as the number of points
is usually huge to obtain a good horizontal resolution. Check any memory
limit is disabled on your platform before attempting a run with the \verb=BAND=
option active.
\item The model with the \verb=BAND= option scales very well on a cluster
with a large number of processors.
\end{enumerate}

\subsection{CLM option}
\label{clm}

The \verb=CLM= option if activated allows the user to run a simulation using
the CLM surface model instead of the default BATS1e model. We will not here go
in deep in the difference between the two models, read the Reference Manual
for this.
The executable of the model is different in the case of the \verb=CLM=, and is
named \verb=regcmMPI_clm=.
Note that in the CLM case only the MPI enabled compilation is supported (no
serial), and no subgridding is possible (\verb=nsg= is always $1$).

\subsubsection{Enable}

At configure stage (see \ref{modconf}), the option is to be enabled with the
right command line argument to the configure script

\begin{Verbatim}
  --enable-clm            Supply this option if you plan on using CLM option.
\end{Verbatim}

This will enable a preprocessing flag, and build a different model executable.
Note that no modifications are needed for any other part of the model, but
this triggers the building of another pre-processing program, \verb=clm2rcm=.

\subsubsection{Prepare and run}
\label{clmrun}

The \verb=CLM= configuration requires a separate stanza in the namelist input
file.

{\footnotesize
\begin{Verbatim}
 &clmparam
 dirclm = 'input/', ! CLM path to Input data produced by clm2rcm. If 
                    ! relative, It should be how to reach the Input dir
                    ! from the Run dir.
 clmfrq =  12.,     ! Frequency for CLM own output write
 imask  =   1,      ! For CLM, Type of land surface parameterization
                    !   1 => using DOMAIN.INFO for landmask (same as BATS)
                    !   2 => using mksrf_navyoro file landfraction for
                    !        landmask and perform a weighted average over
                    !        ocean/land gridcells; for example:
                    ! tgb = tgb_ocean*(1-landfraction)+tgb_land*landfraction
 /
\end{Verbatim}
}

Things you need to know here:

\begin{enumerate}
\item The \verb=inpter= path defined in \verb=terrainparam= stanza described in
\ref{terparam} is used also by the \verb=clm2rcm= program. See at \ref{clmdata}
how to obtain needed datasets.
\item The file \verb=pft-physiology.c070207= should be manually copied in the
\verb=dirclm= directory before running the model.
\item The clmfrq is relative to the output produced by the \verb=CLM= model
itself, and does not control the RegCM model output. To know the \verb=CLM=
output file content, refer to CLM 3.5 documentation.
\item The \verb'imask = 2' option cannot be used with the \verb'icup'
cumulus convection schemes $2, 98, 99$, which rely on the BATS1e landmask.
\end{enumerate}

In the case of \verb=CLM= run, the user needs to run, after the \verb=terrain=
program, the \verb=clm2rcm= program, and copy the \verb=pft-physiology.c070207=
in the input directory:

\begin{Verbatim}
$> cd $REGCM_RUN
$> ./Bin/terrain regcm.in
$> ./Bin/clm2rcm regcm.in
$> cp $REGCM_GLOBEDAT/CLM/pft-physiology.c070207 input/
\end{Verbatim}

The \verb=clm2rcm= program interpolates global land characteristics datasets
to the RegCM projected grid. The content of the \verb=pft-physiology.c070207=
file are described in the \verb=pft-physiology.c070207.readme= file.
All the other pre-processing steps are just equal to the one detailed in
chapter \ref{tutorial}. To run the \verb=CLM= option in the RegCM model, just
substitute the executable name:

\begin{Verbatim}
$> mpirun -np 2 ./Bin/regcmMPI_clm regcm.in
\end{Verbatim}

Note that the \verb=CLM= land model is much heavier than the BATS1e model, and
computing time increases.

\section{Sensitivity experiments hint}

Although the LBC forcing does provide a constraint for the model, as any RCM,
RegCM4 is characterized by a certain level of internal variability due to its
non-liner processes (e.g. convection).

For example, if small perturbations are introduced in the initial or lateral
boundary conditions, the model will generally produce different patterns of,
e.g. precipitation, that appear as (sometimes seemingly organized) noise when
compared to the control simulation.

This noise depends on domain size and climatic regimes, for example it is
especially pronounced in warm climate regimes (e.g. tropics or during the
summer season) and large doamins.

When doing for example sensitivity experiments to model modifications, e.g. to
land use change, this internal variability “noise” can be misinterpreted as a
model response to the factor modified.

Users of RegCM4 should be aware of this when they do sensitivity experiments.
The best way to filter out this noise is to perform ensembles of simulations
and lok at the ensemble averages to extract the real model response from the
noise.
