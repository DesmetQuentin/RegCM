%
% This file is part of ICTP RegCM model.
% Copyright (C) 2011 ICTP Trieste
% See the file COPYING for copying conditions.
%

We will examine in this chapter in more detail the namelist configuration file,
to give you the User a deeper knowledge of model capabilities.

\section{The commented namelist}

In this section we will show you the commented namelist input file you will
find under \verb=$REGCM_ROOT/Doc= with the name \verb=README.namelist= .
All model programs seen so far, with the exception of the GrADS helper program,
use as input this namelist file, which is unique to a particular simulation.
The model input namelist file is composed by a number of different namelists,
each one devoted to configuring the model capabilities.
A namelist in the namelist file is identified with a starting \verb=&= character
followed by namelist name, and ends on a single line with the \verb=\=
character.

\subsection{dimparam namelist}
\label{dimparam}

This namelist contains the base X,Y,Z domain dimension information, used
by the model dynamic memory allocator to request the Operating System the
memory space to store the model internal variables.

{\footnotesize
\begin{Verbatim}
 &dimparam
 iy     = 34,   ! This is number of points in the N/S direction
 jx     = 48,   ! This is number of points in the E/W direction
 kz     = 23,   ! Number of vertical levels
 dsmin  = 0.01, ! Minimum sigma spacing (only used if kz is not 14, 18, or 23)
 dsmax  = 0.05, ! Maximum sigma spacing (only used if kz is not 14, 18, or 23)
 nsg    = 1,    ! For subgridding, number of points to decompose. If nsg=1,
                ! no subgridding is performed. CLM3.5 does NOT work with
                ! subgridding enabled.
 njxcpus = -1,  ! Number of CPUS to be used in the jx (lon) dimension.
                ! If <=0 , the executable will try to figure out a suitable
                ! decomposition.
 niycpus = -1,  ! Number of CPUS to be used in the iy (lat) dimension.
                ! If <=0 , the executable will try to figure out a suitable
                ! decomposition.
 /
\end{Verbatim}
}

The things you need to know here:

\begin{enumerate}
\item In the current version 4.5 the model parallelizes execution dividing
the work between the processors, with the minimum work per processor is 9
points or a box $3\times3$, so the maximum theorical number of processors which
can be used in a parallel run for the above configuration is roughly $150$,
but for the communication overhead the optimal would be around $12$ or a
$10\times10$ patch per processor.
\item If a custom number of sigma level is chosen (not 14, 18 or 23), the actual
sigma values are calculated mimimizing the $a,b$ coefficients for the 
equation:

\begin{equation}
  dsig(i) = dsmax*a^{i-1}*b^{0.5*(i-2)*(i-1)}
\end{equation}

derived from the recursive relation:

\begin{equation}
  dsig(i) = a(i)*dsig(i-1)
\end{equation}

where $a(i) = b*a(i-1)$. We at ICTP normally use 23 levels for the
non-hydrostatic core and 18 for the hydrostatic core.
\item Specifying an nsg number greater than one triggers the subgrid BATS/CLM45
model on. There is no plan to extend this feature to CLM3.5 model.
This affects only surface variable calculations. All dynamical
variables are calculated still on the coarser grid.
Rain in the current implementation is also calculated on the coarser
grid.
\item The \verb=njxcpus=, \verb=niycpus= parameters can be used to force
a particular domain decomposition for any particular hardware architecture.
Normally the algorithm in the model code should be able to fix the
supposedly optimally balanced decomposition.
\end{enumerate}

\subsection{coreparam namelist}
\label{coreparam}

This namelist is new to the RegCM 4.5 version, and controls which dynamical
core is used in the model.

{\footnotesize
\begin{Verbatim}
 &coreparam
 idynamic = 1,  ! Choice of dynamical core
                ! 1 = MM4 hydrostatic core
                ! 2 = MM5 NON hydrostatic core
 /
\end{Verbatim}
}

The things you need to know here:

\begin{enumerate}
\item For details about the two dynamical cores, refer to the Reference Manual.
\item The hydrostatic core is cheaper computationally, but physically it should
be limited to resolution greater than $15 km$. For higher resolution, the non
hydrostatic core is to be used.
\item The output variables of the two dynamical cores differ in the ATM files
because of the different state variables used. Vertical levels also differ.
\item The ICBC for the non-hydrostatic cannot be used for the hydrostatic run
and the other way round.
\item Nesting using the \verb=FNEST= option is supported in these
configurations:
\begin{enumerate}
\item hydrostatic nested into hydrostatic
\item non-hydrostatic nested into hydrostatic
\item non-hydrostatic nested into non-hydrostatic
\end{enumerate}
\item Selecting the non-hydrostatic core, the nonhydroparam namelist is read
in.
\end{enumerate}

\subsection{geoparam namelist}
\label{geoparam}

This namelist is used by the \verb=terrain= program to geolocate the model grid
on the earth surface. The RegCM model uses a limited number of projection
engines. The value here are used by the other model programs to assert
consistency with the geolocation information written by the \verb=terrain=
program in the \verb=DOMAIN= file.

The first step in any application is the selection of model domain and
resolution. There are no strict rules for this selection, which in fact is
mostly determined by the nature of the problem and the availability of
computing resources. The domain should be large enough to allow the model to
develop its own circulations and to include all relevant forcings and
processes, and the resolution should be high enough to capture local processes
of interest (e.g. due to complex topography or land surface).

On the other hand the model computational cost increases rapidly with
resolution and domain size, so a compromise needs to be usually reached
between all these factors.

This is usually achieved by experience, understanding of the problem or
trial and error, however one tip to remember is to avoid that the boundaries
of the domain cross major topographical systems.

This is because the mismatch in the resolution of the coarse scale lateral
driving fields and the model fields in the presence of steep topography may
generate spurious local effects (e.g. localized precipitation areas) which can
affect the model behavior, at least in adjacent areas. 

{\footnotesize
\begin{Verbatim}
 &geoparam
 iproj = 'LAMCON', ! Domain cartographic projection. Supported values are:
                   ! 'LAMCON', Lambert conformal.
                   ! 'POLSTR', Polar stereographic. (Doesn't work)
                   ! 'NORMER', Normal  Mercator.
                   ! 'ROTMER', Rotated Mercator.
 ds = 60.0,        ! Grid point horizontal resolution in km
 ptop = 5.0,       ! Pressure of model top in cbar
 clat = 45.39,     ! Central latitude  of model domain in degrees
                   ! North hemisphere is positive
 clon = 13.48,     ! Central longitude of model domain in degrees
                   ! West is negative.
 plat = 45.39,     ! Pole latitude (only for rotated Mercator Proj)
 plon = 13.48,     ! Pole longitude (only for rotated Mercator Proj)
 truelatl = 30.0,  ! Lambert true latitude (low latitude side)
 truelath = 60,    ! Lambert true latitude (high latitude side)
 i_band = 0,       ! Use this to enable a tropical band. In this case the ds,
                   ! iproj, clat, clon parameters are not considered.
 /
\end{Verbatim}
}

The things you need to know here:

\begin{enumerate}
\item The different projection engines produce better results depending on the
position and extent of the domain. In particular, regardless of hemisphere:
\begin{itemize}
\item Middle latitudes (around 45 degrees) - Lambert Conformal
\item Polar latitudes (more than 75 degrees) - Polar Stereographic
\item Low latitudes (up to 30 degrees and crossing the equator) - Mercator
\item Crossing more than 45 degrees extent in latitude - Rotated Mercator
\end{itemize}
\item The model hydrostatic engine does not allow a resolution lower than
$20 km$. If you want a higher resolution consider using the subgridding scheme.
ICTP plans to introduce in the future a non-hydrostatic compressible core to
the RegCM model.
\item Lowering the top pressure of the model can give you problems in regions
with complex topography. Touch the default after thinking twice on that.
\item Always specify \verb=clat= and \verb=clon=, the central domain point,
and do fine adjustment of the position moving it around a little bit. A
little shift in position and some tests can help you obtain a better
representation of coastlines and topography at the coarse resolutions.
\item If using \verb=LAMCON= projection, take care to place the two
true latitudes at around one fourth and three fourth of the domain latitude
space to better correct the projection distortion of the domain.
\item The pole position for the rotated mercator position should be as near as
possible to the center domain position.
\item For the \verb=i_band= parameter, selecting this will enable the tropical
band experiment, and the horizontal resolution will be calculated from the
number of jx points.
The projection is set to Normal Mercator, the center of the projection is set
to \verb'clat = 0.0', \verb'clon = 180.0',
and the grid point resolution is calculated as:
\begin{equation}
\frac{2*\pi*6370.0}{jx}
\end{equation}
Just remember:
\begin{enumerate}
\item The model for a tropical band simulation is heavy, as the number of points
is usually huge to obtain a good horizontal resolution. Check any memory
limit is disabled on your platform before attempting a run.
\item The model scales well on a cluster with a large number of processors.
\end{enumerate}

\end{enumerate}

\subsection{terrainparam namelist}
\label{terparam}
This namelist is used by the \verb=terrain= program to know how you want
to generate the \verb=DOMAIN= file. You can control its work using a number of
parameters to obtain what you consider the best representation of the
physical reality. Do not underestimate what you can do at this early stage,
having a good representation of the surface can lead to valuable results
later when the model calculates climatic parameters.

{\footnotesize
\begin{Verbatim}
 &terrainparam
 domname  = 'AQWA',           ! Name of the domain/experiment.
                              ! Controls naming of input files
 smthbdy = .false.,           ! Smoothing Control flag
                              !  true -> Perform extra smoothing in boundaries
 lakedpth    = .false.,       ! If using lakemod (see below), produce from
                              ! terrain program the domain bathymetry
 ltexture    = .false.,       ! If using DUST tracers (see below), produce
                              ! the domain soil texture dataset
 lsmoist     = .false.,       ! Use Satellite Soil Moisture Dataset for
                              ! initialization of soil moisture.
 fudge_lnd   = .false.,       ! Fudging Control flag, for landuse of grid 
 fudge_lnd_s = .false.,       ! Fudging Control flag, for landuse of subgrid
 fudge_tex   = .false.,       ! Fudging Control flag, for texture of grid
 fudge_tex_s = .false.,       ! Fudging Control flag, for texture of subgrid
 fudge_lak   = .false.,       ! Fudging Control flag, for lake of grid
 fudge_lak_s = .false.,       ! Fudging Control flag, for lake of subgrid
 h2opct = 50.,                ! Surface min H2O percent to be considered water
 h2ohgt = .true.,             ! Allow water points to have hgt greater than 0
 ismthlev = 1,                ! How many times apply the 121 smoothing
 dirter = 'input/',           ! Output directory for terrain files
 inpter = 'globdata/',        ! Input directory for SURFACE dataset
 moist_filename = 'moist.nc', ! Read initial moisture and snow from this file
 /
\end{Verbatim}
}

The things you need to know here:

\begin{enumerate}
\item The \verb=domname= will control the output file naming convention, all
generated files will add this prefix to the old V3 naming convention,
giving you the capability to recognize different runs. Try to use always
meaningful names.
\item You can control the final land-water mask using the h2opct parameter.
This parameter can be used to have more land points than calculated by
the simple interpolation engine. Try it with different values to find best
land shapes. A zero value means use just the interpolation engine, higher
values will extend into ocean points the land at land-water interface.
The h2ohgt parameter allows also water points to have elevation greater
than zero to avoid wall effects on the coasts.
\item A number of flags control the capability of the \verb=terrain= program
to modify on request the class type variables in the \verb=DOMAIN= file. You can
modify on request the landuse, the texture and the lake/land interface.
Running once the \verb=terrain= program, it will generate for you aside from the
\verb=DOMAIN= file a series of ASCII files you can modify with any text
editor. Running the \verb=terrain= program the second time and setting
a \verb=fudge= flag, will tell the program to overwrite the selected
variable with the modified value in the ASCII file. This can be useful
for sensitivity experiments in the BATS surface model or to design
a scenario experiment.
\item Some of the land surface types in BATS have been little tested and used
or are extremely simplified and thus should be used cautiously. Specifically
the types are: sea ice, bog/marsh, irrigated crop, glacier. If such types are
present in a domain, the user is advised to carefully check the model behavior
at such points and eventually substitute these types with others.
\item The \verb=inpter= directory is expected to contain a \verb=SURFACE=
directory where the actual netCDF global dataset are stored. The overall
path is limited to $256$ characters.
\label{pathnote}
\item If the netCDF library is compiled with OpenDAP support, an URL
can be used as a path in the \verb=dirter= and \verb=inpter= variables.
Note that the $256$ character limit for paths holds in the whole program.
\item The \verb=lsmoist= namelist entry triggers the interpolation on the
domain area of a global dataset of satellite measured surface soil moisture
to estimate the initial model soil moisture. A simple algorithm extend the
surface soil moisture to all the soil layers.
\item The \verb=moist_filename= if present triggers reading from the
specified file the soil moisture on all model vertical layers and use it to
initialize the soil moisture content in an initial run. Note that for this
a \verb=DOMAIN= file created for a CLM45 run cannot be used for a BATS run.
\end{enumerate}

\subsection{debugparam namelist}

This namelist is used by all RegCM programs to enable/disable some debug
printout.  In the current release this flag is honored only by the model
itself. If you are not a developer you may find this flags useless.

{\footnotesize
\begin{Verbatim}
 &debugparam
 debug_level = 0, ! Currently value of 2 and 3 control previous DIAG flag
 dbgfrq = 3,      ! Interval for printout if debug_level >= 3
 /
\end{Verbatim}
}

Just note that with current implementation, the output file syncing is left
to the netCDF library. If You want to examine step by step the output while
the model is running, set the \verb=debug_level= at value 3.

\subsection{boundaryparam namelist}

Being a limited area model, in order to be run RegCM4 requires the provision
of meteorological initial and time dependent lateral boundary conditions,
typically for wind components, temperature, water vapor and surface pressure.
These are obtained by interpolation from output from reanalysis of observations
or global climate model simulations, which thus “drive” the regional climate
model.

The lateral boundary conditions (LBC) are provided through the so called
relaxation/diffusion technique which consists of:

\begin{enumerate}
\item selecting a lateral buffer zone of n grid point width (\verb=nspgx=)
\item interpolating the driving large scale fields onto the model grid
\item applying the relaxation + diffusion term
\begin{equation}
\frac{\partial \alpha}{\partial t} = F(n)F_1 * (\alpha_{LBC}-\alpha_{mod}) -
    F(n)F2 * \Delta_2(\alpha_{LBC}-\alpha_{mod})
\end{equation}
where $\alpha$ is a prognostic variable (wind components, temperature, water
vapor, surface pressure). The first term on the rhs is a Newtonian relaxation
term which brings the model solution ($mod$) towards the LBC field ($LBC$)
and the second term diffuses the differences between model solution and LBC.
$F(n)$ is an exponential function given by:
\begin{equation}
F(n) = exp\left(\frac{-(n-1)}{anudge(k)}\right)
\end{equation}
Where $n$ is the grid point distance from the boundary (varying from $1$ to
$nspgx$): $n-1$ is the outermost grid point, $n=2$ the adjacent one etc.
The $anudge$ array determines the strength of the LBC forcing and depends on
the model level $k$. In practice $F(n)$ is equal to 1 at the outermost grid
point row and decreases exponentially to $0$ at the internal edge of the buffer
zone ($nspgd$) at a rate determined by $anudge$. Larger buffer zones and larger
values of $anudge$ will yield a greater forcing by the LBC.  
\end{enumerate}

Typically for domain sizes of $~100$ grid points we use a buffer zone width
of $10-12$ grid points, for large domains this buffer zone can increase to
values of $15$ or even $20$.

In the model $anudge$ has three increasing values from the lower, to the mid
and higher troposphere. For example for $nspgx = 10$ we use $anudge$ equals to
$1, 2, 3$ for the lower, mid and upper troposphere, respectively.

This allows a stronger forcing in the upper troposphere to insure a greater
consistency of large scale circulations with the forcing LBC while allowing
more freedom to the model in the lower troposphere where local high resolution
forcings (e.g. complex topography) are more important.

For nspgx of $15-20$, for example, $anudge$ values could be increased to
$2,3,4$. As a rule of thumb, the choice of the maximum $anudge$ value should
follow the conditions:

\begin{equation}
\frac{(nspgx-1)}{anudge(k)} \ge 3
\end{equation}

{\footnotesize
\begin{Verbatim}
 &boundaryparam
 nspgx  = 12, ! nspgx-1 represent the number of cross point slices on
              ! the boundary sponge or relaxation boundary conditions.
 nspgd  = 12, ! nspgd-1 represent the number of dot point slices on
              ! the boundary sponge or relaxation boundary conditions.
 high_nudge   =     3.0, ! Nudge value high range
 medium_nudge =     2.0, ! Nudge value medium range
 low_nudge    =     1.0  ! Nudge value low range
 /
\end{Verbatim}
}

\subsection{globdatparam namelist}
\label{globparam}

This namelist is used by the \verb=sst= and \verb=icbc= ICBC programs. You can
tell them how to build initial and bondary conditions.

{\footnotesize
\begin{Verbatim}
 &globdatparam
 ibdyfrq =     6,            ! boundary condition interval (hours)
 ssttyp = 'OI_WK',           ! Type of Sea Surface Temperature used
                             !  One in: GISST, OISST, OI2ST, OI_WK, OI2WK,
                             !          FV_A2, FV_B2, EH5A2, EH5B1, EHA1B,
                             !          EIN75, EIN15, ERSST, ERSKT, CCSST,
                             !          CA_XX, HA_XX, EC_XX, IP_XX, GF_XX,
                             !          CN_XX
 dattyp = 'EIN15',           ! Type of global analysis datasets used
                             !  One in: ECMWF, ERA40, EIN75, EIN15, EIN25,
                             !          ERAHI, NNRP1, NNRP2, NRP2W, GFS11,
                             !          FVGCM, FNEST, EH5A2, EH5B1, EHA1B,
                             !          CCSMN, ECEXY, CA_XX, HA_XX, EC_XX,
                             !          IP_XX, GF_XX, CN_XX, MP_XX
 chemtyp = 'MZCLM',          ! Type of Global Chemistry boundary conditions
                             ! One in : MZ6HR, 6 hours MOZART output
                             !        : MZCLM, MOZART climatology 1999-2009
 gdate1 = 1990060100,        ! Start date for ICBC data generation
 gdate2 = 1990070100,        ! End data for ICBC data generation
 calendar = 'gregorian',     ! Calendar to use (gregorian, noleap or 360_day)
 dirglob = 'input/',         ! Path for ICBC produced input files
 inpglob = 'globdata/',      ! Path for ICBC global input datasets.
 ensemble_run = .false.,     ! If this is a member of a perturbed ensemble
                             ! run. Activate random noise added to input
                   ! Look http://users.ictp.it/~pubregcm/RegCM4/globedat.htm
                   ! on how to download them.
 /
\end{Verbatim}
}

Things you need to know here:

\begin{enumerate}
\item The gdate time window to build ICBC must be always greater or equal to
the time window you plan to run the model in.
Different GCMs and reanalysis products have different length of the year.
For example, the reanalysis products employ the real year length (365 days +
real leap years, i.e. and average length of 365.2422), the CCSM has a length
of 365 days (no leap year), the HadCM has a length of 360 days (30 day months).
The RegCM4 length of the year has to be the same as in the forcing fields, and
this can be set in the variable \verb=dayspy=.
Please remember to always check the consistency of the length of the year.
\item Even if listed, not all the input engines are fully tested. Some of them
need data which have been reformatted by ICTP (they are not in the original
format with which they are distributed by the institution producing them).
Some input data are not freely distibutable by ICTP, and you need a special
agreement with the owner to use them.
Hopefully the situation is changing, and data exchange is becoming more and more
the basis for good science in the climatic field.
\item The \verb=chemtyp= paramter is read by the \verb=chem_icbc= program.
See below in \ref{chemparam} about chemistry boundary conditions.
\item For notes on path, you can see the above in terrainparam namelist
description at \ref{pathnote}.
\end{enumerate}

\subsection{fnestparam namelist}

This namelist is read if the \verb=FNEST= is selected as \verb=dattyp= in
\verb=globdatparam= namelist (see above in \ref{globparam}), and permits
the user to specify the output directory of the coarse resolution run
already completed and the name of the coarse domain, i.e. the \verb=domname=
namelist parameter used in the \verb=terrainparam= coarse namelist
(see above in \ref{terparam}).

The nested domain must be contained inside the coarse domain, and possibly
the nested domain should not overlap the boundary region of the coarse run.
Those checks are left to the user.

If nothing is specified or namelist not present default is to search for
a directory \verb=RegCM= inside \verb=inpglob= directory for files without
a \verb=domname=, i.e. like \verb=ATM.YYYYMMDDHH.nc=.

{\footnotesize
\begin{Verbatim}
!
! Nesting control
!
 &fnestparam
 coarse_outdir = 'globdata/RegCM', ! Coarse domain output dir if FNEST
 coarse_domname = 'EUROPE',        ! Coarse domain domname
 /
\end{Verbatim}
}

\subsection{perturbparam namelist}

This namelist lets you control to which input field and of what fractional
level a perturbation is added at ICBC stage on the input fields.
It is read by the ICBC program if the \verb=ensemble_run= parameter in the
\verb=globdatparam= namelist is set to \verb=true=.

{\footnotesize
\begin{Verbatim}
!
! Perturbation control for ensembles
!
 &perturbparam
 lperturb_topo = .false.,     ! Add perturbation to surface elevation
 perturb_frac_topo = 0.001D0, ! Fractional value of the perturbation on topo
 lperturb_ts = .false.,       ! Add perturbation to surface temeprature
 perturb_frac_ts = 0.001D0,   ! Fractional value of the perturbation on ts
 lperturb_ps = .false.,       ! Add perturbation to surface pressure
 perturb_frac_ps = 0.001D0,   ! Fractional value of the perturbation on ps
 lperturb_t  = .false.,       ! Add perturbation to temperature
 perturb_frac_t  = 0.001D0,   ! Fractional value of the perturbation on t
 lperturb_q  = .false.,       ! Add perturbation to humidity mixing ratio
 perturb_frac_q  = 0.001D0,   ! Fractional value of the perturbation on q
 lperturb_u  = .false.,       ! Add perturbation to zonal velocity
 perturb_frac_u  = 0.001D0,   ! Fractional value of the perturbation on u
 lperturb_v  = .false.,       ! Add perturbation to meridional velocity
 perturb_frac_v  = 0.001D0,   ! Fractional value of the perturbation on v
 /
\end{Verbatim}
}

Things you need to know here:

\begin{enumerate}
\item The \verb=perturb_frac= should not exceed a percent of the field value.
The algorithm detail of the applied noise can be found in \cite{tao_ense}.
\end{enumerate}

\subsection{restartparam namelist}

This namelist lets you control the time period the model is currently simulating
in this particular run. You may want to split longer runs for which you have
prepared the ICBC's into shorter runs, to schedule HPC resource usage in a more
collaborative way with other researcher sharing it: the regcm model allows
restart, so be friendly with other research projects which may not have this
fortune (unless you are late for publication).

{\footnotesize
\begin{Verbatim}
 &restartparam
 ifrest  = .false. ,   ! If a restart
 mdate0  = 1990060100, ! Global start (is gdate1, most probably)
 mdate1  = 1990060100, ! Start date of this run
 mdate2  = 1990060200, ! End date for this run
 /
\end{Verbatim}
}

Things you need to know here:

\begin{enumerate}
\item After the simulation starts, on restart NEVER change the \verb=mdate0=
value. The correct scheme for restart is:
\begin{itemize}
\item Set \verb=ifrest= to \verb=.true.=
\item Set \verb=mdate1= to the value in \verb=mdate2=
\item Define the new value for \verb=mdate2=
\end{itemize}
\item Consider that current RegCM convention is to place midnight of first
day of month as the last timestep in previous month, except on first model
output file (\verb=ifrest= = \verb=.false.=). It is for this reason better
to use as start and end time a month boundary. We usually consider a month
data file the basic unit of output, each time you cross a month a new output
file will be created for you.
\end{enumerate}

\subsection{timeparam namelist}

This namelist contains model internal timesteps, used by the model as basic
integration timestep and triggers for calling internal parametric schemes.

{\footnotesize
\begin{Verbatim}
 &timeparam
 dt     =   150., ! time step in seconds
 dtrad  =     0., ! time interval solar radiation calculated (minutes)
 dtsrf  =     0., ! time interval at which land model is called (seconds)
 dtcum  =     0., ! time interval at which cumuls is called (seconds)
 dtabem =     0., ! time interval absorption-emission calculated (hours)
 dtchem =   900., ! time interval for chemistry reactions (seconds)
 /
\end{Verbatim}
}

Things you need to know here:

\begin{enumerate}
\item If only dt is chosen, the other values are computed to align with dt
and output frequencies to have $~5minutes$ for cumulus, $~10minutes$ for
surface, $~30 minutes$ for radiation, $18hr$ for absorption-emission.
\item All the internal timesteps need to be multiples of the base
timestep $dt$.  Note that the units are different, so you need to convert the
other timesteps in seconds before the check.
\item Surface, radiation and cumulus timesteps must be aligned on a $24h$
time window.
\item The dynamical hydrostatical core of RegCM requires a fixed timestep,
and you need to manually find the correct value which permits not to break
the Courant–Friedrichs–Lewy condition considering \cite{CFL}.
A good rule of thumb is to have a \verb=dt= not greater than three times
the \verb=ds= value in $km$ specified in the \verb=geoparam= namelist at
\ref{geoparam}.
A greater value may lower computing time, but in case of strong advection
may lead to non accurate computation or even the violation of CFL condition
and the divergence of the solution.
\item If you hit a non stable condition, the restart capability of the model
may help find the correct timestep just for a particular period, using a
different timestep at different times.
\item The surface model timestep should be of the order of $10 minutes$ to
keep computational cost low: lower values may be used only in the case of
very strong surface gradients.
\item A negative value for $dtcum$ (the default) actually sets $dtcum = dt$,
calling the cumulus convection scheme every model timestep. For small $dt$,
this value should be set to $~5minute$.
\item The absorption-emission computations are the most expensive in
computational time in the radiation scheme, which accounts overall for
around $30\%$ of the total execution time. The $18 hours$ default
allows for a compromise for accuracy/cost and should avoid aliasing on a
daily timeframe window.
\end{enumerate}

\subsection{outparam namelist}

This namelist controls the model output engine, allowing you to enable/disable
any of the output file writeout, or to modify the frequency the fields are
written in the files.

{\footnotesize
\begin{Verbatim}
 &outparam
 ifsave  = .true. ,           ! Create SAV files for restart
 savfrq  =    48.,            ! Frequency in hours to create them
 ifatm   = .true. ,           ! Output ATM ?
 atmfrq  =     6.,            ! Frequency in hours to write to ATM
 ifrad   = .true. ,           ! Output RAD ?
 radfrq  =     6.,            ! Frequency in hours to write to RAD
 ifsts   = .true. ,           ! Output STS (frequence is daily) ?
 ifsrf   = .true. ,           ! Output SRF ?
 srffrq  =     3.,            ! Frequency in hours to write to SRF
 ifsub   = .true. ,           ! Output SUB ?
 subfrq  =     6.,            ! Frequency in hours to write to SUB
 iflak   = .true.,            ! Output LAK ?
 lakfrq  =     6.,            ! Frequency in hours to write to LAK
 ifchem  = .true.,            ! Output CHE ?
 ifopt   = .false.,           ! Output OPT ?
 chemfrq =     6.,            ! Frequency in hours to write to CHE
 enable_atm_vars = 67*.true., ! Mask to eventually disable variables ATM
 enable_srf_vars = 35*.true., ! Mask to eventually disable variables SRF
 enable_rad_vars = 25*.true., ! Mask to eventually disable variables RAD
 enable_sub_vars = 18*.true., ! Mask to eventually disable variables SUB
 enable_sts_vars = 18*.true., ! Mask to eventually disable variables STS
 enable_lak_vars = 18*.true., ! Mask to eventually disable variables LAK
 enable_opt_vars = 19*.true., ! Mask to eventually disable variables OPT
 enable_che_vars = 26*.true., ! Mask to eventually disable variables CHE
 dirout  = './output',        ! Path where all output will be placed
 lsync   = .false.,           ! If sync of output files at every timestep is
                              ! requested. Note, it has a performance impact.
                              ! Enabled by default if debug_level > 2
 idiag = 0,                   ! Enable tendency diagnostic output in the ATM
                              ! file. NOTE: output file gets HUGE.
 do_parallel_netcdf_in  = .false., ! This enables paralell input
                                   ! Each processors reads its slice in the
                                   ! input file. Enable ONLY in case of
                                   ! HUGE input bandwidth,
 do_parallel_netcdf_out = .false., ! This enables paralell output if the 
                                   ! hdf5/netcdf libraries support it and
                                   ! the model is compiled with :
                                   !    --enable-nc4-parallel
 /
\end{Verbatim}
}

Things you need to know here:

\begin{enumerate}
\item The surface fields are the mean values in the interval specified by
the frequency values. The dynamical fields are instead the point value at the
output time. Refer to the Reference Manual \cite{refman_11} for a detailed
description of the model output fields.
\item If the chemistry or lake model are not enabled, the values specified in
the control flags are not considered. If \verb=nsg= is not greater than one
in dimparam at \ref{dimparam}, the \verb=ifsub= flag is not considered.
\item For the output directory, the path variable has a limit of $256$
characters. This path must be a local path on disk where the user running
the model has write permissions granted.
\item The enablevar logical arrays can be used to avoid saving one of the time
dependent variables in the output file, in the order they are saved in the
output file itself. Note that geolocation and pressure variables cannot be
disabled.
\end{enumerate}

\subsection{physicsparam namelist}
\label{physicsparam}

This namelist controls the model physics. You have a number of option here,
and the best way to select the right set is to carefully read the the
Reference Manual \cite{refman_11}. We are for the purposes of this User Guide
not going in detail in here, except in saying that probably you will need
to run some experiments especially with different cumulus convection
schemes before finding out the best model setting.
Although the mixed convection scheme (Grell over land and Emanuel over ocean)
seems to provide an overall better performance, our experience is that there
is no scheme that works best everywhere, therefore we advice to always do some
sensitivity experiments to select the best scheme for your application.

{\footnotesize
\begin{Verbatim}
 &physicsparam
 iboudy  =          5,  ! Lateral Boundary conditions scheme
                        !   0 => Fixed
                        !   1 => Relaxation, linear technique.
                        !   2 => Time-dependent
                        !   3 => Time and inflow/outflow dependent.
                        !   4 => Sponge (Perkey & Kreitzberg, MWR 1976)
                        !   5 => Relaxation, exponential technique.
 isladvec =         0,  ! Semilagrangian advection scheme for tracers and
                        ! humidity
                        !   0 => Disabled
                        !   1 => Enable Semi Lagrangian Scheme
   iqmsl =          1,  ! Quasi-monotonic Semi Lagrangian
                        !   0 => Standard Semi-Lagrangian
                        !   1 => Bermejo and Staniforth 1992 QMSL scheme
 ibltyp  =          1,  ! Boundary layer scheme
                        !   0 => Frictionless
                        !   1 => Holtslag PBL (Holtslag, 1990)
                        !   2 => UW PBL (Bretherton and McCaa, 2004)
 icup_lnd =          4, ! Cumulus convection scheme Over Land
 icup_ocn =          4, ! Cumulus convection scheme Over Icean
                        !   1 => Kuo
                        !   2 => Grell
                        !   3 => Betts-Miller (1986) DOES NOT WORK !!!
                        !   4 => Emanuel (1991)
                        !   5 => Tiedtke (1996)
                        !   6 => Kain-Fritsch (1990), Kain (2004)
   igcc  =          2,  ! Grell Scheme (icup == 2) Cumulus closure scheme
                        !   1 => Arakawa & Schubert (1974)
                        !   2 => Fritsch & Chappell (1980)
 ipptls  =          1,  ! Moisture scheme
                        !   1 => Explicit moisture (SUBEX; Pal et al 2000)
                        !   2 => Explicit moisture Nogherotto/Tompkins
 iocncpl =          0,  ! Ocean SST from coupled Ocean Model through RegESM
                        !   1 => Coupling activated
 iwavcpl =          0,  ! Ocean roughness from coupled Wave Model through RegESM
                        !   1 => Coupling activated
 iocnflx =          2,  ! Ocean Flux scheme
                        !   1 => Use BATS1e Monin-Obukhov
                        !   2 => Zeng et al (1998)
                        !   3 => Coare bulk flux algorithm
   iocnrough =      1,  ! Zeng Ocean model roughness formula to use.
                        !   1 => (0.0065*ustar*ustar)/egrav
                        !   2 => (0.013*ustar*ustar)/egrav + 0.11*visa/ustar
			!   3 => (0.017*ustar*ustar)/egrav
			!   4 => Huang 2012 free convection and swell effects
			!   5 => four regime formulation
   iocnzoq =        1,  ! Zeng Ocean model factors for t,q roughness
                        !   1 => 2.67*(re**d_rfour) - 2.57
			!   2 => min(4.0e-4, 2.0e-4*re**(-3.3))
			!   3 => COARE formulation as in bulk flux above
 ipgf    =          0,  ! Pressure gradient force scheme
                        !   0 => Use full fields
                        !   1 => Hydrostatic deduction with pert. temperature
 iemiss  =          0,  ! Use computed long wave emissivity
 lakemod =          0,  ! Use lake model
 ichem   =          0,  ! Use active aerosol chemical model
 scenario =      'A1B', ! IPCC Scenario to use in A1B,A2,B1,B2
                        ! RCP Scenarios in RPC2.6,RCP4.5,RCP6,RCP8.5
 idcsst   =          0, ! Use diurnal cycle sst scheme
 iseaice  =          0, ! Model seaice effects
 idesseas =          0, ! Model desert seasonal albedo variability
 iconvlwp =          1, ! Use convective algo for lwp in the large-scale
                        ! This is reset to zero if using ipptls = 2
 icldfrac =          0, ! Cloud fraction algorithm
                        !   0 : Original SUBEX
                        !   1 : Xu-Randall empirical
 icldmstrat =        1, ! Simulate stratocumulus clouds
 icumcloud =         1, ! Formulas to use for cumulus clouds (cf and lwc)
                        ! Cloud fractions, only if mass fluxes are not
			! available (Kuo and BM):
			!   0,1 => cf = 1-(1-clfrcv)**(1/kdepth)
			!   2   => cf = cloud profile
			! Liquid water content:
			!   0   => constant in cloud
			!   1,2 => function of temperature
 irrtm    =          0, ! Use RRTM radiation scheme instead of CCSM
 iclimao3 =          0, ! Use O3 climatic dataset from SPARC CMIP5
 isolconst =         0, ! Use a constant 1367 W/m^2 instead of the prescribed
                        ! TSI recommended CMIP5 solar forcing data.
 islab_ocean =       0, ! Activate the SLAB ocean model
 itweak =            0, ! Enable tweak scenario
 \
\end{Verbatim}
}

\subsection{nonhydroparam namelist}

This namelist controls non-hydrostatic core for the upper radiative boundary
conditions and the base state temperature vertical profile curve.

{\footnotesize
\begin{Verbatim}
 &nonhydroparam
 logp_lrate = 50.0D0, ! Logp lapse rate d(T)/d(ln P) K/ln(Pa)
 ifupr = 0,           ! Upper radiative boundary condition (Klemp and Durran,
                      ! Bougeault, 1983)
 ckh = 1.0D0,         ! Background diffusion multiplication factor
 diffu_hgtf = 1,      ! Add topographic effect to diffusion
 /
\end{Verbatim}
}

\begin{enumerate}
\item The upper radiative boundary layer option has a high computational cost.
\end{enumerate}

\subsection{subexparam namelist}

This namelist controls the SUBEX moisture scheme. Please consider carefully
reporting in your work the tuning you perform on this parameters.
The parameters below are the ones currently used at ICTP.

{\footnotesize
\begin{Verbatim}
 &subexparam
 ncld      = 1,      ! # of bottom model levels with no clouds (rad only)
 qck1land  = 0.0005, ! Autoconversion Rate for Land
 qck1oce   = 0.0005, ! Autoconversion Rate for Ocean
 gulland   = 0.65,   ! Fract of Gultepe eqn (qcth) when prcp occurs (land)
 guloce    = 0.30,   ! Fract of Gultepe eqn (qcth) for ocean
 rhmax     = 1.01,   ! RH at whicn FCC = 1.0
 rhmin     = 0.01,   ! RH min value
 rh0land   = 0.80,   ! Relative humidity threshold for land
 rh0oce    = 0.90,   ! Relative humidity threshold for ocean
 tc0       = 238.0,  ! Below this temp, rh0 begins to approach unity
 cevaplnd  = 1.0e-5, ! Raindrop evap rate coef land [[(kg m-2 s-1)-1/2]/s]
 cevapoce  = 1.0e-5, ! Raindrop evap rate coef ocean [[(kg m-2 s-1)-1/2]/s]
 caccrlnd  = 6.0,    ! Raindrop accretion rate land  [m3/kg/s]
 caccroce  = 6.0,    ! Raindrop accretion rate ocean [m3/kg/s]
 cllwcv    = 0.3e-3, ! Cloud liquid water content for convective precip.
 clfrcvmax = 0.75,   ! Max cloud fractional cover for convective precip.
 cftotmax  = 0.75,   ! Max total cover cloud fraction for radiation
 conf      = 1.00,   ! Condensation efficiency
 rcrit     = 13.5,   ! Mean critical radius
 coef_ccn  = 2.0,    ! Geometric mean Diameter and standard deviation
 abulk     = 0.9,    ! Bulk activation ratio
 lsrfhack  = .false. ! Surface radiation hack
 /
\end{Verbatim}
}

We found that RegCM4 is especially sensitive to:

\begin{enumerate}
\item \verb=cevap= : increasing cevap will generally decrease precipitation
\item \verb=gulland=, \verb=guloce= : increase of guland/guloce will generally
lead to reduce precipitation
\end{enumerate}

\subsection{microparam namelist}

This namelist controls the new microphysics scheme.

{\footnotesize
\begin{Verbatim}
 &microparam
 stats = .false.,           ! Produce debug variables in output files
 budget_compute = .false.,  ! Verify enthalpy and moisture conservation
 nssopt = 1,                ! Supersaturation Computation
                            ! 0 => No scheme
                            ! 1 => Tompkins
                            ! 2 => Lohmann and Karcher
                            ! 3 => Gierens
 kautoconv = 4,             !  Choose the autoconversion paramaterization
                            ! => 1 Klein & Pincus (2000)
                            ! => 2 Khairoutdinov and Kogan (2000)
                            ! => 3 Kessler (1969)
                            ! => 4 Sundqvist
 ksemi = 1.0,               ! Implicit/Explicit control 
                            ! NOT ACTIVATED YET - IT DOES NOT WORK!
                            ! ksemi == 0 => scheme is fully explicit
                            ! ksemi == 1 => scheme is fully implicit
                            ! 0<ksemi<1  => scheme is semi-implicit
 vqxr = 4.0,                ! Rain fall speed (default is 4 m/s)
 vqxi = 0.15,               ! Ice fall speed (default is 0.15 m/s)
 vqxs = 1.0,                ! Snow fall speed (default is 1 m/s)
 zauto_rate_khair = 0.355,  ! Autoconversion coefficient for kautoconv=2
 zauto_rate_kessl = 1.e-3,  ! Autoconversion coefficient for kautoconv=3
 zauto_rate_klepi = 0.5e-3, ! Autoconversion coefficient for kautoconv=1
 rkconv = 1.666e-4,         ! Autoconversion coefficient for kautoconv=4
 rcovpmin = 0.1,            ! Minimum precipitation coverage
 rpecons = 5.547e-5,        ! Evaporation constant Kessler
 /
\end{Verbatim}
}

\subsection{grellparam, emanparam, tiedtkeparam and kfparam namelists}

You are allowed here to tune the convection scheme selected above in
\ref{physicsparam} with the \verb=icup_lnd= or \verb=icup_ocn= number if
selected number is $2, 4, 5$.

{\footnotesize
\begin{Verbatim}
 &grellparam
 gcr0 = 0.0020,      ! Conversion rate from cloud to rain
 edtmin      = 0.20, ! Minimum Precipitation Efficiency land
 edtmin_ocn  = 0.20, ! Minimum Precipitation Efficiency ocean
 edtmax      = 0.80, ! Maximum Precipitation Efficiency land
 edtmax_ocn  = 0.80, ! Maximum Precipitation Efficiency ocean
 edtmino     = 0.20, ! Minimum Tendency Efficiency (o var) land
 edtmino_ocn = 0.20, ! Minimum Tendency Efficiency (o var) ocean
 edtmaxo     = 0.80, ! Maximum Tendency Efficiency (o var) land
 edtmaxo_ocn = 0.80, ! Maximum Tendency Efficiency (o var) ocean
 edtminx     = 0.20, ! Minimum Tendency Efficiency (x var) land
 edtminx_ocn = 0.20, ! Minimum Tendency Efficiency (x var) ocean
 edtmaxx     = 0.80, ! Maximum Tendency Efficiency (x var) land
 edtmaxx_ocn = 0.80, ! Maximum Tendency Efficiency (x var) ocean
 shrmin      = 0.30, ! Minimum Shear effect on precip eff. land
 shrmin_ocn  = 0.30, ! Minimum Shear effect on precip eff. ocean
 shrmax      = 0.90, ! Maximum Shear effect on precip eff. land
 shrmax_ocn  = 0.90, ! Maximum Shear effect on precip eff. ocean
 pbcmax = 50.0,      ! Max depth (mb) of stable layer b/twn LCL & LFC
 mincld = 150.0,     ! Min cloud depth (mb).
 htmin = -250.0,     ! Min convective heating
 htmax = 500.0,      ! Max convective heating
 skbmax = 0.4,       ! Max cloud base height in sigma
 dtauc = 30.0D0      ! Fritsch & Chappell (1980) ABE Removal Timescale (min)
 /

 &emanparam
 minsig = 0.95,       ! Lowest sigma level from which convection can originate
 elcrit_ocn = 0.0011, ! Autoconversion threshold water content (g/g) (ocean)
 elcrit_lnd = 0.0011, ! Autoconversion threshold water content (g/g) (land)
 tlcrit = -55.0,      ! Below tlcrit auto-conversion threshold is zero
 entp = 1.5,          ! Coefficient of mixing in the entrainment formulation
 sigd = 0.05,         ! Fractional area covered by unsaturated dndraft
 sigs = 0.12,         ! Fraction of precipitation falling outside of cloud
 omtrain = 50.0,      ! Fall speed of rain (Pa/s)
 omtsnow = 5.5,       ! Fall speed of snow (Pa/s)
 coeffr = 1.0,        ! Coefficient governing the rate of rain evaporation
 coeffs = 0.8,        ! Coefficient governing the rate of snow evaporation
 cu = 0.7,            ! Coefficient governing convective momentum transport
 betae = 10.0,        ! Controls downdraft velocity scale
 dtmax = 0.9,         ! Max negative parcel temperature perturbation below LFC
 alphae = 0.2,        ! Controls the approach rate to quasi-equilibrium
 damp = 0.1,          ! Controls the approach rate to quasi-equilibrium
 epmax_ocn = 0.999,   ! Maximum precipitation efficiency (ocean)
 epmax_lnd = 0.999,   ! Maximum precipitation efficiency (land)
 /

 &tiedtkeparam
 iconv = 4,           ! Actual used scheme.
 entrmax = 1.75e-3,   ! Max entrainment iconv=[1,2,3]
 entrdd  = 3.0e-4,    ! Entrainment rate for cumulus downdrafts
 entrpen = 1.75e-3,   ! Entrainment rate for penetrative convection
 entrscv = 3.0e-4,    ! Entrainment rate for shallow convection iconv=[1,2,3]
 entrmid = 1.0e-4,    ! Entrainment rate for midlevel convection iconv=[1,2,3]
 cprcon = 1.0e-4,     ! Coefficient for determining conversion iconv=[1,2,3]
 detrpen = 0.75e-4,   ! Detrainment rate for penetrative convection iconv=4
 entshalp = 2.0,      ! shallow entrainment factor for entrorg iconv=4
 rcuc_lnd = 0.05,     ! Convective cloud cover for rain evporation iconv=4
 rcuc_ocn = 0.05,     ! Convective cloud cover for rain evporation iconv=4
 rcpec_lnd = 5.55e-5, ! Coefficient for rain evaporation below cloud iconv=4
 rcpec_ocn = 5.55e-5, ! Coefficient for rain evaporation below cloud iconv=4
 rhebc_lnd = 0.7,     ! Critical rh below cloud for evaporation iconv=4
 rhebc_ocn = 0.9,     ! Critical rh below cloud for evaporation iconv=4
 rprc_lnd = 1.4e-3,   ! conversion coefficient from cloud water iconv=4
 rprc_ocn = 1.4e-3,   ! conversion coefficient from cloud water iconv=4
 rcrit1   = 13.5,     ! Mean critical radius for ccn
 /

 &kfparam
 kf_entrate = 0.03,      ! Entrainment rate
 kf_min_pef = 0.2,       ! Minimum precipitation efficiency
 kf_max_pef = 0.9,       ! Maximum precipitation efficiency
 kf_dpp     = 150.0,     ! Start elevation for downdraft above cloud base (mb)
 kf_tkemax = 5.0,        ! Maximum turbolent kinetic energy in sub cloud layer
 kf_min_dtcape = 1800.0, ! Consumption time of CAPE low limit
 kf_max_dtcape = 3600.0, ! Consumption time of CAPE high limit
 /
\end{Verbatim}
}

Things you need to know here:

\begin{enumerate}
\item In case of mixed cumulus schemes (land/ocean), both the selected
schemes configuration namelist are read in.  Note in this case for the
schemes only the relevant (Ocean or Land) control values are used.
\item Minimum and maximum values of the fraction of reevaporated water in the
downdraft for the Grell scheme is essentially a measure of the precipitation
efficiency: increasing their value generally decrease convective precipitation.
\item Again, read carefully the Reference Manual before attempting any tuning,
and report in any work modification of this parameters.
\end{enumerate}

\subsection{holtslagparam namelist}

You are allowed here to tune the Holtslag PBL scheme selected above in
\ref{physicsparam} with the \verb=ibltyp= number if selected number is
$1$. 

{\footnotesize
\begin{Verbatim}
 &holtslagparam
 ricr_ocn = 0.25,  ! Critical Richardson Number over Ocean
 ricr_lnd = 0.25,  ! Critical Richardson Number over Land
 zhnew_fac = 0.25, ! Multiplicative factor for zzhnew in holtpbl
 ifaholtth10 = 1,  ! First approximation for obhukov length, th10 formula:
                   !      1 => 0.5 * (t+tg) * (1+0.61*q)
                   !      2 => (0.25*t + 0.75*tg) * (1+0.61*q)
                   !      3 => theta + hf/(k*us)*log(z/10)
                   ! t = air temp., tg = ground temp., q = wv mix. ratio
                   ! hf = total heat flux, z = elevation
                   ! theta = virt. pot. t
 ifaholt = 1,      ! th10 final adjustment:
                   !      0 => no adjustment
                   !      1 => max(th10,tg)
                   !      2 => min(th10,tg)
 /
\end{Verbatim}
}

\subsection{uwparam namelist}

You are allowed here to tune the UW PBL scheme selected above in
\ref{physicsparam} with the \verb=ibltyp= number if selected number is
$2$. 

{\footnotesize
\begin{Verbatim}
 &uwparam
 iuwvadv = 0,   ! 0=standard T/QV/QC advection, 1=GB01-style advection
                ! 1 is ideal for MSc simulation, but may have stability issues
 atwo = 15.0,   ! Efficiency of enhancement of entrainment by cloud evap.
                !  see Grenier and Bretherton (2001) Mon. Wea. Rev.
                !  and Bretherton and Park (2009) J. Clim.
 rstbl = 1.5,   ! Scaling parameter for stable boundary layer eddy length
                ! scale.  Higher values mean stronger mixing in stable 
                ! conditions
 czero = 5.869, ! Czero constant in UW PBL (eqn 44a and pgs 856-857)
 nuk = 5.0,     ! Multiplicative factor for diffusion coefficients
 /
\end{Verbatim}
}

\subsection{slabocparam namelist}

Here you define parameter and stage fot the Ocean q-flux adjusted mixed layer
model.

{\footnotesize
\begin{Verbatim}
 &slabocparam
 do_qflux_adj  = .false.,  ! Activate SLAB Ocean model surface fluxes adjust
                           ! from an already created climatology
 do_restore_sst = .true.,  ! Create during the run the SLAB Ocean model surface
                           ! fluxes climatology to be used in a subsequent run
 sst_restore_timescale = 5.0D0,  ! Time interval in days in building the
                                 ! q-flux adjustment
 mixed_layer_depth     = 50.0D0, ! Depth in meters of the Ocean mixed layer.
 /
\end{Verbatim}
}

\subsection{tweakparam namelist}

This namelist controls the tweaking of the model to obtain custom scenarioes,
Is enabled if itweak is $1$ in \ref{physicsparam}.

{\footnotesize
\begin{Verbatim}
&tweakparam
 itweak_sst = 0,              ! Enable adding sst_tweak to input TS
 itweak_temperature = 0,      ! Enable adding temperature_tweak to input T
 itweak_solar_irradiance = 0, ! Add solar_tweak to solar constant
 itweak_greenhouse_gases = 0, ! Multiply gas_tweak_factors to GG concentrations
 sst_tweak = 0.0D0,           ! In K
 temperature_tweak = 0.0D0,   ! In K
 solar_tweak = 0.0D0,         ! In W m-2 (1367.0 is default solar)
 gas_tweak_factors = 1.0D0, 1.0D0 , 1.0D0 , 1.0D0 , 1.0D0,
 !                   CO2    CH4     N2O     CFC11   CFC12
\end{Verbatim}
}


\subsection{rrtmparam namelist}

You are allowed here to tune the RRTM radiative scheme selected above in
\ref{physicsparam} with the \verb=irrtm= number if selected number is $1$. 

{\footnotesize
\begin{Verbatim}
 &rrtmparam
 inflgsw  = 2, ! 0 = use the optical properties calculated in prep_dat_rrtm
               !     (same as standard radiation)
               ! 2 = use RRTM option to calculate cloud optical properties
               !     from water path and cloud drop radius
 iceflgsw = 3, ! Flag for ice particle specification
               !   0 => ice effective radius, r_ec, (Ebert and Curry, 1992),
               !        r_ec must be >= 10.0 microns
               !   1 => ice effective radius, r_ec, (Ebert and Curry, 1992),
               !        r_ec range is limited to 13.0 to 130.0 microns
               !   2 => ice effective radius, r_k, (Key, Streamer Ref. Manual,
               !        1996), r_k range is limited to 5.0 to 131.0 microns
               !   3 => generalized effective size, dge, (Fu, 1996),
               !        dge range is limited to 5.0 to 140.0 microns
               !        [dge = 1.0315 * r_ec]
 liqflgsw = 1, ! Flag for liquid droplet specification
               !   0 => Compute the optical depths due to water clouds in ATM
               !        (currently not supported)
               !   1 => The water droplet effective radius (microns) is input
               !        and the optical depths due to water clouds are computed
               !        as in Hu and Stamnes, J., Clim., 6, 728-742, (1993).
 inflglw  = 2, ! Flag for cloud optical properties as above but for LW
 iceflglw = 3, ! Flag for ice particle specification as above but for LW
 liqflglw = 1, ! Flag for liquid droplet specification as above but for LW
 icld  = 1,    ! Cloud Overlap hypothesis
 irng = 1,     ! mersenne twister random generator for McICA COH
 /
\end{Verbatim}
}

\subsection{chemparam namelist}
\label{chemparam}

This namelist controls the chemistry and aerosol options in the RegCM model.

{\footnotesize
\begin{Verbatim}
 &chemparam
 chemsimtype = 'CBMZ    ', ! Which chemical tracers to be activated.
                           ! One in :
                           !   DUST   : Activate 4 dust bins scheme
                           !   SSLT   : Activate 2 bins Sea salt scheme
                           !   DUSS   : Activate DUST +SSLT
                           !   DU12   : Activate 12 dust bins scheme
                           !   CARB   : Activate 4 species black/anthropic
                           !            carbon simulations
                           !   SULF   : Activate SO2 and SO4 tracers
                           !   SUCA   : Activate both SUKF and CARB
                           !   AERO   : Activate all DUST, SSLT, CARB and SULF
                           !   CBMZ   : Activate gas phase and sulfate
                           !   DCCB   : Activate CBMZ +DUST +CARB
                           !   POLLEN : Activate POLLEN transport scheme
 ichsolver = 1,  ! Activate the gas phase chemical solver CBMZ
 ichsursrc = 1,  ! Enable the emissions fluxes.
 ichdrdepo = 1,  ! 1 = enable tracer surface dry deposition. For dust,
                 !     it is calculated by a size settling and dry
                 !     deposition scheme. For other aerosol,a dry
                 !     deposition velocity is simply prescribed further.
 ichebdy = 1,    ! Enable reading of chemical boundary conditions, otherwise
                 ! put 0 in boundary conditions.
 ichcumtra = 1,  ! 1 = enable tracer convective transport.
 ichremlsc = 1,  ! 1 = enable tracer rainout     
 ichremcvc = 1,  ! 1 = enable tracer washout 
 ichdustemd = 1, ! Choice for parametrisation of dust emission size distribution
                 ! 1 = use the standard scheme (Alfaro et al., Zakey et al.)
                 ! 2 = use a revised soil granulometry ref + 
                 !     Kok et al emission size distribution :
                 !     Menut et al.,2012; + Kok et al., 2011
 ichdiag = 0,    ! 1 = enable writing of additional tracer tendency
                 ! diagnostics in the output
 idirect   = 1,  ! CHoice to enable or not aerosol feedbacks on radiation and
                 ! dynamics (aerosol direct and semi direct effects)
                 ! possible choice 1 or 2:
                 ! 1 = no coupling to dynamic and thermodynamic. However
                 !     the clear sky surface and top of atmosphere 
                 !     aerosol radiative forcings are diagnosed.
                 ! 2 = allows aerosol feedbacks on radiative,
                 !     thermodynamic and dynamic fields.
 iindirect = 0,  ! Enable sulfate first indirect effect in radiation scheme
                 ! based on Qian et al., 2001 
 rdstemfac = 1.0,! Dust emission adjustment factor ( soil erodibility)
                 ! linearly reduce or increase the dust flux
 /
\end{Verbatim}
}

The \verb=chemsimtype= parameter select one in a number of fixed sets which
define the nature and number of chemical species and/or transported aerosols,
together with wich relevant scheme is to be used in the simulation.
The implemented possible simulation types for the aerosol/chemistry options
are:

\begin{enumerate}
  \item \verb=DUST= : Activate 4 dust bins scheme, with on line emission,
    transport and removal.
  \item \verb=SSLT= : Activate 2 sea salt bins scheme, with on line emission,
    transport and removal.
  \item \verb=DUSS= : Activate Dust and seasalt scheme, 6 tracers.
  \item \verb=DU12= : Activate 12 dust bins scheme
  \item \verb=CARB= : Activate 4 species organic and black carbon in both
    hydrophobic and hydrophilic aerosol scheme, with on line emission,
    transport and removal.
  \item \verb=SULF= : Activate SO2 and SO4 tracers with simple sulfate
    oxidation from oxidant climatology, with on line emission,
    transport and removal.
  \item \verb=SUCA= : Activate both SULF and CARB.
  \item \verb=AERO= : Activate all DUST, SSLT, CARB and SULF
  \item \verb=CBMZ= : Activate CBMZ gas phase only option : 37 tracer are
    considered here.
  \item \verb=DCCB= : Activate CBMZ +DUST +CARB 
         + sulfate-nitrate-ammonium aerosol calculated with the ISORROPIA 
         gas-aerosol thermodynamical scheme: 50 tracers are considered here.
\end{enumerate}

The more tracer are used, the heavier computationally are the simulations and
the outputs. The chemistry outputs consist of one netCDF file per tracer,
named explicitely and containing concentration fields + different diagnostics,
and one netCDF file giving the optical properties of the total aerosol mixture
i.e. aerosol optical depth and radiative forcing. For a big domain, this can
require a huge amount of disk space to store the model results.

We will now detail the steps required to run a chemistry/aerosol simulation with
the RegCM model.

\subsubsection{Pre Processing}

We need to perform a couple of operations in the pre-processing stage to prepare
input datasets for an aerosol/chemistry simulation.

\begin{enumerate}
  \item In the case of a \verb=DUST=, \verb=AERO= or \verb=DCCB= simulation,
    we need the model to prepare soil type dataset to be used for dust emission
    calculation at the \verb=terrain= program stage. The \verb=ltexture=
    parameter in the \verb=terrainparam= namelist (see above in \ref{terparam})
    should be set to \verb=.true.=.
  \item After having prepared the static and boundary condition data with the
    \verb=icbc= program for the atmosphere, we need also to define chemical
    emissions and chemical boundary conditions.
\end{enumerate}

The data needed for this second task come from different sources, both
measurements data or GCM model with a chemistry parametrization active.

\begin{enumerate}
  \item Emission dataset. The pre processor can manage CMIP5 RCP and IASA  anthropogenic
    emissions for present day and future emission. For this, the global RCP
    emission dataset have first been processed by ICTP
    to match the species considered in the chemical solver CBMZ, and to
    aggregate different sector of emissions that are present in the RCP fields
    (e.g. CO emission from biomass burning + fossil fuel + ship + $\ldots$).
    The resulting global files, as well as grid informations are publicly
    available on ICTP site :
    \begin{center}
    \verb=http://clima-dods.ictp.it/regcm4/RCP_EMGLOB_PROCESSED=
    \end{center}
    
  \item Chemical boundary conditions for important tracersare available through
    ICTP. We use monthly boundary condition coming from global simulations
    (CAM + EC-EARTH) for aerosols, following different RCP scenarios
    (HIST + future).
    For gas phase species, we now have 6 hourly chemical boundary 
    conditions issued from the MOZART CTM. 
    Alternatively, we can use  a climatology representative
    of monthly average concentrations over the period 2000-2007 coming from
    the MOZART CTM. 
    The oxidant fields, used in the simple
    sulfate scheme, is also a climatology coming from MOZART CTM.
    The data are available on ICTP site.
\end{enumerate}

The steps to prepare the chemistry boundary conditions data are the following:

\begin{enumerate}
  \item In case of a chemistry simulation type (\verb=CBMZ= or \verb=DCCB=), the
    global emission files must first be interpolate to the RegCM model grid
    using the following procedure:
    \begin{itemize}
      \item Create the RegCM model grid description file to be used by
        \verb=cdo= to calculate weghts for a conservative remapping
        interpolation:
        \begin{Verbatim}
        $> cd $REGCM_RUN
        $> ./bin/emcre_grid test_001.in
        \end{Verbatim}
      \item Interpolate the global data on the RegCM grid with the interpolation
        script:
        \begin{Verbatim}
        $> cd $REGCM_RUN
        $> ./bin/interp_emissions test_001.in
        \end{Verbatim}
    \end{itemize}
    The \verb=cdo= program installation is mandatory in this case to perform
    this step. The interpolation is mass conservative and is consistent for
    any ratio of model resolution/global emission resolution.
    Note the programs and script uses the same root path of \verb=terrain= and
    \verb=icbc= programs for input and data directory. By default, we expect
    the global emission to be at the same level than e.g. EIN15 in your data
    path layout.
    This results in a file named \verb=*_CHEMISS.nc= of monthly emission for
    the whole RCP period. You don’t need to reprocess the file if you change
    the date of your simulation, as long as you are in the RCP temporal windows
    (for now, Historical from 1990-2010) .
    Which scenario to use is controlled by the \verb=scenario= variable in the
    \verb=physicsparam= parameter namelist as above in \ref{physicsparam}
  \item In function of the value of the \verb=chemsimtype= parameter, the
    relevant boundary conditions will be produced on the RegCM domain by
    running :
        \begin{Verbatim}
        $> cd $REGCM_RUN
        $> ./bin/chem_icbc test_001.in
        \end{Verbatim}
    That will result in 6 hourly chemical boundary conditions in your input
    directory (\verb=*_CHBC*.nc= and/or \verb=*_AEBC*.nc=).
\end{enumerate}

\subsubsection{Run time parameters}
The other \verb=chemparam= namelist parameters, let you control run time behaviour
of the model chemistry and aerosol schemes.

\begin{enumerate}
  \item \verb=ichsolver= : relevant for gas-phase chemistry options, activate
    the chemical solver CBMZ. If different from $1$, there is no chemical
    reactions and the tracer are only emitted, transported and removed.
  \item \verb=ichsursrc= : if set different from $1$, the emission term is
    suppressed and only boundary conditions are generating tracer in the domain.
  \item \verb=ichdrdepo= : if set different from $1$, the dry deposition and
    sedimentation of tracers is disabled for chemistry species.
  \item \verb=ichremlsc= : if set different from $1$,  rainout of chemical
    species is disabled.
  \item \verb=ichremcvc= : if set different from $1$,  washout of chemical
    species is disabled.
  \item \verb=ichcumtra= : if set different form $1$, the convective transport
    of tracers is disabled.
  \item= \verb=ichdiag= : if set to $1$, the writing of additional diagnostics
    in the chemistry output is enabled. Particularly, all the 3D tendency terms
    (advection, turbulence, convection, boundary condition, chemistry, removal,
    etc.) of the tracer equation are outputted at the frequency \verb=ichfreq=.
    This is usefull for budget and sensitivity studies, as well as debugging.
    This potentially can generate HUGE files.
  \item \verb=idirect= : Enable aerosol feedbacks on radiation.
    \begin{itemize}
      \item if equal to $1$, only aerosol radiative forcing is calculated and
        outputted but there is NO aerosol radiative feedback on climate.
        This can be viewed as a control run option.
      \item if equal to $2$, there is a feedback of aerosol radiative forcing
        on climate fields, via perturbation of the temperature tendency.
        This can be view as the perturbed run option.
    \end{itemize}
  \item \verb=ichdustemd= : Choice for parametrisation of dust emission size
    distribution:
    \begin{itemize}
      \item if set to $1$, the standard scheme s used (Alfaro et al.,
        Zakey et al., 2006)
      \item if set to $2$, the revised soil granulometry + Kok et al., 2011
        emission size distribution is used.
    \end{itemize}
  \item \verb=rdstemfac= : Scaling factor (erodibility) for tuning  \verb=DUST= emission flux
\end{enumerate}

\subsubsection{Sparse notes}

\begin{enumerate}
  \item Outputs are in netCDF, so process with your favorite software.
  \item The flux and tendency variables, as well as radiative forcings,
    are accumulated and averaged between 2 output time steps (like
    precipitations). The concentration, burden and aerosol optical depth
    are instantaneous.
  \item The outputs size can be huge, especially for full chemistry and
    diagnostic options. In the future, we might have the choice of outputting
    selected variables only.
\end{enumerate}

Not every possible dynamical configuration has been tested for the chemistry
option, so bugs might appear: please report!
CLM enables to calculate on line biogenic volatile hydrocarbon emissions, as
well as chemical deposition that can be used in RegCM. There are some flags to
activate when compiling CLM, we will update the documentation when fully tested.

The Tiedke and Emmanuel schemes, when activated, offer a more detailed
treatment of convective transport than the simple mixing hypothesis used with
other schemes.
The UW planetary boundary layer option integrate directly the emission and
deposition flux terms as part of the calculation of trurbulent tracer tendency.

\section{The CLM options}
\label{clm}

We will now discuss from the user point of view how to use the model
setup which need to be activated at configure stage.

The \verb=CLM= option if activated allows the user to run a simulation using
the CLM surface model instead of the default BATS1e model. We will not here go
in deep in the difference between the two models, read the Reference Manual
for this.
The executable of the model is different in the case of the \verb=CLM=, and is
named \verb=regcmMPICLM=.
Note that in the CLM case only the MPI enabled compilation is supported (no
serial), and no subgridding is possible (\verb=nsg= is always $1$).

\subsubsection{Enable}

At configure stage (see \ref{modconf}), the option is to be enabled with the
right command line argument to the configure script

\begin{Verbatim}
  --enable-clm   Supply this option if you plan on using CLM option.
\end{Verbatim}

This will enable a preprocessing flag, and build a different model executable.
Note that no modifications are needed for any other part of the model, but
this triggers the building of another pre-processing program, \verb=clm2rcm=.

\subsubsection{Prepare and run}
\label{clmrun}

The \verb=CLM= configuration requires a separate namelist in the namelist input
file.

{\footnotesize
\begin{Verbatim}
 &clmparam
 dirclm = 'input/', ! CLM path to Input data produced by clm2rcm. If 
                    ! relative, It should be how to reach the Input dir
                    ! from the Run dir.
 clmfrq =  12.,     ! Frequency for CLM own output write
 imask  =   1,      ! For CLM, Type of land surface parameterization
                    !   1 => using DOMAIN.INFO for landmask (same as BATS)
                    !   2 => using mksrf_navyoro file landfraction for
                    !        landmask and perform a weighted average over
                    !        ocean/land gridcells; for example:
                    ! tgb = tgb_ocean*(1-landfraction)+tgb_land*landfraction
 /
\end{Verbatim}
}

Things you need to know here:

\begin{enumerate}
\item The \verb=inpter= path defined in \verb=terrainparam= namelist
described in \ref{terparam} is used also by the \verb=clm2rcm= program.
See at \ref{clmdata} how to obtain needed datasets.
\item The file \verb=pft-physiology.c070207= should be manually copied in the
\verb=dirclm= directory before running the model.
\item The clmfrq is relative to the output produced by the \verb=CLM= model
itself, and does not control the RegCM model output. To know the \verb=CLM=
output file content, refer to CLM 3.5 documentation.
\item The \verb'imask = 2' option cannot be used with the \verb'icup_lnd' or
\verb'icup_ocn' cumulus convection schemes $2$, which rely on the BATS1e
landmask.
\end{enumerate}

In the case of \verb=CLM= run, the user needs to run, after the \verb=terrain=
program, the \verb=clm2rcm= program, and copy the \verb=pft-physiology.c070207=
in the input directory:

\begin{Verbatim}
$> cd $REGCM_RUN
$> ./bin/terrain regcm.in
$> ./bin/clm2rcm regcm.in
$> cp $REGCM_GLOBEDAT/CLM/pft-physiology.c070207 input/
\end{Verbatim}

The \verb=clm2rcm= program interpolates global land characteristics datasets
to the RegCM projected grid. The content of the \verb=pft-physiology.c070207=
file are described in the \verb=pft-physiology.c070207.readme= file.
All the other pre-processing steps are just equal to the one detailed in
chapter \ref{tutorial}. To run the \verb=CLM= option in the RegCM model, just
substitute the executable name:

\begin{Verbatim}
$> mpirun -np 2 ./bin/regcmMPICLM regcm.in
\end{Verbatim}

Note that the \verb=CLM= land model is much heavier than the BATS1e model, and
computing time increases.

\section{The CLM 4.5 options}
\label{clm45}

We will now discuss from the user point of view how to use the model
setup which need to be activated at configure stage.

The \verb=CLM 4.5= option if activated allows the user to run a simulation using
the CLM version 4.5 surface model instead of the default BATS1e model.
We will not here go in deep in the difference between the two models, read
the Reference Manual for this.
The executable of the model is different in the case of the \verb=CLM 4.5=,
and is named \verb=regcmMPICLM45=.
Note that in the \verb=CLM 4.5= case only the MPI enabled compilation is
supported (no serial).

\subsubsection{Enable}

At configure stage (see \ref{modconf}), the option is to be enabled with the
right command line argument to the configure script

\begin{Verbatim}
  --enable-clm45  Supply this option if you plan on using CLM45 option.
\end{Verbatim}

This will enable a preprocessing flag, and build a different model executable.
Note that no modifications are needed for any other part of the model, but
this triggers the building of another pre-processing program, \verb=mksurfdata=.

\subsubsection{Prepare and run}
\label{clm45run}

The \verb=CLM45= configuration requires a separate entries in the namelist input
file.

{\footnotesize
\begin{Verbatim}
 &clm_inparm
 fpftcon = 'pft-physiology.c130503.nc',
 fsnowoptics = 'snicar_optics_5bnd_c090915.nc',
 fsnowaging = 'snicar_drdt_bst_fit_60_c070416.nc',
 hist_nhtfrq = 0,
 /
 &clm_soilhydrology_inparm
 h2osfcflag = 1,
 origflag = 0,
 /
 &clm_hydrology1_inparm
 oldfflag = 0,
 /
\end{Verbatim}
}

Things you need to know here:

\begin{enumerate}
\item The namelist \ref{globparam} is used also by the \verb=mksurfdata=
program. See at \ref{clm45data} how to obtain needed datasets.
\item The \verb=hist_nhtfrq= is relative to the output produced by the
\verb=CLM 4.5= model itself, and does not control the RegCM model output.
\item To know the \verb=CLM 4.5= output file content, refer to CLM 4.5
documentation.
\end{enumerate}

In the case of \verb=CLM 4.5= run, the user needs to run, after the
\verb=terrain= program, the \verb=mksurfdata= program.

\begin{Verbatim}
$> cd $REGCM_RUN
$> ./bin/terrain regcm.in
$> ./bin/mksurfdata regcm.in
$> cp $REGCM_GLOBEDAT/CLM/pft-physiology.c070207 input/
\end{Verbatim}

All the other pre-processing steps are just equal to the one detailed in
chapter \ref{tutorial}. To run the \verb=CLM 4.5= option in the RegCM model,
just substitute the executable name:

\begin{Verbatim}
$> mpirun -np 2 ./bin/regcmMPICLM45 regcm.in
\end{Verbatim}

\section{Sensitivity experiments hint}

Although the LBC forcing does provide a constraint for the model, as any RCM,
RegCM4 is characterized by a certain level of internal variability due to its
non-liner processes (e.g. convection).

For example, if small perturbations are introduced in the initial or lateral
boundary conditions, the model will generally produce different patterns of,
e.g. precipitation, that appear as (sometimes seemingly organized) noise when
compared to the control simulation.

This noise depends on domain size and climatic regimes, for example it is
especially pronounced in warm climate regimes (e.g. tropics or during the
summer season) and large doamins.

When doing for example sensitivity experiments to model modifications, e.g. to
land use change, this internal variability “noise” can be misinterpreted as a
model response to the factor modified.

Users of RegCM4 should be aware of this when they do sensitivity experiments.
The best way to filter out this noise is to perform ensembles of simulations
and lok at the ensemble averages to extract the real model response from the
noise.
