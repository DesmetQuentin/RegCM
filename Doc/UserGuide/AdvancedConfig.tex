%
% This file is part of ICTP RegCM model.
% Copyright (C) 2011 ICTP Trieste
% See the file COPYING for copying conditions.
%

We will examin in this chapter in more detail the namelist configuration file,
to give you the User a deeper knowledge of model capabilities.

\section{The commented namelist}

In this section we will show you the commented namelist input file you will
find under \verb=$REGCM_ROOT/Doc= with the name \verb=README.namelist= .
All model programs seen so far, with the exception of the GrADS helper program,
use as input this namelist file, which is unique to a particular simulation.
The model input namelist file is divided in stanzas, each one devoted to
configuring the model capabilities.
A stanza in the namelist is identified with a starting \verb=&= character
followed by stanza name, and ends on a single line with the \verb=\=
character.

\subsection{dimparam stanza}

This stanza contains the base X,Y,Z domain dimension information, used
by the model dynamic memory allocator to request the Operating System the
memory space to store the model internal variables.

{\footnotesize
\begin{Verbatim}
 &dimparam
 iy     = 34,  ! This is number of points in the N/S direction
 jx     = 48,  ! This is number of points in the E/W direction
 kz     = 18,  ! Number of vertical levels: supported are 14, 18 and 23
 nsg    = 1,   ! For subgridding, number of points to decompose. If nsg=1,
               ! no subgridding is performed. CLM does NOT work as of now with
               ! subgridding enabled.
 /
\end{Verbatim}
}

The things you need to know here:

\begin{enumerate}
\item In the current version 4.1 the model parallelize execution dividing
the work between the processors along the jx (longitude) dimension. The
minimum work per processor is 3 points along the jx dimension, so the maximum
number of processors which can be used in a parallel run for the above
configuration is just $16$. In future revision ICTP plans to introduce 2D
decomposition.
\item In the future model revision the kz (vertical) dimension will be
more configurable, but for now you are limited to have 14, 18 and 23 levels.
We at ICTP normally use 18 levels.
\item Specifying an nsg number greater than one triggers the subgrid BATS model
on. There is no plan to extend this feature to CLM model. This affects only
surface variable calculations. All dynamics variable are calculated still
on the coarser grid. Rain in the current implementation is also calculated on
the coarser grid.
\end{enumerate}

\subsection{geoparam stanza}

This stanza is used by the \verb=terrain= program to geolocate the model grid
on the earth surface. The RegCM model uses a limited number of projection
engines. The value here are used by the other model programs to assert
consistency with the geolocation information written by the \verb=terrain=
program in the \verb=DOMAIN= file.

{\footnotesize
\begin{Verbatim}
 &geoparam
 iproj = 'LAMCON', ! Domain cartographic projection. Supported values are:
                   ! 'LAMCON', Lambert conformal.
                   ! 'POLSTR', Polar stereographic. (Doesn't work)
                   ! 'NORMER', Normal  Mercator.
                   ! 'ROTMER', Rotated Mercator.
 ds = 60.0,        ! Grid point horizontal resolution in km
 ptop = 5.0,       ! Pressure of model top in cbar
 clat = 45.39,     ! Central latitude  of model domain in degrees
                   ! North hemisphere is positive
 clon = 13.48,     ! entral longitude of model domain in degrees
                   ! West is negative.
 plat = 45.39,     ! Pole latitude (only for rotated Mercator Proj)
 plon = 13.48,     ! Pole longitude (only for rotated Mercator Proj)
 truelatl = 30.0,  ! Lambert true latitude (low latitude side)
 truelath = 60,    ! Lambert true latitude (high latitude side)
 /
\end{Verbatim}
}

The things you need to know here:

\begin{enumerate}
\item The different projection engines produce better results depending on the
position and extent of the domain. In particular, regardless of emisphere:
\begin{itemize}
\item Middle latitudes (around 45 degrees) - Lambert Conformal
\item Polar latitudes (more than 75 degrees) - Polar Stereographic
\item Low latitudes (up to 30 degrees and crossing the equator) - Mercator
\item Crossing more than 45 degrees extent in latitude - Rotated Mercator
\end{itemize}
\item The model hydrostatic engine does not allow resolution lower than
$20 km$. If want higher resolution consider using the subgridding scheme.
ICTP plans to introduce in the future a non-hydrostatic compressible core to
the RegCM model.
\item Lowering the top pressure of the model can give you problems in regions
with complex topography. Touch the default after thinking twice on that.
\item Always specify \verb=clat= and \verb=clon=, the central domain point,
and do fine adjustment of the position moving it around a little bit. A
little shift in position and some tests can help you obtain a better
representation of coastlines and topography at the coarse resolutions.
\item If using \verb=LAMCON= projection, take care to place the two
true latitudes at around one fourth and three fourth of the domain latitude
space to better correct the projection distorsion of the domain.
\item The pole position for the rotated mercator position should be as near as
possible to the center domain position.
\end{enumerate}

\subsection{aerosolparam stanza}

This stanza allows the user to specify aerosol usage in the model. It does
enable building of soil texture database in the \verb=terrain= program and
control the dimension of the sumber of optical active tracers used in the
active chemistry tracers part of the model.

{\footnotesize
\begin{Verbatim}
 &aerosolparam
 aertyp = 'AER00D0' ! Aerosol dataset used
                    ! One in :
                    ! AER00D0 -> Neither aerosol, nor dust used
                    ! AER01D0 -> Biomass, SO2 + BC + OC, no dust
                    ! AER10D0 -> Anthropogenic, SO2 + BC + OC, no dust
                    ! AER11D0 -> Anthropogenic+Biomass, SO2 + BC + OC, no dust
                    ! AER00D1 -> No aerosol, with dust
                    ! AER01D1 -> Biomass, SO2 + BC + OC, with dust
                    ! AER10D1 -> Anthropogenic, SO2 + BC + OC, with dust
                    ! AER11D1 -> Anthropogenic+Biomass, SO2 + BC + OC, with dust
 ntr =  4,          ! Tracer parameters: number of tracers
 nbin = 2,          ! Tracer parameters: bins number for dust
 /
\end{Verbatim}
}

The things you need to know here:

\begin{enumerate}
\item If \verb=aertype= is left to \verb=AER00D0=, it is nonsense to activate
chemistry in the model.
\item The total number of tracers activated must be greater than nbin.
\item If Anthropogenic and/or Biomass is activated, the model will also need
the user to run the aerosol program. It can be run at the same level of the
sst program, with the same calling syntax. Just replace sst with aerosol.
\begin{Verbatim}
$> cd $REGCM_RUN
$> ./Bin/aerosol myregcm.in
\end{Verbatim}
The aerosol program preapares an emission dataset used by the model
to consider optical active chemistry species effects in the radiation
calculation. The surface dust emission are calculated using the soil
texture dataset prepared by the \verb=terrain= program if the last
$0$ is set to $1$.
\end{enumerate}

\subsection{terrainparam stanza}

\label{aertyp}
This stanza is used by the \verb=terrain= program to know how you want him
to generate the \verb=DOMAIN= file. You can control its work using a number of
parameter to obtain what you consider the best representation of the
physical reality. Do not underestimate what you can do at this early stage,
having a good representation of the surface can lead to valuable results
later when the model calculates climatic parameters.

{\footnotesize
\begin{Verbatim}
 &terrainparam
 domname  = 'AQWA',      ! Name of the domain. Controls naming of input files
 ntypec = 5,             ! Resolution of the global terrain and landuse data
                         ! Use 60, for  1  degree resolution
                         !     30, for 30 minutes resolution
                         !     10, for 10 minutes resolution
                         !      5, for  5 minutes resolution
                         !      3, for  3 minutes resolution
                         !      2, for  2 minutes resolution
 ntypec_s = 2,           ! Same for subgrid (Used only if nsg > 1)
 smthbdy = .false.,      ! Smoothing Control flag
                         !  true  -> Perform extra smoothing in boundaries
 lakedpth    = .false.,  ! If using lakemod (see below), produce from
                         ! terrain program the domain bathymetry
 fudge_lnd   = .false.,  ! Fudging Control flag, for landuse of grid
 fudge_lnd_s = .false.,  ! Fudging Control flag, for landuse of subgrid
 fudge_tex   = .false.,  ! Fudging Control flag, for texture of grid
 fudge_tex_s = .false.,  ! Fudging Control flag, for texture of subgrid
 fudge_lak   = .false.,  ! Fudging Control flag, for lake of grid
 fudge_lak_s = .false.,  ! Fudging Control flag, for lake of subgrid
 h2opct = 75.,           ! Surface minimum H2O percent to be considered water
 dirter = '../../Input', ! Output directory for terrain files
 inpter = '../DATA',     ! Input directory for SURFACE dataset
 /
\end{Verbatim}
}

The things you need to know here:

\begin{enumerate}
\item The \verb=domname= will control output file naming convention, all
generated file will prepend this string to the old V3 naming convention,
giving you the capability to recognize different run. Try to use always
meaningful names.
\item In version 4.1 does exist a single input dataset, the 30s one. The
\verb=ntypec= parameter controls initial subsampling of input dataset before
the smoothing interpolation performed by the \verb=terrain= program.
\item Use lakedepth if you plan to use the Hostetler lake model in the
model. It will be useless otherwise. It may be used in the future to
have a common sea bathymetry with the ocean coupled model. The coupling
engine of RegCM will be included in future model releases.
\item You can control the final land-water mask using the h2opct parameter.
This parameter can be used to have more land point than calculated by
the simple interpolation engine. Try it with different values to find best
land shapes. A zero value does mean use just interpolation engine, higher
values will extend into ocean points the land at land-water interface.
\item A number of flag control the capability of the \verb=terrain= program
to modify on request the class type variables in the \verb=DOMAIN= file. You can
modify on request the landuse, the texture and the lake/land interface.
Running once the \verb=terrain= program, it will generate for you aside the
\verb=DOMAIN= file a series of ASCII files you can modify with any text
editor. Running the \verb=terrain= program the second time and setting
a \verb=fudge= flag, will tell the program to overwrite the selected
variable with the modified value in the ASCII file. This can be useful
for sensitivity experiments on the BATS surface model or to design
a scenario experiment.
\item If the NetCDF library is compiled with OpenDAP support, an URL
can be used as a path in the \verb=dirter= and \verb=inpter= variables.
Note that the $256$ character limit for paths holds in the whole program.
For \verb=terrain= program you may want to try the following URL: \\
http://clima-dods.ictp.it/thredds/dodsC
\item The texture dataset is built if the aerosol model is activated.
This is controlled by the \verb=AERTYP= flag. See above in \ref{aertyp}.
\end{enumerate}

\subsection{ioparam stanza}

{\footnotesize
\begin{Verbatim}
 &ioparam
 ibyte = 4,   ! Number of bytes in reclen. Usually 4
 /
\end{Verbatim}
}

Leave this untouched. The model expects input record syze to be 4 bytes.
You will need to change some compilation parameters if you change this value.
\footnote{This namelist stanza will be removed in future versions}

\subsection{debugparam stanza}

This stanza is used by all RegCM programs to enable/disable some debug printout.
In the current release this flag is honored only by the model itself. If you
are not a developer you may find this flags useless.

{\footnotesize
\begin{Verbatim}
 &debugparam
 debug_level = 1, ! Currently value of 2 and 3 control previous DIAG flag
 dbgfrq = 3,      ! Interval for printout if debug_level >= 3
 /
\end{Verbatim}
}

