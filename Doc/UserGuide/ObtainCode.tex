%%::::::::::::::::::::::::::::::::::::::::::::::::::::::::::::::::::::::::::::::
%%
%%    This file is part of ICTP RegCM.
%%    
%%    Use of this source code is governed by an MIT-style license that can
%%    be found in the LICENSE file or at
%%
%%         https://opensource.org/licenses/MIT.
%%
%%    ICTP RegCM is distributed in the hope that it will be useful,
%%    but WITHOUT ANY WARRANTY; without even the implied warranty of
%%    MERCHANTABILITY or FITNESS FOR A PARTICULAR PURPOSE.
%%
%%::::::::::::::::::::::::::::::::::::::::::::::::::::::::::::::::::::::::::::::

\section{Simple Model User}

The Software source code for the RegCM model is available through GitHub:

\begin{Verbatim}
https://github.com/ICTP/RegCM
\end{Verbatim}

New users can download the packaged files:

\begin{Verbatim}
https://github.com/ICTP/RegCM/releases
\end{Verbatim}

or get the CORDEX experiment codebase:

\begin{Verbatim}
https://github.com/ICTP/RegCM/archive/refs/heads/CORDEX-5.zip
\end{Verbatim}

or get the latest development code by using git command line programs:

\begin{Verbatim}
$> git clone https://github.com/ICTP/RegCM.git
\end{Verbatim}


\section{Model Developer}

If you plan to become a model developer, it is suggested to fork the RegCM
repository:

\begin{Verbatim}
https://t.ly/iP0G-
\end{Verbatim}

The RegCM team strongly encourages the contributing developers to frequently
pull the modifications from the master repository, because the RegCM is
actively developed by the ICTP and collaborators.

Check that {\bf git} software is installed on your machine typing
the following command:

\begin{verbatim}
$> git --version
\end{verbatim}
