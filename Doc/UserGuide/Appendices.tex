%
% This file is part of ICTP RegCM model.
% Copyright (C) 2011 ICTP Trieste
% See the file COPYING for copying conditions.
%

We will review here a sample installation session of software needed to
install the RegCM model.

Starting point is here a Linux system on a multicore processor box, and final
goal is to have an optimized system to run the model.
I will use \verb=bash= as my shell and assume that GNU development tools like
\verb=make=, \verb=sed=, \verb=awk= are installed as part of the default
Operating System environment as it is the case in most Linux distro.
I will require also for commodity a command line web downloader such as
\verb=curl= installed on the system, along its development libraries to be
used to enable OpenDAP remote data access protocol capabilities of netCDF
library. Standard file management tools such as \verb=tar= and \verb=gzip=
are also required.
The symbol \verb=$>= will stand for a shell prompt.
I will assume that the process is performed as a normal system user, which
will own all the toolchain. I will be now just the \verb=regcm= user.

\section{Identify Processor}

First step is to identify the processor to know its capabilities:

\begin{Verbatim}
$> cat /proc/cpuinfo
\end{Verbatim}

This command will ask to the operating system to print processor informations.
A sample answer on my laptop is:

\begin{Verbatim}
processor       : 0
vendor_id       : GenuineIntel
cpu family      : 6
model           : 30
model name      : Intel(R) Core(TM) i7 CPU       Q 740  @ 1.73GHz
stepping        : 5
cpu MHz         : 933.000
cache size      : 6144 KB
physical id     : 0
siblings        : 8
core id         : 0
cpu cores       : 4
apicid          : 0
initial apicid  : 0
fpu             : yes
fpu_exception   : yes
cpuid level     : 11
wp              : yes
flags           : fpu vme de pse tsc msr pae mce cx8 apic sep mtrr pge
mca cmov pat pse36 clflush dts acpi mmx fxsr sse sse2 ss ht tm pbe syscall
nx rdtscp lm constant_tsc arch_perfmon pebs bts rep_good nopl xtopology
nonstop_tsc aperfmperf pni dtes64 monitor ds_cpl vmx smx est tm2 ssse3
cx16 xtpr pdcm sse4_1 sse4_2 popcnt lahf_lm ida dts tpr_shadow vnmi
flexpriority ept vpid
bogomips        : 3467.81                                                       
clflush size    : 64                                                            
cache_alignment : 64                                                            
address sizes   : 36 bits physical, 48 bits virtual                             
power management:
\end{Verbatim}

repeated eight time with Processor Ids from 0 to 7: I have a Quad Core Intel
with Hyperthreading on (this multiply by 2 the reported processor list).
The processor reports here also to support Intel Streaming SIMD Extensions V4.2,
which can be later used to speed up code execution vectorizing floating point
operation on any single CPU core.

\section{Chose compiler}

Depending on the processor, we can chose which compiler to use. On a Linux box,
we have multiple choices:

\begin{itemize}
\item GNU Gfortran
\item G95
\item Intel ifort compiler
\item Portland pgf90 compiler
\item Absoft ProFortran
\item NAG Fortran Compiler
\end{itemize}

and for sure other which I may not be aware of. Any of this compilers has pros
and cons, so I am just for now selecting one in the pool only to continue the
exposition. I am not selecting the trivial solution of Gfortran as most
Linux distributions have it already packaged, and all the other required
software as well (most complete distribution I am aware of for this is Fedora:
all needed software is packaged and it is a matter of yum install).

So let us assume I have licensed the Intel Composer XE Professional Suite
12.0.2 on my laptop. My system administrator installed it on the default
location under \verb=/opt/intel=, and I have my shell environment update
loading vendor provided script:

\begin{Verbatim}
$> source /opt/intel/bin/compilervars.sh intel64
\end{Verbatim}

With some modification (the path, the script, the arguments to the script),
same step is to be performed for all non-GNU compilers in the above list, and
is documented in the installation manual of the compiler itself.

In case of Intel, to check the correct behaviour of the compiler, try to
type the following command:

\begin{Verbatim}
$> ifort --version
ifort (IFORT) 12.0.2 20110112
Copyright (C) 1985-2011 Intel Corporation.  All rights reserved.

\end{Verbatim}

I am skipping here any problem that may arise from license installation for any
of the compilers, so I am assuming that if the compiler is callable, it works.
As this step is usually performed by a system administrator on the machine,
I am assuming a skilled professional will take care of that.

\section{Environment setup}

To efficiently use the compilers, I will setup now some environment variables.
On my system (see the above processor informations) I will use:

\begin{Verbatim}
$> # Where all the software will be installed ?
$> # I am chosing here a place under my home directory.
$> export INTELROOT=/home/regcm/intelsoft
$> export INTELSRC=/home/regcm/intelsoft/src
$> mkdir -p $INTELROOT/{bin,include,lib,share/man,src}
$> # the C compiler. I am assuming here to have the whole Intel
$> # Composer XE suite, so I will use the intel C compiler.
$> export CC=icc
$> # the C++ compiler, the intel one.
$> export CXX=icpc
$> # the Fortran 9X compiler.
$> export FC=ifort
$> # the Foirtran 77 compiler. For intel, is just the fortran one.
$> export F77=ifort
$> # C Compiler flags
$> export CFLAGS="-O3 -xHost -axSSE4.2 -fPIC"
$> # F9X Compiler flags
$> export FCFLAGS="-O3 -xHost -axSSE4.2 -fPIC"
$> # F77 Compiler flags
$> export FFLAGS="-O3 -xHost -axSSE4.2 -fPIC"
$> # CXX Compiler flags
$> export CXXFLAGS="-O3 -xHost -axSSE4.2 -fPIC"
$> # Linker flags
$> export LDFLAGS="-Wl,-rpath=$INTELROOT/lib \
> -Wl,-rpath=/opt/intel/lib/intel64 -i-dynamic"
$> # Preset PATH to use the installed software during build
$> export PATH=$INTELROOT/bin:$PATH
$> export LD_LIBRARY_PATH=$LD_LIBRARY_PATH:$INTELROOT/lib
$> export MANPATH=$INTELROOT/share/man:$MANPATH
\end{Verbatim}

This step will allow me not to specify those variables at every following step.
Depending on the above compiler selected, those flags my differ for you, but the
concept is that I am selecting a performance target build for the machine I am
on. I am now ready to compile software.

\section{Compression Library Installation}

To have a complete optimized stack, I will compile also here the compression
libraries optionally needed by HDF5 library.

I will need the sources:

\begin{Verbatim}
$> cd $INTELSRC
$> curl -o zlib-1.2.5.tar.gz http://zlib.net/zlib-1.2.5.tar.gz
$> curl -o szip-2.1.tar.gz \
> http://www.hdfgroup.org/ftp/lib-external/szip/2.1/src/szip-2.1.tar.gz
\end{Verbatim}

Please note that the szip library license allows to use that software only
together with HDF5 library.

Then I must decompress the sources:

\begin{Verbatim}
$> cd $INTELSRC
$> tar -zxvf zlib-1.2.5.tar.gz
$> tar -zxvf szip-2.1.tar.gz
\end{Verbatim}

Let us start with zlib:

\begin{Verbatim}
$> cd $INTELSRC
$> cd zlib-1.2.5
$> LDSHARED="icc -shared -Wl,-soname \
> -Wl,libz.so.1,--version-script,zlib.map" \
> ./configure --prefix=$INTELROOT
$> make
$> make check
$> make install
\end{Verbatim}

This will install the zlib software under \verb=/home/regcm/intelsoft=.
Let us continue with szip.

\begin{Verbatim}
$> cd $INTELSRC
$> cd szip-2.1
$> ./configure --prefix=$INTELROOT
$> make
$> make check
$> make install
\end{Verbatim}

This will install the szip software under \verb=/home/regcm/intelsoft=.

\section{HDF5 Library installation}

Download the source pack:

\begin{Verbatim}
$> cd $INTELSRC
$> curl -o hdf5-1.8.6.tar.bz2 \
> ftp://ftp.hdfgroup.org/HDF5/current/src/hdf5-1.8.6.tar.bz2
\end{Verbatim}

Install it:

\begin{Verbatim}
$> cd $INTELSRC
$> cd hdf5-1.8.6
$> ./configure --prefix=$INTELROOT --enable-hl --enable-linux-lfs \
> --enable-production --with-pic --docdir=$INTELROOT/share/doc/hdf5/ \
> --with-szlib=$INTELROOT --with-zlib=$INTELROOT
$> make
$> make check
$> make install
\end{Verbatim}

\section{netCDF Library installation}

Download the source pack:

\begin{Verbatim}
$> cd $INTELSRC
$> curl -o netcdf-4.1.2.tar.gz \
> http://www.unidata.ucar.edu/downloads/netcdf/ftp/netcdf-4.1.2.tar.gz
\end{Verbatim}

Install it:

\begin{Verbatim}
$> cd $INTELSRC
$> cd netcdf-4.1.2
$> ./configure --prefix=$INTELROOT --enable-shared --enable-netcdf-4 \
> --with-udunits --with-libcf --enable-dap-netcdf --enable-cxx-4 \
> --with-hdf5=$INTELROOT --with-zlib=$INTELROOT --with-szlib=$INTELROOT
$> make
$> make check
$> make install
\end{Verbatim}

\section{OpenMPI library installation}

This optional step will install OpenMPI message passing library to enable
parallel run of the RegCM model using all cores of my processor.
Download the souece pack:

\begin{Verbatim}
$> cd $INTELSRC
$> curl -o openmpi-1.4.3.tar.bz2 \
> http://www.open-mpi.org/software/ompi/v1.5/downloads/openmpi-1.4.3.tar.bz2
\end{Verbatim}

Install it:

\begin{Verbatim}
$> cd $INTELSRC
$> cd openmpi-1.4.3
$> ./configure --prefix=$INTELROOT --sysconfdir=$INTELROOT/etc/openmpi \
> --mandir=$INTELROOT/share/man --libdir=$INTELROOT/lib --enable-mpi-f90 
$> make
$> make check
$> make install
\end{Verbatim}

\section{Final step}

To enable all the ready installed software to be used by the regcm user whenever
it logs in the box, edit the \verb=.bashrc= file in the home directory and add
the following lines:

\begin{Verbatim}
export PATH=$INTELROOT/bin:$PATH
export LD_LIBRARY_PATH=$LD_LIBRARY_PATH:$INTELROOT/lib
export MANPATH=$INTELROOT/share/man:$MANPATH
\end{Verbatim}
