%
% This file is part of ICTP RegCM model.
% Copyright (C) 2011 ICTP Trieste
% See the file COPYING for copying conditions.
%

The new netCDF output format allows users to use a number of general purpose
tools to postprocess model output files. We will in this section do a quick
review of some of the Open Source and Free Software ones.

\section{Command line tools}

Three major set of tools may help you do even complex calculation just from
command line prompt. 

\subsection{netCDF library tools}

The netCDF library itself offers three basic tools to play with netCDF
archived data.

\begin{itemize}
\item \verb=ncdump= program, generates a text representation of a specified
netCDF file on standard output. The text representation is in a form called
CDL (‘‘network Common Data form Language’’) that can be viewed, edited, or
serve as input to \verb=ncgen=, thus \verb=ncdump= and \verb=ncgen= can be
used as inverses to transform data representation between binary and text
representations. \verb=ncdump= may also be used as a simple browser for netCDF
datasets, to display the dimension names and lengths; variable names, types,
and shapes; attribute names and values; and optionally, the values of data
for all variables or selected variables in a netCDF dataset. 
Sample usage patterns:
\begin{enumerate}
\item Look at the structure of the data in the netCDF dataset:
\begin{Verbatim}
ncdump -c test_001_SRF.1990060100.nc
\end{Verbatim}
\item Produce a fully-annotated (one data value per line) listing of the data
for the variables time and tas, using FORTRAN conventions for indices, and show
the floating-point data with only four significant digits of precision and the
time values with ISO format:
\begin{Verbatim}
ncdump -v time,tas -p 4 -t -f \
         fortran test_001_SRF.1990060100.nc
\end{Verbatim}
\end{enumerate}
\item \verb=ncgen= program, the reverse of the \verb=ncdump= program: generates
a netCDF file or a C or FORTRAN program that creates a netCDF dataset from a
CDL input.
Sample usage patterns:
\begin{enumerate}
\item From a CDL file, generate a binary netCDF file:
\begin{Verbatim}
ncgen -o test_001_SRF.1990060100_modif.nc \
         test_001_SRF.1990060100.cdl
\end{Verbatim}
\item From a CDL file, generate a Fortran program to write the netCDF file:
\begin{Verbatim}
ncgen -f test_001_SRF.1990060100.cdl > prog.f
\end{Verbatim}
\end{enumerate}
\item \verb=nccopy= utility copies an input netCDF file to an output netCDF
file, in any of the four format variants, if possible, and in function of
the selected output format add compression filter and/or data chunking.
Sample usage patterns:
\begin{enumerate}
\item Convert a netCDF dataset to a netCDF 4 classic model compressed
data file using shuffling to enhance compression level:
\begin{Verbatim}
nccopy -k 4 -d 9 -s test_001_SRF.1990060100.nc \
         test_001_SRF.1990060100_compressed.nc
\end{Verbatim}
\end{enumerate}
\end{itemize}

You can also find, in the \verb=Tools/Programs/RegCM_read= directory
under \verb=$REGCM_ROOT= a sample program to read an output file using
the netCDF library you can modify to fit your needs.
Another sample program to read SAV output file format can be found
under the directory \verb=SAV_read= in the \verb=Tools/Programs= directory.

\subsection{NetCDF operators NCO}

This set of tools can be considered a swiss army knife to manage netCDF
datasets. There are multiple operators, and Each operator takes netCDF
files as input, then operates (e.g., derives new data, averages, hyperslabs,
manipulates metadata) and produces a netCDF output file.
The single-command style of NCO allows users to manipulate and analyze files
interactively, or with simple scripts that avoid some overhead of higher level
programming environments.
The major tools are:
\begin{itemize}
\item \verb=ncap2= netCDF Arithmetic Processor 
\item \verb=ncatted= netCDF Attribute Editor
\item \verb=ncbo= netCDF Binary Operator
\item \verb=ncea= netCDF Ensemble Averager
\item \verb=ncecat= netCDF Ensemble Concatenator
\item \verb=ncflint= netCDF File Interpolator
\item \verb=ncks= netCDF Kitchen Sink
\item \verb=ncpdq= netCDF Permute Dimensions Quickly, Pack Data Quietly
\item \verb=ncra= netCDF Record Averager
\item \verb=ncrcat= netCDF Record Concatenator
\item \verb=ncrename= netCDF Renamer
\item \verb=ncwa= netCDF Weighted Averager
\end{itemize}

A comprehensive user guide can be found at:

\begin{Verbatim}
http://nco.sourceforge.net/nco.html
\end{Verbatim}

Sample usage patterns:
\begin{enumerate}
\item Get value of tas variable at a particular point for all timesteps with
a prescribed format one per line on stdout:
\begin{Verbatim}
ncks -C -H -s "%6.2f\n" -v tas -d iy,16 -d jx,16 \
         test_001_SRF.1990060100.nc
\end{Verbatim}
\item Extract one timestep of tas from a file and save into a new netCDF file:
\begin{Verbatim}
ncks -c -v tas -d time,6 test_001_SRF.1990060100.nc \
       test_001_SRF.1990060212.nc
\end{Verbatim}
\item Cat together a year worth of output data for the single tas variable
into a single file:
\begin{Verbatim}
ncrcat -c -v tas test_001_SRF.1990??0100.nc \
         test_001_T2M.1990.nc
\end{Verbatim}
\item Get the DJF mean value of the tempertaure from a multiyear run:
\begin{Verbatim}
ncra -c -v tas test_001_SRF.????120100.nc \
               test_001_SRF.????010100.nc \
               test_001_SRF.????020100.nc \
               test_001_DJF_T2M.nc
\end{Verbatim}
\end{enumerate}

We strongly encourage you to read the on-line user guide of the NCO tools.
You will for sure get a boost on your data manipulation and analysis skills.

\subsection{Climate data Operators CDO}

The monolithic \verb=cdo= program from the Max Planck Institut f\''ur
Meteorologie implements a really comprehensive collection of command line
Operators to manipulate and analyse Climate and NWP model Data either in netCDF
or GRIB format. There are more than 400 operators available, covering the
following topics:

\begin{itemize}
\item File information and file operations
\item Selection and Comparision
\item Modification of meta data
\item Arithmetic operations
\item Statistical analysis
\item Regression and Interpolation
\item Vector and spectral Transformations
\item Formatted I/O
\item Climate indices
\end{itemize}

We wont make here a comprehensive analysis of this tool, but you can
find some ideas in the \verb=PostProc= directory on \verb=$REGCM_ROOT= reading
the two sample \verb=average= and \verb=regrid= scripts, which use a combination
of \verb=NCO= programs and \verb=cdo= operators to reach goal.
A very simple usage pattern for example to obtain a monthly mean is:

\begin{Verbatim}
cdo monmean test_001_T2M.1990.nc
\end{Verbatim}

\section{GrADS program}

This tool is the one mostly used at ICTP to analyze and plot model output
results. It can be used either as an interactive tool either as a batch
data analysis tool. We have already written in chapter \ref{tutorial} about
the helper program \verb=GrADSNcPlot= which can be used to interactively
plot model output results. We will here detail why an helper program is
needed and how it does work. For information regarding the \verb=grads=
program itself, a comprehensive guide may be found at:

\begin{Verbatim}
http://www.iges.org/grads/gadoc/users.html
\end{Verbatim}

\subsection{GrADS limits}

The \verb=grads= program is powerful, yet has limits:

\begin{enumerate}
\item Only the equirectangular projection or Plate Carr\'ee is supported. Some
other projections can be used through a pdef entry in the \verb=CTL= file
using the internal direct preprojection engines, but not all RegCM supported
projections are supported using direct engine.
\item NetCDF format allows multidimensional variables, while \verb=grads=
supports just four dimensional (time,level,latitude,longitude) variables.
\end{enumerate}

Luckily, these limits can be exceeded, carefully telling \verb=grads= the
RegCM data structure using the \verb=CTL= file and one ancillary \verb=proj=
file:

\begin{enumerate}
\item The \verb=grads= program allows usage of the \verb=pdef BILIN= option
in the \verb=CTL= file, which allows the user to specify a supplementary file
name. In this file are stored three lat-lon floating-point grids which have
for each point on the equirectangular grid the indexes i,j on the projected
grid, as well as wind rotation values.
\item The \verb=grads= program allows identifying four dimentional slices of
a multidimensional variable as new variables, providing them a unique name.
This is how we are able to see in \verb=grads= chemical output variables.
\end{enumerate}

While the \verb=GrADSNcPlot= program allows interactive plotting and after
quitting the \verb=grads= program removes the \verb=CTL= file and the
\verb=proj= file, the \verb=GrADSNcPrepare= program only creates this two
files, allowing share of the \verb=proj= file between multiple \verb=CTL=
files sharing the same RegCM domain (i.e. it creates just only once the
\verb=proj= file). To use the \verb=grads= program, you need to have both
this ancillary files together with the data netCDF file.

To create the \verb=CTL= file for the history \verb=CLM= output file, you need
to provide to the helper programs the path to both the history \verb=CLM= and
the \verb=RegCM= \verb=DOMAIN= file, as in:

\begin{Verbatim}
$ GrADSNcPrepare clmoutput.clm2.h0.2000-07-30-00000.nc test_DOMAIN000.nc
\end{Verbatim}

A collection of sample \verb=grads= scripts commonly used at ICTP to plot
simulation results can be found in the \verb=Tools/Scripts/GrADS=
directory under \verb=$REGCM_ROOT=.

\section{CISL's NCL : NCAR Command Language}

This awesome tool from NCAR is an interpreted language designed for scientific
data analysis and visualization. Noah Diffenbaugh and Mark Snyder have
created a website dedicated to visualizing RegCM3 output using the NCAR Command Language (NCL). These scripts where built using RegCM3 model output
converted to netCDF using an external converter.
They have been adapted to serve as very basic example scripts to process a
native RegCM 4.2 output data file or do some data analysis using the \verb=NCL=
language and are available in the \verb=Tools/Scripts/NCL/examples= directory.
Travis O'Brien from the User Community also contributed sample scripts,
which may be found under the \verb=Tools/Scripts/NCL= directory.

\section{R Statistical Computing Language}

The \verb=R= statistical computing language is able with an add on package
to load into interal data structure a meteorological field read from a
netCDF RegCM output.
A sample script to load and plot the 2m Temperature at a selected timestep
can be used as a reference to develop a real powerful statistical analysis
of model results: it is under \verb=Tools/Scripts/R=.

\section{Non free tools}

Note that the netCDF format, using plugins or native capabilities, allows
clean access to model output from a number of non free tools like
Matlab\texttrademark or IDL\texttrademark.

For a more complete list of tools, you are invited to scroll down the
very long list of tools at:

\vspace{0.5cm}
\begin{tabular}{|c|}
\hline
{\bf http://www.unidata.ucar.edu/software/netcdf/software.html} \\
\hline
\end{tabular}
\vspace{0.5cm}
