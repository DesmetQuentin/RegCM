%
% This file is part of ICTP RegCM model.
% Copyright (C) 2011 ICTP Trieste
% See the file COPYING for copying conditions.
%

First step to run a test simulation is to obtain static data to localize
model DOMAIN and Atmosphere and Ocean global model dataset to build initial
and boundary conditions ICBC to run a local area simulation.

ICTP does mantain a public accessible web repository of datasets on:

{\bf http://users.ictp.it/~pubregcm/RegCM4/globedat.htm}

As of now You are requested to download required global data on Your local disk
storage before any run attempt. In the next future the ICTP ESP team has
plannend to make available an OpenDAP RAMADDA Server to give remote access
to global dataset for creating DOMAIN and ICBC without the need to
download the global dataset, but just the required subset in space and time,
using the ICTP web server capabilities to create that subset.

\section{Global dataset directory Layout}

You are suggested to establish a convenient location for global datasets
on Your local storage. Keep in mind that required space for a Year of global
data can be as large as 8 GBytes.

Having this in mind, we will now consider that You the user have identified
on Your system or have network access to such a storage resource to store say
100 GB of data, and have it reachable on Your system under the
\verb=$REGCM_GLOBEDAT= location.
On this directory, You are required to make the following directories:

\begin{Verbatim}
$> cd $REGCM_GLOBEDAT
$> mkdir SURFACE SST AERGLOB EIN15
\end{Verbatim}

This does not fill all possible global data sources paths, but will be enough
for the scope of running the model for testing its capabilities.

\section{Static Surface Dataset}

The model needs to be localized on a particular DOMAIN. The needed informations
are topography, land type classification and optionally lake depth (to run
Hostetler lake model) and soil texture classification (to run chemistry option
with DUST enabled).

This means downloading four files, which are global archives at $30 second$
horizontal resolution on a global latitude-longitude grid of the above data.

\begin{Verbatim}
$> cd $REGCM_GLOBEDAT
$> cd SURFACE
$> curl -o GTOPO_DEM_30s.nc.gz \
> http://clima-dods.ictp.it/data/d4/SURFACE/GTOPO_DEM_30s.nc.gz
$> gunzip GTOPO_DEM_30s.nc.gz
$> curl -o GLCC_BATS_30s.nc.gz \
> http://clima-dods.ictp.it/data/d4/SURFACE/GLCC_BATS_30s.nc.gz
$> gunzip GLCC_BATS_30s.nc.gz
\end{Verbatim}

Optional Lake and Texture datasets:

\begin{Verbatim}
$> cd $REGCM_GLOBEDAT
$> cd SURFACE
$> curl -o ETOPO_BTM_30s.nc.gz \
> http://clima-dods.ictp.it/data/d4/SURFACE/ETOPO_BTM_30s.nc.gz
$> gunzip ETOPO_BTM_30s.nc.gz
$> curl -o GLZB_SOIL_30s.nc.gz \
> http://clima-dods.ictp.it/data/d4/SURFACE/GLZB_SOIL_30s.nc.gz
$> gunzip GLZB_SOIL_30s.nc.gz
\end{Verbatim}

\section{Aerosol Database}

If You are planning to enable chemistry in the model, You will need a single
file, which contains sources of chemical species used in the model regridded
from global model run.

\begin{Verbatim}
$> cd $REGCM_GLOBEDAT
$> cd AERGLOB
$> curl -o AEROSOL.dat \
> http://clima-dods.ictp.it/data/d4/AEROSOL/AEROSOL.dat
\end{Verbatim}

This is the input file for the \verb=aerosol= icbc program.

\section{Sea Surface Temperature}

The model needs a global SST dataset to feed the model with ocean temperature.
You have multiple choices for SST source:

\begin{enumerate}
\item GISST - UKMO SST (Rayner et al 1996), 1 degree from
{\bf http://www.badc.rl.ac.uk} UKMO DATA archive reformed as direct access
binary format from the original ASCII format.
\item OISST - CAC Monthly Optimal Interpolation dataset in the original 
NetCDF format.
\item OI2ST - Same as above, but both SST and Sea Ice dataset (used if seaice
option is enabled in the model).
\item OI\_WK - OISST CAC Weekly Optimal Interpolation dataset in the original
NetCDF format
\item OI2WK - Same as above, but both SST and Sea Ice dataset
\item EH5RF - EC-MPI 6 hourly 1.875x1.875, reference from 1941 to 2000
\item EH5A2 - Same as above, from 2001 to 2100 IPCC A2 scenario
\item EH5B1 - Same as above, from 2001 to 2100 IPCC B1 scenario
\item EHA1B - Same as above, from 2001 to 2100 IPCC A1B scenario
\item ERSST - ERA interim Project 6 hourly 1.5x1.5 degree SST
\item ERSKT - ERA interim as above but Skin temperature
\item FV\_RF - HadAMH\_SST in the original netCDF format, from 1959 to 1991
\item FV\_A2 - Same as above, IPCC A2 scenario
\item FV\_B2 - Same as above, IPCC B2 scenario 
\item CCSST - CCSM3 POP gx1v3 regridded 1x1 data
\end{enumerate}

We will for now for our test run download just CAC OISST weekly for the
period 1981 - present.

\begin{Verbatim}
$> cd $REGCM_GLOBEDAT
$> cd SST
$> curl -o sst.wkmean.1981-1989.nc \
> ftp://ftp.cdc.noaa.gov/pub/Datasets/noaa.oisst.v2/sst.wkmean.1981-1989.nc
$> curl -o sst.wkmean.1990-present.nc \
> ftp://ftp.cdc.noaa.gov/pub/Datasets/noaa.oisst.v2/sst.wkmean.1990-present.nc
\end{Verbatim}

\section{Atmosphere and Land temperature Global Dataset}
