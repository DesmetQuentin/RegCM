%
% This file is part of ICTP RegCM model.
% Copyright (C) 2011 ICTP Trieste
% See the file COPYING for copying conditions.
%

The first step to run a test simulation is to obtain static data to localize
model DOMAIN and Atmosphere and Ocean global model dataset to build initial
and boundary conditions ICBC to run a local area simulation.

ICTP maintains a public accessible web repository of datasets on:

{\bf http://users.ictp.it/~pubregcm/RegCM4/globedat.htm}

As of now you are requested to download required global data on your local disk
storage before any run attempt. In the future, the ICTP ESP team has
planned to make available an OpenDAP THREDDS Server to give remote access
to global dataset for creating DOMAIN and ICBC without the need to
download the global dataset, but just the required subset in space and time,
using the ICTP web server capabilities to create that subset.

\section{Global dataset directory Layout}

You are suggested to establish a convenient location for global datasets
on your local storage. Keep in mind that required space for a year of global
data can be as large as 8 GBytes.

Having this in mind, we will now consider that you the user have identified
on your system or have network access to such a storage resource to store say
100 GB of data, and have it reachable on your system under the
\verb=$REGCM_GLOBEDAT= location.
On this directory, you are required to make the following directories:

\begin{Verbatim}
$> cd $REGCM_GLOBEDAT
$> mkdir SURFACE CLM SST AERGLOB EIN15
\end{Verbatim}

This does not fill all possible global data sources paths, but will be enough
for the scope of running the model for testing its capabilities.

\section{Static Surface Dataset}

The model needs to be localized on a particular DOMAIN. The needed information
are topography, land type classification and optionally lake depth (to run the
Hostetler lake model) and soil texture classification (to run the chemistry option
with DUST enabled).

This means downloading four files, which are global archives at $30 second$
horizontal resolution on a global latitude-longitude grid of the above data.

\begin{Verbatim}
$> cd $REGCM_GLOBEDAT
$> cd SURFACE
$> curl -o GTOPO_DEM_30s.nc.gz \
> http://clima-dods.ictp.it/data/d4/SURFACE/GTOPO_DEM_30s.nc.gz
$> gunzip GTOPO_DEM_30s.nc.gz
$> curl -o GLCC_BATS_30s.nc.gz \
> http://clima-dods.ictp.it/data/d4/SURFACE/GLCC_BATS_30s.nc.gz
$> gunzip GLCC_BATS_30s.nc.gz
\end{Verbatim}

Optional Lake and Texture datasets:

\begin{Verbatim}
$> cd $REGCM_GLOBEDAT
$> cd SURFACE
$> curl -o ETOPO_BTM_30s.nc.gz \
> http://clima-dods.ictp.it/data/d4/SURFACE/ETOPO_BTM_30s.nc.gz
$> gunzip ETOPO_BTM_30s.nc.gz
$> curl -o GLZB_SOIL_30s.nc.gz \
> http://clima-dods.ictp.it/data/d4/SURFACE/GLZB_SOIL_30s.nc.gz
$> gunzip GLZB_SOIL_30s.nc.gz
\end{Verbatim}

\section{Aerosol Database}

If you are planning to enable aerosol in the model, you will need a single
file, which contains sources of optical active species used in the model
regridded from global model run.

\begin{Verbatim}
$> cd $REGCM_GLOBEDAT
$> cd AERGLOB
$> curl -o AEROSOL.dat \
> http://clima-dods.ictp.it/data/d4/AEROSOL/AEROSOL.dat
\end{Verbatim}

This is the input file for the \verb=aerosol= icbc program.

\section{CLM Dataset}
\label{clmdata}

If you are planning to enable the \verb=CLM= option in the model, you will need
a series of files with global land surface characteristics datasets.

\begin{Verbatim}
$> cd $REGCM_GLOBEDAT
$> cd CLM
$> CLMURL="clima-dods.ictp.it/data/d4/CLM"
$> curl -o mksrf_fmax.nc.gz \
> http://$CLMURL/mksrf_fmax.nc.gz
$> curl -o mksrf_glacier.nc.gz \
> http://$CLMURL/mksrf_glacier.nc.gz
$> curl -o mksrf_lai.nc.gz \
> http://$CLMURL/mksrf_lai.nc.gz
$> curl -o mksrf_lanwat.nc.gz \
> http://$CLMURL/mksrf_lanwat.nc.gz
$> curl -o mksrf_navyoro_20min.nc.gz \
> http://$CLMURL/mksrf_navyoro_20min.nc.gz
$> curl -o mksrf_pft.nc.gz \
> http://$CLMURL/mksrf_pft.nc.gz
$> curl -o mksrf_soicol_clm2.nc.gz \
> http://$CLMURL/mksrf_soicol_clm2.nc.gz
$> curl -o mksrf_soitex.10level.nc.gz \
> http://$CLMURL/mksrf_soitex.10level.nc.gz
$> curl -o mksrf_urban.nc.gz \
> http://$CLMURL/mksrf_urban.nc.gz
$> curl -o pft-physiology.c070207.gz \
> http://$CLMURL/pft-physiology.c070207.gz
$> curl -o pft-physiology.c070207.readme.gz \
> http://$CLMURL/pft-physiology.c070207.readme.gz
$> curl -o rdirc.05.061026.gz \
> http://$CLMURL/rdirc.05.061026.gz
$> gunzip *.gz
\end{Verbatim}

This is the input file for the \verb=clm2rcm= program (see at \ref{clm}).

\section{Sea Surface Temperature}

The model needs a global SST dataset to feed the model with ocean temperature.
You have multiple choices for SST data:

\begin{enumerate}
\item {\em GISST} - UKMO SST (Rayner et al 1996), 1 degree from \\
{\bf http://www.badc.rl.ac.uk} \\ UKMO DATA archive reformed as direct access
binary format from the original ASCII format.
\item {\em OISST} - CAC Monthly Optimal Interpolation dataset in the original 
netCDF format.
\item {\em OI2ST} - Same as above, but both SST and Sea Ice dataset (used if
seaice option is enabled in the model).
\item {\em OI\_WK} - OISST CAC Weekly Optimal Interpolation dataset in the
original netCDF format
\item {\em OI2WK} - Same as above, but both SST and Sea Ice dataset
\item {\em EH5RF} - EC-MPI 6 hourly 1.875x1.875, reference from 1941 to 2000
\item {\em EH5A2} - Same as above, from 2001 to 2100 IPCC A2 scenario
\item {\em EH5B1} - Same as above, from 2001 to 2100 IPCC B1 scenario
\item {\em EHA1B} - Same as above, from 2001 to 2100 IPCC A1B scenario
\item {\em ERSST} - ERA interim Project 6 hourly 1.5x1.5 degree SST
\item {\em ERSKT} - ERA interim as above but Skin temperature
\item {\em FV\_RF} - HadAMH\_SST in the original netCDF format, from 1959
to 1991
\item {\em FV\_A2} - Same as above, IPCC A2 scenario
\item {\em FV\_B2} - Same as above, IPCC B2 scenario 
\item {\em CCSST} - CCSM3 POP gx1v3 regridded 1x1 data
\end{enumerate}

We will for now for our test run download just CAC OISST weekly for the
period 1981 - present.

\begin{Verbatim}
$> cd $REGCM_GLOBEDAT
$> cd SST
$> CDCSITE="ftp.cdc.noaa.gov/pub/Datasets/noaa.oisst.v2"
$> curl -o sst.wkmean.1981-1989.nc \
> ftp://$CDCSITE/sst.wkmean.1981-1989.nc
$> curl -o sst.wkmean.1990-present.nc \
> ftp://$CDCSITE/sst.wkmean.1990-present.nc
\end{Verbatim}

\section{Atmosphere and Land temperature Global Dataset}

The model needs to build initial and boundary conditions for the regional scale,
interpolating on the RegCM grid the data from a Global Climatic Model output.
The GCM dataset can come from any of the supported models:

\begin{enumerate}
\item {\em EINXX} - ECMWF INTERIM 10 year reanalysis datasets,
XX can have values 25, 15 or 75 for resolution 2.5x2.5L37, 1.5x1.5L37,
0.75x0.75L37. Time resolution is 4 times daily.
\item {\em ECMWF} - ECMWF TOGA/WCRP Uninitialized Data - (ECWCRP).
Reformatted by PWC/ICTP to direct-access binary, T42L15, Gaussian Grid.
\footnote{As of RegCM 4.1 this input source is not tested}
\item {\em ERA40} - ECMWF 40 year reanalysis datasets, available from \\
{\bf http://data.ecmwf.int/data/d/era40\_daily}, \\
Pressure levels, 2.5x2.5L23, 4 times daily.
\item {\em ERAHI} - ECMWF 40 year reanalysis datasets, origigal model level
fields: T, U, V and log(Ps) are in spectral coefficients, Oro and Q are at the
reduced Gaussian grids. T159L60 (N80L60).
\item {\em NNRPY} - NCEP/NCAR Reanalysis datasets, Y can have values 1 and 2
for the two reanalysis experimens. Data are available in the original netCDF
format at \\ {\bf ftp://ftp.cdc.noaa.gov/Datasets/ncep.reanalysis} \\
($1948 - present$, 2.5x2.5L13) and \\
{\bf ftp://ftp.cdc.noaa.gov/Datasets/ncep.reanalysis2} \\
($1979 - 2009$, 2.5x2.5L13). 
\item {\em NRP2W} - Small Window (instead of global) of NNRP1/2 to save disk
space. This window can be created from original files with NCO tools.
\footnote{As of RegCM 4.1 this input source is not tested}
\item {\em GFS11} - NCEP Global Forecast System (GFS) product FNL, from \\
{\bf http://dss.ucar.edu/datasets/ds083.2/data/fnl-yyyymm}, \\
Pressure levels, 1.0x1.0L27, 4 times daily.
\item {\em FVGCM} - FVGCM run by the PWC group of Abdus Salam ICTP.
\footnote{As of RegCM 4.1 this input source is not tested}
\item {\em EH5XX} - Echam run by the MPI at Hamburg, T63, Gaussian grid.
For present day  run: $1941 - 2000$, for A1B scenario run: $2001 - 2100$. 
17 pressure levels, 4 times daily, direct-access binary.
\item {\em CCSMN} - unpacked CCSM3 NETCDF L26 (six hourly) data, either global
or window, can be obtained from \\ {\bf http://www.earthsystemgrid.org}
\item {\em FNEST} - Further oneway NESTing from previous RegCM run.
\end{enumerate}

We will for now for our test run download just the EIN15 dataset for the
year 1990 (Jan 01 00:00:00 UTC to Dec 31 18:00:00 UTC)

\begin{Verbatim}
$> cd $REGCM_GLOBEDAT
$> cd EIN15
$> mkdir 1990
$> cd 1990
$> ICTPSITE="clima-dods.ictp.it/data/d9/ERAIN150/1990/"
$> for type in "air hgt rhum uwnd vwnd"
>  do
>    for hh in "00 06 12 18"
>    do
>      curl -o ${type}.1990.${hh}.nc \
>         http://$ICTPSITE}/${type}.1990.${hh}.nc
>    done
>  done
\end{Verbatim}

With this dataset we are now ready to go through the RegCM Little Tutorial
in the next chapter of this User Guide.
