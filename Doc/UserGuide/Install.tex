%
% This file is part of ICTP RegCM model.
% Copyright (C) 2011 ICTP Trieste
% See the file COPYING for copying conditions.
%

\section{Software requirements}

In order to configure and install the RegCM code, the following software are
needed:

\begin{enumerate}
\item Python 2 language interpreter
\item GNU Make program
\item Fortran 90 compiler
\item NetCDF Format I/O library compiled with the above compiler.
   Source code can be found from \\
{\bf ftp://ftp.unidata.ucar.edu/pub/netcdf/netcdf.tar.gz} \\
Note that current version 4.1.2 is dependent on HDF5 1.8.6.
\end{enumerate}

Optional requirements strongly suggested are :

\begin{enumerate}
\item GNU \verb=patch= program if CLM option is activated. See below.
\item MPI2 Message Passing Library compiled with the above fortran compiler
for parallel runs using multiple core single machines or cluster of machines.
Source code for the implementation code was tested with can be obtained at: \\
{\bf http://www.open-mpi.org/software/ompi/v1.4/downloads}
\item HDF5 Format I/O Library compiled with the above fortran compiler to
enable NetCDF V4 features. Source code can be obtained at: \\
{\bf http://www.hdfgroup.org/ftp/HDF5/current/src}
\item NCO NetCDF Operators for manging NetCDF file. Most Linux distribution
have this already packed, and you should refer to your System Administrator or
OS Software Installation manual to obtain it. Source code is at: \\
{\bf http://nco.sourceforge.net/src}
\item CDO Climatic data Operators for managing NetCDF file. Most Linux
distribution have this already packed, and you should refer to your System
Administrator or OS Software Installation manual to obtain it.
Source code is at: \\
{\bf https://code.zmaw.de/projects/cdo/files}
\item A Scientific Plotting and Data Analysis Software such as:
\begin{itemize}
\item IGES GrADS 2.0 Graphical Analysis and Display System. Convenient helpers
are packed in RegCM to use GrADS with RegCM NetCDF output files.
Binaries and source code can be obtained from \\
{\bf http://www.iges.org/grads/downloads.html}
\item NCL, NCAR CISL Command Language. The NCL can read NetCDF output files, and
sample scripts can be found in the {\em Tools/Scripts/NCL} directory.
Binaries and source code can be obtained from \\
{\bf http://www.ncl.ucar.edu}
\end{itemize}
\item A quick viewer for NetCDF files like NcView: \\
{\bf http://meteora.ucsd.edu/~pierce/ncview\_home\_page.html}
\end{enumerate}

An example session of installation of basic software needed to compile the
RegCM model is detailed in chapter $[$\ref{Appendice}$]$.

\section{Configuring build}

The RegCM Version 4.1 is configured by a python2 script, which will select
and edit for you sample configuration files for the supported architectures.
These files are kept in the \verb=Arch= directory under \verb=$REGCM_ROOT=.

Currently tested and supported configurations (OS/Compiler) are:

\begin{enumerate}
\item Linux with GNU gfortran compiler version $\ge 4.4$
\item Linux with Intel ifort compiler version $\ge 10.0$
\item Linux with Portland pgf90 compiler version $\ge 9.3$
\item Mac OsX with g95 compiler
\item IBM AIX with xlf compiler
\item Oracle Solaris with Oracle Solaris Studio compiler $\ge 8.3$
\end{enumerate}

First step is change working directory to \verb=$REGCM_ROOT= and run the
\verb=configure= script:

\begin{Verbatim}
$> cd $REGCM_ROOT
$> ./configure
        #######################################
        # Welcome to the RegCM configuration! #
        #######################################

  This script will help you to setup the RegCM distribution
  with some simple questions. Default options [between square brackets] 
  will be assumed if you don't select valid options.
\end{Verbatim}

The script interactively asks you a series of questions, divided into three
distinct groups:
\begin{itemize}
\item General path and software requirements
\item Model configuration
\item Choice of Fortran compiler
\end{itemize}

\subsection{General path and software requirements}
\begin{enumerate}
\item The location of \verb=$REGCM_ROOT=.
\begin{Verbatim}
  ****  Please enter the path to your RegCM distribution: 
  [ /home/regcm/RegCM4.1 ]
  regcm_root =
\end{Verbatim}
This because you may be running the script from a different directory.
You probably are not, so just press enter.
\item Built binaries destination directory.
\begin{Verbatim}
  ****  Please enter the path where the RegCM binaries will be stored: 
  [ /home/regcm/RegCM4.1/Bin ]
\end{Verbatim}
If you are the only one using the model on the machine, you may safely
accept the default choice pressing enter.
\item NetCDF installation path. The script tries to detect the NetCDF
installation automatically using the \verb=nc-config= program, searching
standard directories and using the \verb=NETCDF= environment variable.
If successfull, the message
\begin{Verbatim}
NetCDF library found...
\end{Verbatim}
should appear. Otherwise you will be prompted the following question:
\begin{Verbatim}
   Unable to find a working NetCDF. Please input a valid path...
   For NetCDF library =
\end{Verbatim}
The answer should be the directory where on your system the file
\verb=libnetcdf.a= is stored. Next the file will ask:
\begin{Verbatim}
   For NetCDF include files =
\end{Verbatim}
The answer should be the directory where on your system the file
\verb=netcdf.mod= is stored.
\item OPTIONAL - If version 4 of NetCDF library is detected, but the
script is unable to use the \verb=nc-config= program, and autodetection
using standard paths or the HDF5 environment variable fails, you may be
prompted also for location of HDF5 library
\begin{Verbatim}
   Unable to find a working HDF5. Please input a valid path...
   For HDF5 library =
\end{Verbatim}
The answer should be the directory where on your system the file
\verb=libhdf5.a= is stored.
\item OPTIONAL - If version 4 of NetCDF library is detected, the script
will ask if to enable NetCDF file compression filter.
\begin{Verbatim}
  ****  Would you like to enable NetCDF v4 compression? (0/1)
  [0]
  zlib =
\end{Verbatim}
As the compression causes a run time penalty, it is not enabled by
default. If you are concerned by disk space, this option may help
significantly reduce storage usage by model output, expecially if
chemistry option is enabled. To enable it enter \verb=1=, otherwise
just accept default option pressing enter.
\end{enumerate}

\subsection{Model configuration}

The configure script goes to second phase.

\begin{Verbatim}
  !!!!  The following options are all documented in detail 
        on the eForge wiki page and in the RegCM Users' Guide,
        so please check there for more information.
\end{Verbatim}

\begin{enumerate}
\item Enable debug
\begin{Verbatim}
  ****  Do you want a debug - 0 or a production - 1  binary?
        [1]
        dbg =
\end{Verbatim}
If enabled, the model will be compiled using debug flags for the compiler,
which will allow the use of a debugger such as gdb. More diagnostic will
also be generated during model run. To enable debug mode, enter \verb=0=.
The default is to build production binaries with all optimization flags
turned on.
\item Parallel code using MPI library
\begin{Verbatim}
  ****  Do you want a serial - 0 or MPI-parallel - 1 binary?
        [1]
        mpi =
\end{Verbatim}
The model is coded to use an MPI2 library to run in parallel mode using
multiple cores/processors or run on a cluster. To enable this option you
will need to have installed on your system this library. Note that even
if built using MPI, the model can be still executed in serial mode using
at run time just one processor. The penalty time in this case is less than
$1\%$ of the run time of a serial built executable. The RegCM team
strongly suggest to build MPI enabled model, as the serial option will be
dropped in future releases. To enable serial build, enter \verb=0=, otherwise
accept the default choice pressing enter. \footnote{The serial version of the
model is deprecated, and support for this non MPI version will be dropped in
future model releases.}
\item BAND option
\begin{Verbatim}
  ****  Do you want to run the tropical band version, BAND - 1 or not - 0?
        [0]
        BAND =
\end{Verbatim}
This option build a special version of the model capable of running an
experiment with a spatial domain configured as a full circular equatorial
band around earth. This is documented in the \cite{refman_11}.
The default is to not enable this feature, i.e. to run the model on a limited
area not going round the whole earth. If you are planning to run such a
simulation, enter \verb=1=, otherwise accept the default choice pressing enter.
For the scope of the tutorial test run in chapter \ref{tutorial}, use the
default option.
\item CLM option
\begin{Verbatim}
  ****  Do you want to enable CLM  - 1 or not - 0? 
        [0]
        CLM =
\end{Verbatim}
This option switch off the default Land model of RegCM (derived from BATS1e),
and enables the use of the Community Land Model V3.5 inside RegCM. The default
is to use the RegCM BATS Land Model. If you are planning to run a simulation
using CLM, enter \verb=1=, otherwise accept the default choice pressing enter.
\footnote{The CLM option needs the GNU patch program to be compiled.}
For the scope of the tutorial test run in chapter \ref{tutorial}, use the
default option.
\end{enumerate}

\subsection{Choice of Fortran compiler}

The \verb=configure= script now prompts the user for selecting a fortran
compiler. Depending on your OS/compiler combination, select one in the list
of supported, or ask for a generic configuration.

\begin{Verbatim}
  ****  The following compiler/architecture combinations 
        are tested and known to work, please choose one:

              1. GNU Fortran v. 4.4 (Linux x86-64)
              2. Intel Fortran v. 10 or 11 (Linux x86-64)
              3. PGI Fortran v. 9 (Linux x86-64)
              4. IBM Xlf Compiler (generic AIX)
              5. IBM Xlf Compiler (Cineca SP6)
              6. Sun Ceres Fortran 95 v 8.3 r2008/01/28 (Linux x86-64)
              7. GNU+g95 Fortran compiler (Linux x86-64)
              8. Other

        compiler =
\end{Verbatim}

If your OS/compiler is not listed, enter \verb=8= for a generic system.
You are required afterwards to edit the \verb=Makefile.inc= file in the
\verb=$REGCM_ROOT= directory to configure you build.
Otherwise, select one in the list. The \verb=configure= script should now
create a \verb=Makefile.inc= file inside \verb=$REGCM_ROOT= to be used
to compile the RegCM model. You may now want to doublecheck result editing
the \verb=Makefile.inc= file, or accept what we have arranged for you.

\section{Build the model executables}

Now that everithing is hopefully configured, you may use the \verb=make=
program to build executables.
You have multiple target for the build, but at this stage you may want to

\begin{Verbatim}
$> make all
\end{Verbatim}

This target will builds all model parts. Other allowed targets are:

\begin{enumerate}
\item terrain : Just build the terrain program
\item icbc : Just build the icbc programs
\item clm2rcm : If CLM option selected, just build the clm2rcm program
\item regcm : Build the model executable in Main directory
\item postproc : Build postprocessing programs in PostProc
\item clean : clean up the build directory for objects
\end{enumerate}

The compilation is started in the whole model tree (PreProc, Main and PostProc).
Lot of messages will appear on screen, and if everithing goes fine at the end
you will see this message:

\begin{Verbatim}
##############################################################
YOU HAVE DONE IT! YOU HAVE COMPILED THE MODEL!
LET'S START PLAYING WITH THE BEAST....                        
##############################################################
\end{Verbatim}

Congratulations! You can now go to next step and run a test simulation.
