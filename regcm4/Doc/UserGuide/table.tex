%%
%%   This file is part of ICTP RegCM.
%%
%%   ICTP RegCM is free software: you can redistribute it and/or modify
%%   it under the terms of the GNU General Public License as published by
%%   the Free Software Foundation, either version 3 of the License, or
%%   (at your option) any later version.
%%
%%   ICTP RegCM is distributed in the hope that it will be useful,
%%   but WITHOUT ANY WARRANTY; without even the implied warranty of
%%   MERCHANTABILITY or FITNESS FOR A PARTICULAR PURPOSE.  See the
%%   GNU General Public License for more details.
%%
%%   You should have received a copy of the GNU General Public License
%%   along with ICTP RegCM.  If not, see <http://www.gnu.org/licenses/>.
%%
\begin{table}[h]
\begin{center}
\caption{Land Cover/Vegetation classes} \label{VegTypes}
\begin{tabular}{rl}
\hline\hline
1.&Crop/mixed farming\\
2.&Short grass\\
3.&Evergreen needleleaf tree\\
4.&Deciduous needleleaf tree\\
5.&Deciduous broadleaf tree\\
6.&Evergreen broadleaf tree\\
7.&Tall grass\\
8.&Desert\\
9.&Tundra\\
10.&Irrigated Crop\\
11.&Semi-desert\\
12.&Ice cap/glacier\\
13.&Bog or marsh\\
14.&Inland water\\
15.&Ocean\\
16.&Evergreen shrub\\
17.&Deciduous shrub\\
18.&Mixed Woodland\\
19.&Forest/Field mosaic \\
20.&Water and Land mixture \\
\hline\hline
\end{tabular}
\end{center}
\end{table}



\begin{landscape}
\begin{center}
\begin{table}
\scriptsize{
\caption{BATS vegetation/land-cover}  \label{landuse}
\hspace*{-0.25cm}
\vspace{-2.0cm}
\begin{tabular}{lcccccccccccccccccccc} \hline \hline
\multicolumn{1}{c}{Parameter}&\multicolumn{18}{c}{Land Cover/Vegetation Type}\\
&1&2&3&4&5&6&7&8&9&10&11&12&13&14&15&16&17&18&19&20 \\ \hline
Max fractional \\
vegetation cover&0.85&0.80&0.80&0.80&0.80&0.90&0.80&0.00&0.60&0.80&0.35&0.00&0.80&0.00&0.00&0.80&0.80&0.80&0.80&0.80 \\
Difference between max\\
fractional vegetation \\
cover and cover at 269 K&0.6&0.1&0.1&0.3&0.5&0.3&0.0&0.2&0.6&0.1&0.0&0.4&0.0&0.0&0.2&0.3&0.2&0.4&0.4 \\
Roughness length (m)       &0.08&0.05&1.00&1.00&0.80&2.00&0.10&0.05&0.04&0.06&0.10&0.01&0.03&0.0004&0.0004&0.10&0.10&0.80&0.3&0.3 \\
Displacement height (m)    &0.0&0.0&9.0&9.0&0.0&18.0&0.0&0.0&0.0&0.0&0.0&0.0&0.0&0.0&0.0&0.0&0.0&0.0&0.0&0.0 \\
Min stomatal \\
resistence (s/m)  &45 &60 &80 &80 &120 &60 &60 &200 &80 &45 &150 &200 &45 &200 &200 &80 &120 &100&120&120  \\
Max Leaf Area Index            &6   &2   &6   &6   &6   &6   &6   &0   &6   &6   &6   &0   &6   &0   &0   &6   &6   &6 &6 &6    \\
Min Leaf Area Index            &0.5 &0.5 &5   &1   &1   &5   &0.5 &0   &0.5 &0.5 &0.5 &0   &0.5 &0   &0   &5   &1   &3  &0.5 &0.5   \\
Stem (\& dead matter) \\
area index)&0.5&4.0&2.0&2.0&2.0&2.0&2.0&0.5&0.5&2.0&2.0&2.0&2.0&2.0&2.0&2.0&2.0&2.0&2.0&2.0 \\
Inverse square root of \\
leaf dimension (m$^{-1/2}$)&10&5&5&5&5&5&5&5&5&5&5&5&5&5&5&5&5&5&5&5\\
Light sensitivity \\
factor (m$^2$ W$^{-1}$)&0.02&0.02&0.06&0.06&0.06&0.06&0.02&0.02&0.02&0.02&0.02&0.02&0.02&0.02&0.02&0.02&0.02&0.06&0.02&0.02 \\ 
Upper soil layer \\
depth (mm)     &100 &100 &100 &100 &100 &100 &100 &100 &100 &100 &100 &100 &100 &100 &100 &100 &100 &100 &100 &100  \\
Root zone soil\\
layer depth (mm) &1000 &1000 &1500 &1500 &2000 &1500 &1000 &1000 &1000 &1000 &1000 &1000 &1000 &1000 &1000 &1000 &1000 &2000 &2000 &2000  \\
Depth of total\\
soil (mm) &3000 &3000 &3000 &3000 &3000 &3000 &3000 &3000 &3000 &3000 &3000 &3000 &3000 &3000 &3000 &3000 &3000 &3000 &3000 &3000  \\
Soil texture type    &6   &6   &6   &6   &7   &8   &6   &3   &6   &6   &5   &12   &6   &6   &6   &6   &5   &6 &6 &0    \\
Soil color type    &5   &3   &4   &4   &4   &4   &4   &1   &3   &3   &2   &1   &5   &5   &5   &4   &3   &4 &4 &0    \\
Vegetation albedo for \\
wavelengths $<$ 0.7 $\mu$ m &0.10&0.10&0.05&0.05&0.08&0.04&0.08&0.20&0.10&0.08&0.17&0.80&0.06&0.07&0.07&0.05&0.08&0.06 &0.06 &0.06 \\
Vegetation albedo for \\
wavelengths $>$ 0.7 $\mu$ m &0.30&0.30&0.23&0.23&0.28&0.20&0.30&0.40&0.30&0.28&0.34&0.60&0.18&0.20&0.20&0.23&0.28&0.24&0.18&0.18 \\  \hline \hline
\end{tabular}
}
\end{table}
\end{center}
\end{landscape}


\begin{table}[h]
\begin{center}
\caption{List of variables defined in {\it domain.param} file.}  \label{domain.param_file}
\vspace{0.25cm}
\begin{tabular}{|l|l|} \hline \hline
{\small {\bf Parameter}}   &   {\small {\bf Description}} \\ \hline \hline
{\footnotesize {\bf iproj}}    & {\footnotesize map projection} \\ 
 &  \vspace{-0.15 cm} \hspace{0.5 cm} {\footnotesize 'LAMCON' = Lambert Conformal } \\
 &  \vspace{-0.15 cm} \hspace{0.5 cm} {\footnotesize 'POLSTR' = Polar Stereographic } \\ 
 &  \vspace{-0.15 cm} \hspace{0.5 cm} {\footnotesize 'NORMER' = Normal Mercator} \\
 &  \hspace{0.5 cm} {\footnotesize 'ROTMER' = Rotated Mercator} \\ \hline
{\footnotesize {\bf iy}}   &   {\footnotesize number of grid points in y direction (i)} \\ \hline
{\footnotesize {\bf jx}}   &   {\footnotesize number of grid points in x direction (j)} \\ \hline
{\footnotesize {\bf kz}}   &   {\footnotesize number of vertical levels (k)} \\ \hline
{\footnotesize {\bf nsg}}  &   {\footnotesize number of subgrids in one direction} \\ \hline
{\footnotesize {\bf ds}}   &   {\footnotesize grid point separation in km} \\ \hline
{\footnotesize {\bf ptop}} &   {\footnotesize pressure of model top in cb} \\ \hline
{\footnotesize {\bf clat}} &   {\footnotesize central latitude of model domain in degrees} \\  \hline
{\footnotesize {\bf clon}} &   {\footnotesize central longitude of model domain in degrees} \\  \hline
{\footnotesize {\bf plat}} &   {\footnotesize pole latitude (only for rotated mercator projection)} \\  \hline
{\footnotesize {\bf plon}} &   {\footnotesize pole longitude (only for rotated mercator projection)} \\  \hline
{\footnotesize {\bf truelatL}} &   {\footnotesize Lambert true latitude (low  latitude side)} \\  \hline
{\footnotesize {\bf truelon}} &   {\footnotesize Lambert true latitude (high latitude side)} \\  \hline
{\footnotesize {\bf ntypec}} & {\footnotesize resolution of the global terrain and land-use data} \\ \hline
{\footnotesize {\bf ntypec\_s}} & {\footnotesize same as ntypec, except for subgrid} \\ \hline
 &  \vspace{-0.15 cm} \hspace{0.5 cm} {\footnotesize 60 = 1 degree \hspace{1.25cm} 5 = 5 minute} \\ \hline
 &  \vspace{-0.15 cm} \hspace{0.5 cm} {\footnotesize 30 = 30 minute \hspace{1.1cm} 3 = 3 minute} \\ \hline
 &  \hspace{0.5 cm} {\footnotesize 10 = 10 minute \hspace{1cm} 2 = 2 minute } \\ \hline
{\footnotesize {\bf h2opct}}  & {\footnotesize if water percentage $<$ h2opct, then land else water} \\ \hline
{\footnotesize {\bf ifanal}}  & {\footnotesize  true=perform cressman-type objective analysis} \\  
 & {\footnotesize  false=perform 16-point overlapping parabolic interpolation} \\ \hline
{\footnotesize {\bf smthbdy}}  & {\footnotesize true=extra smoothing in boundaries} \\ \hline
{\footnotesize {\bf lakadj}}   & {\footnotesize true=adjust lake levels according to obs} \\ \hline
{\footnotesize {\bf igrads}}   & {\footnotesize true=output GrADS control file} \\ \hline
{\footnotesize {\bf ibigend}}  & {\footnotesize 1 = big-endian (always 1)} \\ \hline
{\footnotesize {\bf ibyte}}    & {\footnotesize for direct access open statements (1 or 4)} \\ \hline

{\footnotesize {\bf FUDGE\_LND}}   & {\footnotesize land use fudge, true or false} \\ \hline
{\footnotesize {\bf FUDGE\_TEX}}   & {\footnotesize texture fudge, true or false} \\ \hline
{\footnotesize {\bf FUDGE\_LND\_s}} & {\footnotesize land use fudge for subgrid, true or false} \\ \hline
{\footnotesize {\bf FUDGE\_TEX}}   & {\footnotesize texture fudge for subgrid, true or false} \\ \hline
{\footnotesize {\bf filout}}   & {\footnotesize terrain output filename including path} \\ \hline
{\footnotesize {\bf filctl}}   & {\footnotesize GrADS control filename for output including path} \\ \hline
{\footnotesize {\bf IDATE1}}   & {\footnotesize beginning date of simulation (YYYYMMDDHH)} \\ \hline
{\footnotesize {\bf IDATE2}}   & {\footnotesize ending date of simulation (YYYYMMDDHH)} \\ \hline
{\footnotesize {\bf DATTYP}}   & {\footnotesize global analysis dataset} \\ 
 &  \vspace{-0.15 cm} \hspace{0.5 cm} {\footnotesize 'ECMWF'} \hspace{0.5 cm} {\footnotesize 'ERA40'}\\
 &  \vspace{-0.15 cm} \hspace{0.5 cm} {\footnotesize 'ERAHI'} \hspace{0.5 cm} {\footnotesize 'NNRP1'}\\
 &  \vspace{-0.15 cm} \hspace{0.5 cm} {\footnotesize 'NNRP2'} \hspace{0.5 cm} {\footnotesize 'NRP2W'}\\
 &  \vspace{-0.15 cm} \hspace{0.5 cm} {\footnotesize 'FVGCM'} \hspace{0.5 cm} {\footnotesize 'FNEST'}\\
 &  \vspace{-0.15 cm} \hspace{0.5 cm} {\footnotesize 'EH50M'}\\
\hline
{\footnotesize {\bf SSTTYP}}  & {\footnotesize SST dataset } \\ 
 &  \vspace{-0.15 cm} \hspace{0.5 cm} {\footnotesize 'OISST'} \hspace{0.5 cm} {\footnotesize 'OI\_NC'}\\
 &  \vspace{-0.15 cm} \hspace{0.5 cm} {\footnotesize 'GISST'} \hspace{0.5 cm} {\footnotesize 'OI\_WK'}\\ 
 &  \vspace{-0.15 cm} \hspace{0.5 cm} {\footnotesize for FVGCM:}\\
 &  \vspace{-0.15 cm} \hspace{0.5 cm} {\footnotesize 'FV\_RF'}  \hspace{0.5 cm} {\footnotesize 'FV\_A2'}\\
 &  \vspace{-0.15 cm} \hspace{0.5 cm} {\footnotesize 'EH5RF'}  \hspace{0.5 cm} {\footnotesize 'EH5A2'}\\
{\footnotesize {\bf LSMTYP}} & {\footnotesize LANDUSE legend, 'BATS' or 'USGS'} \\ \hline
{\footnotesize {\bf AERTYP}} & {\footnotesize AEROSOL datasets:}\\
 &  \vspace{-0.15 cm} \hspace{0.5 cm} {\footnotesize 'AER00D0'} \hspace{0.5 cm} {\footnotesize 'Neither aerosol, nor dust used'}\\
 &  \vspace{-0.15 cm} \hspace{0.5 cm} {\footnotesize 'AER01D0'} \hspace{0.5 cm} {\footnotesize 'Biomass, SO2 + BC + OC, no dust'}\\ 
 &  \vspace{-0.15 cm} \hspace{0.5 cm} {\footnotesize 'AER10D0'} \hspace{0.5 cm} {\footnotesize 'Anthropogenic, SO2 + BC + OC, no dust'}\\
 &  \vspace{-0.15 cm} \hspace{0.5 cm} {\footnotesize 'AER11D0'} \hspace{0.5 cm} {\footnotesize 'Anthropogenic+Biomass, SO2 + BC + OC, no dust'}\\
 &  \vspace{-0.15 cm} \hspace{0.5 cm} {\footnotesize 'AER00D1'} \hspace{0.5 cm} {\footnotesize 'No aerosol, with dust'}\\
&  \vspace{-0.15 cm} \hspace{0.5 cm} {\footnotesize 'AER01D1'}  \hspace{0.5 cm} {\footnotesize 'Biomass, SO2 + BC + OC, with dust'}\\
&  \vspace{-0.15 cm} \hspace{0.5 cm} {\footnotesize 'AER10D1'}  \hspace{0.5 cm} {\footnotesize 'Anthropogenic, SO2 + BC + OC, with dust'}\\
&  \vspace{-0.15 cm} \hspace{0.5 cm} {\footnotesize 'AER11D1'}  \hspace{0.5 cm} {\footnotesize 'Anthropogenic+Biomass, SO2 + BC + OC, with dust.'}\\
\hline
{\footnotesize {\bf ntex}}   & {\footnotesize Number of SOIL TEXTURE categories, 17} \\ \hline
{\footnotesize {\bf NPROC}}   & {\footnotesize NNumber of CPU used for parallel run.} \\ \hline
\hline
\end{tabular}
\end{center}
\end{table}

\begin{table}[h]
\begin{center}
\caption{List of output variables from Terrain (DOMAIN)}  \label{ter_var}
\vspace{0.25cm}
\begin{tabular}{|l|c|l|} \hline \hline
{\small {\bf Variables}} & {\small {\bf Description}} \\ \hline \hline
{\ {\bf ht}}    & {\ {Surface elevation (m)} }      \\ \hline
{\ {\bf htsd}}    & {\ {Surface elevation standard deviation} }    \\ \hline
{\ {\bf landuse}}    & {\ {Surface landuse type}}       \\ \hline
{\ {\bf xlat}}    & {\ {Latitude of cross points} }      \\ \hline
{\ {\bf xlon}}    & {\ {Longitude of cross points}}   \\ \hline
{\ {\bf dlat}}    & {\ {Latitude of dot points}}       \\ \hline
{\ {\bf dlon}}    & {\ {Longitude of dot points} }      \\ \hline
{\ {\bf xmap}}    & {\ {Map factors of cross points} }     \\ \hline
{\ {\bf dmap}}    & {\ {Map factors of dot points}}       \\ \hline
{\ {\bf coriol}}    & {\ {Coriolis force} }      \\ \hline
{\ {\bf snowam}}    & {\ {Initial snow amount} }     \\ \hline
\end{tabular}
\end{center}
\end{table}
