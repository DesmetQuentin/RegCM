%%
%%   This file is part of ICTP RegCM.
%%
%%   ICTP RegCM is free software: you can redistribute it and/or modify
%%   it under the terms of the GNU General Public License as published by
%%   the Free Software Foundation, either version 3 of the License, or
%%   (at your option) any later version.
%%
%%   ICTP RegCM is distributed in the hope that it will be useful,
%%   but WITHOUT ANY WARRANTY; without even the implied warranty of
%%   MERCHANTABILITY or FITNESS FOR A PARTICULAR PURPOSE.  See the
%%   GNU General Public License for more details.
%%
%%   You should have received a copy of the GNU General Public License
%%   along with ICTP RegCM.  If not, see <http://www.gnu.org/licenses/>.
%%
\newpage
\section{Model Description}

\subsection{Dynamics}
The model dynamic equations and numerical discretization are described by
\cite{Grell_94}.

\noindent
{\bf Horizontal Momentum Equations}
\begin{eqnarray}
\nonumber
{\partial{p^{\ast}u} \over \partial{t}} = -m^2 \left( {\partial{p^{\ast}uu/m} 
\over \partial{x}} + {\partial{p^{\ast}vu/m} \over \partial{y}} \right) -
{\partial{p^{\ast}u \dot{\sigma}} \over \partial{\sigma}}  \hspace{6cm} \\ \nonumber
\\ 
- m p^{\ast} \left[ {RT_v \over ({p^{\ast} + p_t/\sigma)}} {\partial{p^{\ast}} \over 
\partial{x}} + {\partial{\phi} \over \partial{x}} \right] + 
fp^{\ast}v + F_Hu + F_Vu, \hspace{3cm} \\\nonumber 
\\ \nonumber
{\partial{p^{\ast}v} \over \partial{t}} = -m^2 \left( {\partial{p^{\ast}uv/m} 
\over \partial{x}} + {\partial{p^{\ast}vv/m} \over \partial{y}} \right) - 
{\partial{p^{\ast}v \dot{\sigma}} \over \partial{\sigma}}  \hspace{6cm} \\ \nonumber 
\\
- m p^{\ast} \left[ {RT_v \over ({p^{\ast} + p_t/\sigma)}} {\partial{p^{\ast}} \over 
\partial{y}} + {\partial{\phi} \over \partial{y}} \right] + 
fp^{\ast}u + F_Hv + F_Vv, \hspace{3cm} \\ \nonumber 
\end{eqnarray} 
\noindent
where $u$ and $v$ are the eastward and northward components of velocity, $T_v$ is 
virtual temperature, $\phi$ is geopotential height, $f$ is the coriolis parameter, $R$
is the gas constant for dry air, $m$ is the map scale factor for either the Polar
Stereographic, Lambert Conformal, or Mercator map projections, $\dot{\sigma} = 
{d\sigma \over dt}$, and $F_H$ and $F_V$ represent the effects of horizontal and 
vertical diffusion, and $p^{\ast}=p_s-p_t$.
\\

\noindent 
{\bf Continuity and Sigmadot} $({\bf \dot{\sigma}})$ {\bf  Equations}
\begin{eqnarray} \label{eq:continuity}
{{\partial{p^{\ast}} \over \partial{t}}} = -m^2 \left( {\partial{p^{\ast}u/m} 
\over \partial{x}} + {\partial{p^{\ast}v/m} \over \partial{y}} \right) -
{\partial{p^{\ast}\dot{\sigma}} \over \partial{\sigma}}.
\end{eqnarray} \\
\noindent
The vertical integral of Equation~\ref{eq:continuity} is used to compute the 
temporal variation of the surface pressure in the model,

\begin{eqnarray}
{{\partial{p^{\ast}} \over \partial{t}}} = -m^2 \int_{0}^{1} \left( {\partial{p^{\ast}u/m} 
\over \partial{x}} + {\partial{p^{\ast}v/m} \over \partial{y}} \right) d\sigma.
\end{eqnarray} \\
\noindent
After calculation of the surface-pressure tendency ${\partial{p^{\ast}} \over 
\partial{t}}$, the vertical velocity in sigma coordinates $(\dot{\sigma})$ is 
computed at each level in the model from the vertical integral of 
Equation~\ref{eq:continuity}.
\begin{eqnarray}
\dot{\sigma} = - {1 \over p^{\ast}} \int_{0}^{\sigma} \left[ 
{\partial{p^{\ast}} \over \partial{t}} + m^2 \left( {\partial{p^{\ast}u/m} 
\over \partial{x}} + {\partial{p^{\ast}v/m} \over \partial{y}} \right) 
\right] d\sigma\prime,
\end{eqnarray} \\
\noindent
where $\sigma\prime$ is a dummy variable of integration and $\dot{\sigma}(\sigma=0)=0$.
\\
\newpage
\noindent
{\bf Thermodynamic Equation and Equation for Omega} ${\bf (\omega)}$ \\
\noindent 
The thermodynamic equation is
\begin{eqnarray}
\nonumber
{{\partial{p^{\ast}T} \over \partial{t}}} = -m^2 \left( {\partial{p^{\ast}uT/m} 
\over \partial{x}} + {\partial{p^{\ast}vT/m} \over \partial{y}} \right) -
{\partial{p^{\ast}T\dot{\sigma}} \over \partial{\sigma}} +  \\ \nonumber
\\
{RT_v\omega \over c_{pm}(\sigma + P_t/p_{ast})} + {p^{\ast}Q \over c_{pm}} +
F_HT + F_VT,
\end{eqnarray} \\
\noindent
where $c_{pm}$ is the specific heat for moist air at constant pressure, $Q$ is 
the diabatic heating, $F_HT$ represents the effect of horizontal diffusion,
$F_VT$ represents the effect of vertical mixing and dry convective adjustment, and 
$\omega$ is

\begin{eqnarray}
\omega = p^{\ast} \dot{\sigma} + \sigma {dp^{\ast} \over dt},
\end{eqnarray}
\noindent \\
where,
\begin{eqnarray}
{dp^{\ast} \over dt} = {\partial{p^{\ast}} \over \partial{t}} + m 
\left( u {\partial{p^{\ast}} \over \partial{x}} + 
v {\partial{p^{\ast}} \over \partial{y}} \right).
\end{eqnarray} \\

\noindent
The expression for $c_{pm} = c_p(1+0.8q_v)$,
\noindent \\
where $c_p$ is the specific heat at constant pressure for dry air and $q_v$ is the
mixing ratio of water vapor.
\\

\noindent
{\bf Hydrostatic Equation} \\
\noindent
The hydrostatic equation is used to compute the geopotential heights from the 
virtual temperature $T_v$,

\begin{eqnarray}
{\partial{\phi} \over \partial l n (\sigma + p_t / p^{\ast})} = -RT_v
\left[ 1 + {q_c + q_r \over 1 + q_v} \right]^{-1},
\end{eqnarray}
\noindent \\
where $T_v = T(1 + 0.608q_v)$, $q_v$, $q_c$, and $q_r$ are the 
water vapor, cloud water or ice, and rain water or snow, 
mixing ratios.  
%\\

%\noindent
%{\bf Conservation equations for water vapor, cloud water, and rain water} \\
%\noindent
%To resolve precipitation processes, the model prognostically solves
%three conservation equations for water vapor, cloud water
%or ice, and rain water or snow, mixing ratios, respectively:
%\begin{eqnarray}
%{\partial{q_v} \over \partial{t}} & = & - {\bf V} \cdot \nabla q_v +
%P_{q_v} + D_{q_v} \\
%{\partial{q_c} \over \partial{t}} & = & - {\bf V} \cdot \nabla q_c +
%P_{q_c} + D_{q_c} \\
%{\partial{q_r} \over \partial{t}} & = & - {\bf V} \cdot \nabla q_r -
%{\partial{V_f \rho g q_r} \over \partial{\sigma}} + P_r + D_{q_r}
%\end{eqnarray}
%\noindent
%where the $P$ terms represent phase-change processes and the $D$ terms
%represent horizontal and vertical diffusion rates for each of the
%conserved properties, respectively.

\newpage
\subsection{Physics} \label{sec:physics}

\subsubsection{Radiation Scheme}
\noindent
RegCM3 uses the radiation scheme of the NCAR CCM3, which 
is described in \cite{Kiehl_96}.  Briefly, the solar 
component, which accounts for the effect of ${\rm O_3}$, ${\rm H_2O}$, 
${\rm CO_2}$, and ${\rm O_2}$, follows the $\delta$-Eddington 
approximation of \cite{Kiehl_96}.  It includes 18 spectral 
intervals from 0.2 to 5 $\mu {\rm m}$. The cloud scattering and 
absorption parameterization follow that of \cite{Slingo_89}, 
whereby the optical properties of the cloud droplets (extinction 
optical depth, single scattering albedo, and asymmetry parameter)
are expressed in terms of the cloud liquid water content and an 
effective droplet radius.
When cumulus clouds are formed, the gridpoint fractional 
cloud cover is such that the total cover for the column extending
from the model-computed cloud-base level to the cloud-top level 
(calculated assuming random overlap) is a function of horizontal
gridpoint spacing.  The thickness of the cloud layer is assumed
to be equal to that of the model layer, and a different cloud 
water content is specified for middle and low clouds.

\subsubsection{Land Surface Model}
The surface physics are performed using \ac{BATS1e} which is described in detail by \cite{Dickinson_93}. 
BATS is a surface package designed to describe the role
of vegetation and interactive soil moisture in modifying the 
surface-atmosphere exchanges of momentum, energy, and water vapor.
The model has a vegetation layer, a snow layer, a surface soil layer, 
10~cm thick, or root zone layer, 1-2~m thick, and a third deep soil
layer 3~m thick.  Prognostic equations are solved for the soil layer temperatures 
using a generalization of the force-restore method of \cite{Deardoff_78}.  The 
temperature of the canopy and canopy foilage is calculated diagnostically
via an energy balance formulation including sensible, radiative, and 
latent heat fluxes.

The soil hydrology calculations include predictive equations for the water
content of the soil layers.  These equations account for precipitation, 
snowmelt, canopy foiliage drip, evapotranspiration, surface runoff, 
infiltration below the root zone, and diffusive exchange of water between 
soil layers.  The soil water movement formulation is obtained from 
a fit to results from a high-resolution soil model \cite{Dickinson_84} and
the surface runoff rates are expressed as functions of the precipitation 
rates and the degree of soil water saturation.  Snow depth is prognostically
calculated from snowfall, snowmelt, and sublimation.  Precipitation is 
assumed to fall in the form of snow if the temperature of the lowest 
model level is below 271~K.

Sensible heat, water vapor, and momentum fluxes at the surface are 
calculated using a standard surface drag coefficient formulation 
based on surface-layer similarity theory.  The drag coefficient depends
on the surface roughness length and on the atmospheric stability 
in the surface layer.  The surface evapotranspiration rates depend 
on the availability of soil water.  \ac{BATS} has 20 vegetation types (Table~\ref{landuse};  soil
textures ranging from coarse (sand), to intermediate (loam), to fine
 (clay);  and different soil colors (light to dark) for the soil 
albedo calculations.  These are described in \cite{Dickinson_86}. 

In the latest release version, additional modifications have been made to \ac{BATS}
in order to account for the subgrid variability of topography and land
cover using a mosaic-type approach \citep{Giorgi03b}.  This
modification adopts a regular fine-scale surface subgrid for each
coarse model grid cell.  Meteorological variables are disaggregated
from the coarse grid to the fine grid based on the elevation
differences.  The \ac{BATS} calculations are then performed separately
for each subgrid cell, and surface fluxes are reaggregated onto the
coarse grid cell for input to the atmospheric model. This
parameterization showed a marked improvement in the representation of
the surface hydrological cycle in mountainous regions \citep{Giorgi03b}.  

\subsubsection{Planetary Boundary Layer Scheme}
\noindent
The planetary boundary layer
scheme, developed by \cite{Holtslag_90}, is based on a nonlocal
diffusion concept that takes into account countergradient fluxes 
resulting from large-scale eddies in an unstable, well-mixed atmosphere.
The vertical eddy flux within the PBL is given by \\
\begin{eqnarray}
F_c = -K_c \left( {\partial{C} \over \partial{z}} - \gamma_c \right)
\end{eqnarray} \\

\noindent
where $\gamma_c$ is a ``countergradient'' transport term describing 
nonlocal transport due to dry deep convection.  The eddy diffusivity 
is given by the nonlocal formulation \\
\begin{eqnarray}
K_c = kw_tz \left( 1- {z \over h}^2 \right),
\end{eqnarray}

\noindent
where $k$ is the von Karman constant; $w_t$ is a turbulent convective
velocity that depends on the friction velocity, height, and the 
Monin--Obhukov length;  and $h$ is the PBL height.  The 
countergradient term for temperature and water vapor is given by \\ 
\begin{eqnarray}  \label{eq:countergradient}
\gamma_c = C { {\phi_c}^0 \over w_t h},
\end{eqnarray}

\noindent
where C is a constant equal to 8.5, and ${\phi_c}^0$ is the surface
temperature or water vapor flux. Equation~\ref{eq:countergradient} is 
applied between the top of the PBL and the top of the surface layer, 
which is assumed to be equal to $0.1h$. Outside this region and for
momentum, $\gamma_c$ is assumed to be equal to 0.  

For the calculation of the eddy diffusivity and countergradient terms, the 
PBL height is diagnostically computed from \\
\begin{eqnarray}
h = { {\rm Ri_cr} [u(h)^2 + v(h)^2] \over 
(g/\theta_s)[\theta_v(h)-\theta_s] }
\end{eqnarray}

\noindent
where $u(h)$, $v(h)$, and $\theta_v$ are the wind components and the virtual
potential temperature at the PBL height, $g$ is gravity, ${\rm Ri_cr}$ is the 
critical bulk Richardson number, and $\theta_s$ is an appropriate 
temperature of are near the surface.  Refer to \cite{Holtslag_90} and 
\cite{Holtslag_93} for a more detailed description.


\subsubsection{Convective Precipitation Schemes}

Convective precipitation is computed using one of three schemes: (1) Modified-Kuo scheme \cite{Anthes_77};
(2) Grell scheme \cite{Grell_93}; 
and (3) MIT-Emanuel scheme \citep{Emanuel_91,Emanuel_99}. In addition, the Grell
parameterization is implemented using one of two closure assumptions:
(1) the Arakawa and Schubert closure \cite{Grell_94} and (2) 
the Fritsch and Chappell closure \cite{Fritsch_80}, 
hereafter refered to as AS74 and FC80, respectively.\\

\noindent
{\bf 1.  Kuo Scheme:}  Convective activity in the Kuo scheme is initiated when the moisture
convergence $M$ in a column exceeds a given threshold and the
vertical sounding is convectively unstable. A fraction of the
moisture convergence $\beta$ moistens the column and the rest is
converted into rainfall $P^{CU}$ according to the following relation:
\begin{eqnarray}
P^{CU}&=&M(1-\beta).
\label{eqn_model:KUOppt}
\end{eqnarray}
$\beta$ is a function of the average relative humidity $\overline{RH}$
of the sounding as follows:
\begin{eqnarray}
\beta&=&\left\{ \begin{array} {r@{\quad\quad}l} 2(1-\overline{RH}) & \overline{RH}\ge 0.5 \\ 1.0 & \mbox{otherwise} \end{array} \right.
\label{eqn_model:Bfact}
\end{eqnarray}
Note that the moisture convergence term includes only the advective
tendencies for water vapor. However, evapotranspiration from the
previous time step is indirectly included in $M$ since it tends to
moisten the lower atmosphere. Hence, as the evapotranspiration
increases, more and more of it is converted into rainfall assuming the
column is unstable. The latent heating resulting from condensation is
distributed between the cloud top and bottom by a function that
allocates the maximum heating to the upper portion of the cloud
layer. To eliminate numerical point storms, a horizontal diffusion
term and a time release constant are included so that the
redistributions of moisture and the latent heat release are not
performed instantaneously \citep{Giorgi_89b, Giorgi_91c}. \\

\noindent
{\bf 2.  Grell Scheme:}    The Grell scheme \cite{Grell_93}, similar to the AS74 parameterization,
considers clouds as two steady-state circulations: an updraft and a
downdraft. No direct mixing occurs between the cloudy air and the
environmental air except at the top and bottom of the
circulations. The mass flux is constant with height and no entrainment
or detrainment occurs along the cloud edges. The originating levels of
the updraft and downdraft are given by the levels of maximum and
minimum moist static energy, respectively. The Grell scheme is
activated when a lifted parcel attains moist convection. Condensation
in the updraft is calculated by lifting a saturated parcel. The
downdraft mass flux ($m_0$) depends on the updraft mass flux ($m_b$)
according to the following relation:
\begin{eqnarray}
m_0=\frac{\beta I_1}{I_2}m_b,
\end{eqnarray}
where $I_1$ is the normalized updraft condensation, $I_2$ is the
normalized downdraft evaporation, and $\beta$ is the fraction of
updraft condensation that re-evaporates in the downdraft. $\beta$
depends on the wind shear and typically varies between 0.3 and
0.5. Rainfall is given by
\begin{eqnarray}
P^{CU}&=&I_1m_b(1-\beta).
\label{eqn_model:GCCppt}
\end{eqnarray}
Heating and moistening in the Grell scheme are determined both
by the mass fluxes and the detrainment at the cloud top and
bottom. In addition, the cooling effect of moist downdrafts is
included.

Due to the simplistic nature of the Grell scheme, several closure
assumptions can be adopted. RegCM3's earlier version directly
implements the quasi-equilibrium assumption of AS74. It
assumes that convective clouds stabilize the environment as fast as
non-convective processes destabilize it as follows:
\begin{eqnarray}
m_b=\frac{ABE''-ABE}{NA\Delta t},
\label{eqn_model:closureAS}
\end{eqnarray}
where $ABE$ is the buoyant energy available for convection, $ABE''$ is
the amount of buoyant energy available for convection in addition to
the buoyant energy generated by some of the non-convective processes
during the time interval $\Delta t$, and $NA$ is the rate of change of
$ABE$ per unit $m_b$. The difference $ABE''-ABE$ can be thought of as
the rate of destabilization over time $\Delta t$. $ABE''$ is computed
from the current fields plus the future tendencies resulting from the
advection of heat and moisture and the dry adiabatic adjustment.

In the latest RegCM3 version, by default, we use a stability based closure assumption, the FC80 type closure assumption, that is commonly
implemented in GCMs and RCMs. In this closure, it is assumed that convection removes the
$ABE$ over a given time scale as follows:
\begin{eqnarray}
m_b=\frac{ABE}{NA \tau},
\label{eqn_model:closureFC}
\end{eqnarray}
where $\tau$ is the $ABE$ removal time scale.

The fundamental difference between the two assumptions is that the
AS74 closure assumption relates the convective fluxes
and rainfall to the tendencies in the state of the atmosphere, while
the FC80 closure assumption relates the convective fluxes
to the degree of instability in the atmosphere. Both schemes achieve a
statistical equilibrium between convection and the large-scale processes.
\\


\noindent
{\bf 3. MIT-Emanuel scheme:}
The newest cumulus convection option to the \ac{RegCM3} is the
\ac{MIT} scheme. More detailed descriptions can be found in \citet{Emanuel_91} and
\citet{Emanuel_99}.  The scheme assumes that the mixing in clouds is
highly episodic and inhomogeneous (as opposed to a continuous
entraining plume) and considers convective fluxes based on an
idealized model of sub-cloud-scale updrafts and downdrafts.
Convection is triggered when the level of neutral buoyancy is greater
than the cloud base level.  Between these two levels, air is lifted
and a fraction of the condensed moisture forms precipitation while the
remaining fraction forms the cloud.  The cloud is assumed to mix with
the air from the environment according to a uniform spectrum of
mixtures that ascend or descend to their respective levels of neutral
buoyancy.  The mixing entrainment and detrainment rates are functions
of the vertical gradients of buoyancy in clouds.  The fraction of the
total cloud base mass flux that mixes with its environment at each
level is proportional to the undiluted buoyancy rate of change with
altitude.  The cloud base upward mass flux is relaxed towards the
sub-cloud layer quasi equilibrium.

In addition to a more physical representation of convection, the
MIT-Emanuel scheme offers several advantages compared to the
other RegCM3 convection options.  For instance, it includes a
formulation of the auto-conversion of cloud water into precipitation
inside cumulus clouds, and ice processes are accounted for by allowing
the auto-conversion threshold water content to be temperature
dependent.  Additionally, the precipitation is added to a single,
hydrostatic, unsaturated downdraft that transports heat and water.
Lastly, the MIT-Emanuel scheme considers the transport of passive tracers.

\subsubsection{Large-Scale Precipitation Scheme}  
Subgrid Explicit Moisture Scheme (SUBEX) is used to handle 
nonconvective clouds and precipitation resolved by the model.
This is one of the new components of the model.  
SUBEX accounts for the subgrid variability in clouds by 
linking the average grid cell relative humidity
to the cloud fraction and cloud water following the work 
of \cite{Sundqvist_89}.  

The fraction of the grid cell covered by clouds, $FC$, is determined 
by,
\begin{eqnarray}
FC = \sqrt{ { {RH - RH_{min}} \over {RH_{max} - RH_{min} }} }
\end{eqnarray}

\noindent
where ${RH_{min}}$ is the relative humidity threshold at which 
clouds begin to form, and ${RH_{max}}$ is the relative humidity
where $FC$ reaches unity.  $FC$ is assumed to be zero when RH is 
less than ${RH_{min}}$ and unity when RH is greater than 
${RH_{max}}$.  

Precipitation $P$ forms when the cloud water content exceeds the 
autoconversion threshold ${Q^{th}}_c$ according to the following 
relation:
\begin{eqnarray}
P = C_{ppt}(Q_c/FC - {{Q_c}^{th}})FC
\end{eqnarray}

\noindent
where $1/C_{ppt}$ can be considered the characteristic time for
which cloud droplets are converted to raindrops.  The 
threshold is obtained by scaling the median cloud liquid 
water content equation according to the following:
\begin{eqnarray}
{Q^{th}}_c = C_{acs} 10^{-0.49 + 0.013T}, 
\end{eqnarray}

\noindent
where $T$ is temperature in degrees Celsius, and $C_{acs}$ is the 
autoconversion scale factor.  Precipitation is assumed to fall
instantaneously.

SUBEX also includes simple formulations for raindrop accretion 
and evaporation.  The formulation for the accretion of cloud 
droplets by falling rain droplets is based on the work of 
\cite{Beheng_94} and is as follows:
\begin{eqnarray}
P_{acc} = C_{acc} Q P_{sum}
\end{eqnarray}

\noindent
where $P_{acc}$ is the amount of accreted cloud water, $C_{acc}$ is 
the accretion rate coefficient, and $P_{sum}$ is the accumulated
precipitation from above falling through the cloud.  

Precipitation evaporation is based on the work of 
\cite{Sundqvist_89} and is as follows,
\begin{eqnarray}
P_{evap} = C_{evap} (1-RH) {P^{1/2}}_{sum}
\end{eqnarray}

\noindent
where $P_{evap}$ is the amount of evaporated precipitation, and 
$C_{evap}$ is the rate coefficient.  For a more detailed description 
of SUBEX and a list of the parameter values refer to \cite{Pal_00}. 



\subsubsection{Ocean flux Parameterization} 
\noindent
1.  BATS: {BATS} uses standard Monin-Obukhov similarity relations to compute the fluxes with no special treatment of convective and very stable conditions.  In addition, the roughness length is set to a constant, i.e. it is not a function of wind and stability.  

\noindent
2.  Zeng:  The Zeng scheme describes all stability conditions and includes a gustiness velocity to account for the additional flux induced by boundary layer scale variability. Sensible heat (${\rm SH}$), latent heat (${\rm LH}$), 
and momentum ($\tau$) fluxes between the sea surface and 
lower atmosphere are calculated using the following 
bulk aerodynamic algorithms, 
\begin{eqnarray}
\tau = \rho_a {u_{\ast}}^2 ({u_{x}}^2 + {u_{y}}^2)^{1/2} / u \\ \nonumber \\
{\rm SH} = -\rho_a C_{pa} u_{\ast} \theta_{\ast} \hspace{.6cm} \\ \nonumber \\
{\rm LH} =  -\rho_a L_{e} u_{\ast} q_{\ast}  \hspace{.65cm}
\end{eqnarray}

\noindent
where $u_x$ and $u_y$ are mean wind components, $u_{\ast}$ is the frictional wind velocity, 
$\theta_{\ast}$ is the temperature scaling parameter, $q_{\ast}$ is the specific humidity 
scaling parameter,  $\rho_a$ is air density, $C_{pa}$ is specific heat of air, and $L_{e}$ is 
the latent heat of vaporization.  For further details on the calculation of these
parameters refer to \cite{Zeng_98}.
 

\subsubsection{Pressure Gradient Scheme}
Two options are available for calculating the pressure gradient force.  The normal 
way uses the full fields.  The other way is the hydrostatic deduction scheme 
which makes use of a perturbation temperature.  In this scheme, extra smoothing
on the top is done in order to reduce errors related to the PGF calculation. 

\subsubsection{Lake Model}
The lake model developed by  \cite{Hostetler_93} can be interactively 
coupled to the atmospheric model.  In the lake model, fluxes of 
heat, moisture, and momentum are calculated based on meteorological 
inputs and the lake surface temperature and albedo.  Heat is transferred
vertically between lake model layers by eddy and convective mixing.
Ice and snow may cover part or all of the lake surface.

In the lake model, the prognostic equation for temperature is, 
\begin{eqnarray}
{\partial{T} \over \partial{t}} = (k_e + k_m) {\partial^2{T} \over \partial{z}^2 }
\end{eqnarray}

\noindent
where $T$ is the temperature of the lake layer, and $k_e$ and $k_m$ are the eddy 
and molecular diffusivities, respectively.   
The parameterization of \cite{Henderson-Sellers_86} is used to calculate $k_e$ and
$k_m$ is set to a constant value of $39 \times 10^{-7}~m^2~s^{-1}$ except under ice 
and at the deepest points in the lake.

Sensible and latent heat fluxes from the lake are calculated using  
the BATS parameterizations \cite{Dickinson_93}.  The bulk aerodynamic
formulations for latent heat flux ($F_q$) and sensible heat flux ($F_s$) are as 
follows, 
\begin{eqnarray}
F_q = \rho_a C_D V_a (q_s - q_a) \\
F_s = \rho_a C_p C_D V_a (T_s - T_a)
\end{eqnarray}

\noindent
where the subscripts $s$ and $a$ refer to surface and air, respectively;
$\rho_a$ is the density of air, $V_a$ is the wind speed, $C_p$, $q$ is 
specific humidity, and $T$ is temperature.  The momentum drag coefficient,
$C_D$, depends on roughness length and the surface bulk Richardson number.


Under ice-free conditions, the lake surface albedo is calculated as a function 
of solar zenith angle \cite{Henderson-Sellers_86}.  Longwave radiation emitted 
from the lake is calculated according to the Stefan-Boltzmann law. 
The lake model uses the partial ice cover scheme of \cite{Patterson_88} 
to represent the different heat and moisture exchanges between open water 
and ice surfaces and the atmosphere, and to calculate the surface energy 
of lake ice and overlying snow.  For further details refer to \cite{Hostetler_93} and 
\cite{Small_99b}. 

\subsubsection{Aerosols and Dust (Chemistry Model)} 
The representation of dust emission processes is a key element in a dust model and depends on the wind conditions, the soil characteristics and the particle size. Following Marticorena and Bergametti (1995) and Alfaro and Gomes (2001), here the dust emission calculation is based on parameterizations of soil aggregate saltation and sandblasting processes. The main steps in this calculation are: The specification of soil aggregate size distribution for each model grid cell, the calculation of a threshold friction velocity leading to erosion and saltation processes, the calculation of the horizontal saltating soil aggregate mass flux, and finally the calculation of the vertical transportable dust particle mass flux generated by the saltating aggregates. In relation to the BATS interface, these parameterizations become effective in the model for cells dominated by desert and semi desert land cover. 
