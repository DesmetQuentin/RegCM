\section{RegCM}
\begin{table}[h]
\begin{center}
\caption{List of restart, timestep, and output parameters defined in {\it regcm.in} file.}  \label{regcm.in_file}
\vspace{0.25cm}
\begin{tabular}{|l|l|} \hline \hline
{\small {\bf Restart parameters}} &   {\small {\bf Description}} \\ \hline \hline
{\footnotesize {\bf ifrest}} & {\footnotesize true or false for restart simulation} \\ \hline
\hspace{.3cm} {\footnotesize {\bf idate0}} & {\footnotesize start date of first simulation} \\ \hline
\hspace{.3cm} {\footnotesize {\bf idate1}} & {\footnotesize restart date} \\ \hline
\hspace{.3cm} {\footnotesize {\bf idate2}} & {\footnotesize end date of restart simulation} \\ \hline
{\footnotesize {\bf nslice}} & {\footnotesize number of days for next model run} \\ \hline \hline
{\small {\bf Timestep parameters}} &   {\small {\bf Description}} \\ \hline \hline
{\footnotesize {\bf radfrq}} & {\footnotesize time step for radiation model} \\ \hline
\hspace{.3cm} {\footnotesize {\bf abemh}}  & {\footnotesize time step for LW absorption/emissivity} \\ \hline
{\footnotesize {\bf abatm}}  & {\footnotesize time step for lsm} \\ \hline
{\footnotesize {\bf dt}}     & {\footnotesize time step for atmosphere model} \\ \hline
{\footnotesize {\bf ibdyfrq}} & {\footnotesize lateral boundary conditions frequency} \\ \hline \hline
{\small {\bf Output parameters}} &   {\small {\bf Description}} \\ \hline \hline
{\footnotesize {\bf ifsave}} & {\footnotesize save output for restart} \\ \hline
\hspace{.3cm} {\footnotesize {\bf savfrq}} & {\footnotesize time interval to save output for restart (hr)} \\ \hline
{\footnotesize {\bf iftape}} & {\footnotesize save atmospheric output} \\ \hline
\hspace{.3cm} {\footnotesize {\bf tapfrq}} & {\footnotesize time interval to save atmospheric output (hr)} \\ \hline
{\footnotesize {\bf ifrad}}  & {\footnotesize save radiation output } \\ \hline
\hspace{.3cm} {\footnotesize {\bf radisp}} & {\footnotesize time interval to save radiation output (hrs) } \\  \hline
{\footnotesize {\bf ifbat}}  & {\footnotesize save surface model output } \\  \hline
{\footnotesize {\bf ifsub}}  & {\footnotesize save sub-bats model output } \\  \hline
\hspace{.3cm} {\footnotesize {\bf batfrq}} & {\footnotesize time interval to save surface model output (hrs) }  \\ \hline
{\footnotesize {\bf ifprt}}  & {\footnotesize printer output} \\ \hline
\hspace{.3cm} {\footnotesize {\bf prtfrq}} & {\footnotesize time interval for printer output (hrs)} \\ \hline
\hspace{.3cm} {\footnotesize {\bf kxout}}  & {\footnotesize k level of horizontal slice for printer output} \\ \hline
\hspace{.3cm} {\footnotesize {\bf jxsex}}  & {\footnotesize j index of the north-south vertical slice for printer output} \\ \hline
{\footnotesize {\bf iotyp}}  & {\footnotesize Output format; 1=direct access, 2=sequential} \\ \hline
\hspace{.3cm} {\footnotesize {\bf ibintyp}}  & {\footnotesize 1=big\_endian, 2=little\_endian} \\ \hline
{\footnotesize {\bf ifchem}}  & {\footnotesize save tracer model output } \\ \hline
\hspace{.3cm} {\footnotesize {\bf chemfrq}} & {\footnotesize time interval to save tracer model output (hrs) } \\  \hline
\end{tabular}
\end{center}
\end{table}

\begin{table}[h]
\begin{center}
\caption{List of physic options in {\it regcm.in} file.}  \label{regcm_file2}
\vspace{0.25cm}
\begin{tabular}{|l|l|} \hline \hline
{\small {\bf Physics parameter}} & {\small {\bf Description}} \\ \hline \hline
{\footnotesize {\bf iboudy}}   & {\footnotesize lateral boundary conditions; 0=fixed, 1=relaxation (linear),} \\ 
  &  {\footnotesize 2=time dependent, 3=time and inflow/outflow dependent} \\
  &  {\footnotesize 4=sponge, 5=relaxation (exponential)} \\ \hline
{\footnotesize {\bf ibltyp}}   &  {\footnotesize planetary boundary layer scheme; 1=Holtslag} \\ \hline
{\footnotesize {\bf icup}}     & {\footnotesize cumulus scheme; 1=Anthes-Kuo, 2=Grell, 4=MIT-Emanuel}  \\ \hline
\hspace{.3cm} {\footnotesize {\bf igcc}}     & {\footnotesize Grell Scheme Convective Closure Scheme;} \\ 
  & {\footnotesize 1=Arakawa \& Schubert, 2=Fritsch \& Chappell} \\ \hline
{\footnotesize {\bf ipptls}}   & {\footnotesize Large-scale precipitation scheme; 1=SUBEX} \\ \hline
{\footnotesize {\bf iocnflx}}  & {\footnotesize ocean flux parameterization scheme;  1= BATS, 2=Zeng}\\ \hline
{\footnotesize {\bf ipgf}}     & {\footnotesize pressure gradient scheme; 0=normal way, 1= hydrostatic deduction} \\ \hline
{\footnotesize {\bf lakemod}}  & {\footnotesize Lake model;  0=no, 1=yes} \\ \hline
{\footnotesize {\bf ichem}}  & {\footnotesize Tracer/Chemistry  model;  0=no, 1=yes} \\ \hline
{\small {\bf Chemistry parameters}} & {\small {\bf Description}} \\ \hline \hline
{\footnotesize {\bf idirect}}   & {\footnotesize direct radiative effect of aerosols } \\ \hline
{\footnotesize {\bf chtrname}}   & {\footnotesize Chemistry tracer name} \\ \hline
{\footnotesize {\bf chtrsol}}   & {\footnotesize } \\ \hline
{\footnotesize {\bf chtrdpv}}   & {\footnotesize } \\ \hline
{\footnotesize {\bf dustbsiz}}   & {\footnotesize } \\ \hline
\end{tabular}
\end{center}
\end{table}


\begin{table}[!]
\begin{center}
\caption{List of output variables from atmosphere}  \label{atmos_var}
\vspace{0.25cm}
\begin{tabular}{|l|c|l|} \hline \hline
{\small {\bf Variables}} & {\small {\bf Description}} \\ \hline \hline
{\ {\bf u}}    & {\ {Eastward wind (${\rm m~s^{-1}}$) } }      \\ \hline
{\ {\bf v}}    & {\ {Northward wind (${\rm m~s^{-1}}$)} }     \\ \hline
{\ {\bf w}}    & {\ {Omega (hPa) p-velocity}}       \\ \hline
{\ {\bf t}}    & {\ {Temperature (K)}}       \\ \hline
{\ {\bf qv}}    & {\ {Water vaporMixing ratio (${\rm g~kg^{-1}}$)} }      \\ \hline
{\ {\bf qc}}    & {\ {Cloud water mixing ratio (${\rm g~kg^{-1}}$)} }     \\ \hline
{\ {\bf psa}}    & {\ {Surface pressure (Pa)}}       \\ \hline
{\ {\bf tpr}}    & {\ {Total precipitation (mm)} }      \\ \hline
{\ {\bf tgb}}    & {\ {Lower soil layer temp (K)} }     \\ \hline
{\ {\bf smt}}    & {\ {Total soil water (mm)}}       \\ \hline
{\ {\bf rno}}    & {\ {Base flow (${\rm mm~day^{-1}}$)}}       \\ \hline
\end{tabular}
\end{center}
\end{table}

\begin{table}[!]
\begin{center}
\caption{List of output variables from surface model}  \label{lsm_var}
\vspace{0.25cm}
\begin{tabular}{|l|c|l|} \hline \hline
{\small {\bf Variables}} & {\small {\bf Description}} \\ \hline \hline
{\ {\bf u10m}}    & {\ {Anemometer eastward wind (${\rm m~s^{-1}}$)} }      \\ \hline
{\ {\bf v10m}}    & {\ {Anemometer northward wind (${\rm m~s^{-1}}$)} }     \\ \hline
{\ {\bf uvdrag}}    & {\ {Surface drag stress}}       \\ \hline
{\ {\bf tgb}}    & {\ {Ground temperature (K)} }      \\ \hline
{\ {\bf tlef}}    & {\ {Foliage temperature (K)} }     \\ \hline
{\ {\bf t2m}}    & {\ {Anemometer temperature (K)}}       \\ \hline
{\ {\bf q2m}}    & {\ {Anemometer specific humidity ${\rm kg~kg^{-1}}$ } }      \\ \hline
{\ {\bf ssw}}    & {\ {Top layer soil moisture (mm)} }     \\ \hline
{\ {\bf rsw}}    & {\ {Root layer soil moisture (mm)}}       \\ \hline
{\ {\bf tpr}}    & {\ {Total precipitation (${\rm mm~day^{-1}}$)} }      \\ \hline
{\ {\bf evp}}    & {\ {Evapotranspiration (${\rm mm~day^{-1}}$)} }     \\ \hline
{\ {\bf runoff}}    & {\ {Surface runoff (${\rm mm~day^{-1}}$)}}       \\ \hline
{\ {\bf scv}}    & {\ {Snow water equivalent (mm)} }      \\ \hline
{\ {\bf sena}}    & {\ {Sensible heat (${\rm W~m^{-2}}$)} }     \\ \hline
{\ {\bf flw}}    & {\ {Net longwave (${\rm W~m^{-2}}$)}}       \\ \hline
{\ {\bf fsw}}    & {\ {Net solar absorbed (${\rm W~m^{-2}}$)} }      \\ \hline
{\ {\bf flwd}}   & {\ {Downward longwave (${\rm W~m^{-2}}$)} }     \\ \hline
{\ {\bf sina}}    & {\ {Solar incident (${\rm W~m^{-2}}$)}}       \\ \hline
{\ {\bf prcv}}    & {\ {Convective precipitation (${\rm mm~day^{-1}}$)} }      \\ \hline
{\ {\bf psb}}    & {\ {Surface pressure (Pa)} }     \\ \hline
{\ {\bf zpbl}}    & {\ {PBL height (m)}}       \\ \hline
{\ {\bf tgmax}}    & {\ {maximum ground temperature (K)}}       \\ \hline
{\ {\bf tgmin}}    & {\ {minimum ground temperature (K)}}      \\ \hline
{\ {\bf t2max}}    & {\ {maximum 2m temperature (K)}}     \\ \hline
{\ {\bf t2min}}    & {\ {minimum 2m temperature (K)}}       \\ \hline
{\ {\bf w10max}}    & {\ {maximum 10m wind speed ($ m~s^{-1}$)}}     \\ \hline
{\ {\bf psmin}}    & {\ {minimum surface pressure ($hPa$)}}       \\ \hline

\end{tabular}
\end{center}
\end{table}

\begin{table}[!]
\begin{center}
\caption{List of output variables from radiation model}  \label{rad_var}
\vspace{0.25cm}
\begin{tabular}{|l|c|l|} \hline \hline
{\small {\bf Variables}} & {\small {\bf Description}} \\ \hline \hline
{\ {\bf fc}}    & {\ {Cloud fraction (fraction)} }      \\ \hline
{\ {\bf clwp}}    & {\ {Cld liquid ${\rm H_2O}$ path (${\rm g~m^{-2}}$)} }    \\ \hline
{\ {\bf qrs}}    & {\ {Solar heating rate (${\rm K~s^{-1}}$)}}       \\ \hline
{\ {\bf qrl}}    & {\ {LW cooling rate (${\rm K~s^{-1}}$)} }      \\ \hline
{\ {\bf fsw}}    & {\ {Surface abs solar (${\rm W~m^{-2}}$)} }     \\ \hline
{\ {\bf flw}}    & {\ {LW cooling of surface (${\rm W~m^{-2}}$)}}       \\ \hline
{\ {\bf clrst}}    & {\ {Clear sky col abs sol (${\rm W~m^{-2}}$)} }      \\ \hline
{\ {\bf clrss}}    & {\ {Clear sky surf abs sol (${\rm W~m^{-2}}$)} }     \\ \hline
{\ {\bf clrlt}}    & {\ {Clear sky net up flux (${\rm W~m^{-2}}$)}}       \\ \hline
{\ {\bf clrls}}    & {\ {Clear sky LW surf cool (${\rm W~m^{-2}}$)} }      \\ \hline
{\ {\bf solin}}    & {\ {Instant incid solar (${\rm W~m^{-2}}$)} }     \\ \hline
{\ {\bf sabtp}}    & {\ {Column abs solar (${\rm W~m^{-2}}$) }}       \\ \hline
{\ {\bf firtp}}    & {\ {Net up LW flux at TOA (${\rm W~m^{-2}}$)}}       \\ \hline
\end{tabular}
\end{center}
\end{table}

\begin{table}[!]
\begin{center}
\caption{List of output variables from tracer model}  \label{che_var}
\vspace{0.25cm}
\begin{tabular}{|l|c|l|} \hline \hline
{\small {\bf Variables}} & {\small {\bf Description}} \\ \hline \hline
{\ {\bf trac}}        & {\ Tracer mixing ratio (${\rm kg~kg^{-1}}$)  }      \\ \hline
{\ {\bf aext8}}        & {\ aer mix. ext. coef }      \\ \hline
{\ {\bf assa8}}        & {\ aer mix. sin. scat. alb}      \\ \hline
{\ {\bf agfu88}}        & {\ aer mix. ass. par}      \\ \hline
{\ {\bf colb\_tr}}    & {\ Column burden (${\rm kg~m^{-2}}$)}  \\ \hline
{\ {\bf wdlsc\_tr}}    & {\ Wet deposition large-scale (${\rm kg~m^{-2}}$) }      \\ \hline
{\ {\bf wdcvc\_tr}}    & {\ Wet deposition convective (${\rm kg~m^{-2}}$)  }   \\ \hline
{\ {\bf sdrdp\_tr}}    & {\ Surface dry deposition (${\rm kg~m^{-2}}$ ) }  \\ \hline
{\ {\bf xgasc\_tr}}    & {\ chem gas conv. (${\rm mg/m2/d}$)  }   \\ \hline
{\ {\bf xaquc\_tr}}    & {\ chem aqu conv. (${\rm mg/m2/d}$ ) }  \\ \hline
{\ {\bf emiss\_tr}}    & {\ Surface emission (${\rm kg~m^{-2}}$) }       \\ \hline
{\ {\bf acstoarf}}      & {\ TOArad forcing av.(${\rm W~m^{-2}}$)}      \\ \hline
{\ {\bf agfu88}}     & {\ SRFrad forcing av. (${\rm W~m^{-2}}$)}      \\ \hline
\end{tabular}
\end{center}
\end{table}

The source code for the model is in the {\bf RegCM/Main} sub-directory.  The 
{\bf RegCM/Commons} sub-directory contains two files necessary for starting a new 
simulation ({\it regcm.in} and {\it regcm.x}).  The physics options discussed in 
Section~\ref{sec:physics}, as well as the date, timestep,  
output frequency, ect. parameters in Table~\ref{regcm_file2} are selected in the 
{\it regcm.in} file.  

\subsection{Selecting the appropriate time steps}
There are some general rules to follow when selecting the appropriate time steps
for your simulation.  The following time step parameters are defined in 
the {\it regcm.in} file, \\

{\bf radfrq} - time step for radiation model in minutes \\
\indent
{\bf abemh} - time step for LW absorption/emissivity in hours \\
\indent
{\bf abatm} - time step for land surface model in seconds \\
\indent
{\bf dt} - time step for atmosphere model in seconds\\

\noindent
First, the time step for the atmosphere model ({\bf dt}) should be about 3 times the 
horizontal resolution of your domain in km.  So if your resolution is 60~km then 
{\bf dt} should be about 180~seconds.  Here we can increase the time step a little to 
200~seconds.  Increasing the time step will decrease the run time for the simulation
but be careful because if your time step is too large the model will crash.
Then {\bf radfrq}, {\bf abemh}, and {\bf abatm} all need to be divisible by {\bf dt}.  
In this case, setting  {\bf radfrq} to 30~minutes, {\bf abemh} to 18~minutes, and {\bf abatm} 
to 540~seconds would be reasonable.  See Table~\ref{timestep} for more examples of 
time steps for different horizontal resolutions.


\begin{table}
\begin{center}
\caption{Time steps with different resolutions} \label{timestep}
\begin{tabular}{ccccc}
 dx(km) & dt(sec) & abatm(sec) & abemh(hr) & radfrq(min) \\
\hline\hline
10 & 30  & 90 & 18 & 30 \\
20 & 60  & 120 & 18 & 30 \\
30 & 100  & 300 & 18 & 30 \\
45 & 150 & 300 & 18 & 30 \\
50 & 150 & 450 & 18 & 30 \\
60 & 200 & 600 & 18 & 30 \\
90 & 225 & 900 & 18 & 30 \\
\hline\hline
\end{tabular}
\end{center}
\end{table}


\subsection{Starting the simulation}
The {\it regcm.x} script will compile and execute the model.  It is recommended to create a new
directory for specific projects and to copy these two files into this new project 
directory.  Running the script will: 

$\bullet$  Create soft links to the domain file and initial and boundary conditions
files. 

\indent \indent fort.10 $\rightarrow$ {\bf ../Input/}{\it DOMAIN}

\indent \indent fort.10x $\rightarrow$ {\bf ../Input/}{\it ICBCYYYYMMDDHH} 

$\bullet$  Create the sub-directory {\bf output} where the model output files are written. 

$\bullet$  Create the {\it postproc.in} file which will be needed for postprocessing the 
output files -- this is discussed in the next section. 

$\bullet$  Compile the source code and start the simulation. \\


\noindent Running the model generates the following monthly output files, 

\indent Atmospheric model output (see Table~\ref{atmos_var}): {\it ATM.YYYYMMDDHH} 

\indent Land surface model output (see Table~\ref{lsm_var}): {\it SRF.YYYYMMDDHH}  

\indent Radiation model output (see Table~\ref{rad_var}):  {\it RAD.YYYYMMDDHH}  

\indent Chemistry model output (see Table~\ref{che_var}) (if the chemistry model is run): {\it CHE.YYYYMMDDHH}  

\indent Restart file: {\it SAVTMP.YYYYMMDDHH} or {\it SAV.YYYYMMDDHH}


\subsection{Restarting a simulation}
You can use the restart option if your simulation crashes or you want to restart the model from where your previous simulation ended.  The model saves an output file  necessary 
to restart a simulation every month in the {\bf output} subdirectory ({\it SAV.YYYYMMDDHH}). 
In the event of crashes, the model also saves temporary files more frequently in your working 
directory ({\it SAVTMP.YYYYMMDDHH}).  To restart a simulation, simply change the 
``ifrest''' parameter to true in the {\it regcm.in} file and if needed modify the date 
parameters.  You will also need to create a soft link from the appropriate 
{\it SAV.YYYYMMDDHH} file to a file named fort.14 in your working directory 
(note:  {\it YYYYMMDDHH }should match the date that you want to begin restarting 
the simulation).  Depending on your simulations, you may also need to create 
new ICBC files and modify the links in the {\it regcm.x} script.


\newpage
\section{Post-processing}
The model generates three output files every month in your {\bf output} subdirectory \\

$\bullet$  {\it ATM.YYYYMMDDHH} from the atmospheric model (see Table~\ref{atmos_var} 
for list of variables)

$\bullet$  {\it SRF.YYYYMMDDHH} from the land surface model (see Table~\ref{lsm_var} 
for list of variables)

$\bullet$  {\it RAD.YYYYMMDDHH} from the radiation model (see Table~\ref{rad_var} 
for list of variables)

If you have run the chemistry model, you will also have an additional output file.

$\bullet$  {\it CHE.YYYYMMDDHH} from the chemistry model (see Table~\ref{che_var} 
for list of variables)\\

The RegCM postprocessor converts these model output files to new output files of 
averaged variables in commonly used formats such as NetCDF or GrADS.  You will 
need to modify the {\it postproc.in} file in your working directory to specify how
to average the variables (daily, monthly, ect) and the file format.  Then run 
the {\it postproc.x} script which will compile and execute the program.

\subsubsection{Converting sigma-level data to pressure levels}
Often we want to look at our output on pressure levels instead of sigma levels. We provide a conversion program that creates a GrADS-format data file. SIGMAtoP.f is located in RegCM/Commons/tools. Compiling instructions are given in the top two lines of the file. Before compiling and running it, you must edit the following fields in SIGMAtoP.f. 

iy,jx,kx: grid dimensions (should match dimensions in OUT HEAD.CTL, not DOMAIN INFO.CTL)

np: number of pressure levels

plev: set specific pressure levels that you want to create, in hPa (the total number should match np)

nfile: number of ATM files that you want to process

data inout: names of the ATM files that you want to process

data number: number of time slices in each ATM file


A sample .ctl file for the converted data is also available in RegCM/Commons/tools: PLEV VAR.ctl. To use it, simply edit the pdef, xdef, ydef, zdef, and tdef lines according to your domain specifications. You can copy the pdef, xdef, and ydef lines from your OUT HEAD.CTL file. The zdef line should contain the same number of pressure levels that you set as "np" in SIGMAtoP.f. Tdef should be set as the total number of time slices in the output file (the sum of data number - so if data number was set to /20,40/, then you would have 60 total time slices in the file). Replace the start time (06z01Jul1994 in the example) with the start time of the data. You should now be able to look at the converted data in GrADS.


\subsection{Observational Data Interpolator}

In the RegCM/Obs directory, we provided scripts for interpolating several observed data sets to your RegCM grid to facilitate comparisons with observations. 

One often-used data set is the Climate Research Unit (CRU) High Resolution Global Data, which is a global, land only data set available at 0.5 degree resolution. The following monthly-mean variables are available: precipitation, cloud cover, diurnal temperature range, daily maximum temperature, daily minimum temperature, temperature, vapor pressure, wet day frequency, and frost day frequency. Information on CRU datasets is available at http://www.cru.uea.ac.uk/cru/data/

Another precipitation data set is the CPC Merged Analysis of Precipitation (CMAP), which is a 2.5 degree resolution global data set with coverage over land and ocean. Data are available from 1979 to the near present. Data are available as monthly means and pentads and can be downloaded and viewed at the CDC web site, http://www.cdc.noaa.gov/cdc/data.cmap.html. Documentation and guidance on their usage can be found in the original references, \citet{XieArkin_96,XieArkin_97}.

A third source of data are the global precipitation and temperature fields from the University of Delaware at http://climate.geog.udel.edu/~climate/. Most data are available for 1950-1999.

